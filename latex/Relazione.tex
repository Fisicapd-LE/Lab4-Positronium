\documentclass[6pt,a4paper, mathserif]{article} % Prepara un documento con un font grande

\usepackage{iftex}

\ifLuaTeX
  \usepackage{tikz}
\usepackage{pgfplots}

\usepackage{lipsum} % Package to generate dummy text throughout this template

\usepackage{fontspec}
\setmainfont[Ligatures=TeX]{Alegreya}

%\usepackage[sc]{mathpazo} % Use the Palatino font
%\usepackage[T1]{fontenc} % Use 8-bit encoding that has 256 glyphs
%%%%%
%\usepackage{Alegreya} %% Option 'black' gives heavier bold face 
%\renewcommand*\oldstylenums[1]{{\AlegreyaOsF #1}}

%\usepackage[euler-digits,euler-hat-accent]{eulervm}
%%%%%%
%\usepackage[utf8]{inputenc} % Consente l'uso caratteri accentati italiani
%\linespread{1.05} % Line spacing - Palatino needs more space between lines
\usepackage{amsmath, amsthm, amssymb, amsfonts}
\usepackage{microtype} % Slightly tweak font spacing for aesthetics

%%%%%%%%%%%%%%%%%%%%%%%%%%%%%%%%%%%%%%%%%%%%%
%Miei package
\usepackage[italian]{babel}
\usepackage{graphicx}		% Per le immagini

\usepackage{tabularx}		% Per le tabelle con le colonne tutte uguali
\usepackage{tabulary}		% Tabelle migliorate, nelle celle il testo va a capo da solo...
%%%%%%%%%%%%%%%%%%%%%%%%%%%%%%%%%%%%%%%%%%%%%
\usepackage[
	    %hmargin=0.18\paperwidth,% metti la larghezza del testo (margini orizzontali) al 18% del foglio
	    %textwidth=3.1\alphabet,  % http://tex.stackexchange.com/questions/59626/nicely-force-66-characters-per-line
	    hmarginratio=1:1,       % margini destro e sinistro uguali
	    top=35mm,	            % margine sopra a 32mm...
	    vmarginratio=4:5,       % quello sotto uguale (default 2:3)
	    columnsep=20pt]         % Spazio tra le colonne]
	    {geometry} % Document margins
\usepackage{multicol} % Used for the two-column layout of the document
\usepackage[hang, small,labelfont=bf,up,textfont=it,up]{caption} % Custom captions under/above floats in tables or figures
\usepackage{booktabs} % Horizontal rules in tables
\usepackage{float} % Required for tables and figures in the multi-column environment - they need to be placed in specific locations with the [H]
\usepackage[titles]{tocloft} %Pare causi meno casini con fancyhdr
\usepackage{nicefrac} % Per le frazioni tipo ⅛
\usepackage{pdfpages} % Per includere pagine intere in pdf (per la copertina)

\usepackage{lettrine} % The lettrine is the first enlarged letter at the beginning of the text
\usepackage{paralist} % Used for the compactitem environment which makes bullet points with less space between them

\usepackage{abstract} % Allows abstract customization
\renewcommand{\abstractnamefont}{\normalfont\bfseries} % Set the "Abstract" text to bold
\renewcommand{\abstracttextfont}{\normalfont\small\itshape} % Set the abstract itself to small italic text

\usepackage{listingsutf8} % Per includere codice sorgente meglio che con verbatim (e con caratteri non inglesi)
\lstset{ 
  %Preso anche questo da http://en.wikibooks.org/wiki/LaTeX/Source_Code_Listings
  %backgroundcolor=\color{white},   % choose the background color; you must add \usepackage{color} or \usepackage{xcolor}
  basicstyle=\footnotesize\ttfamily,        % the size of the fonts that are used for the code E MESSO IN MONOSPACE
  breakatwhitespace=true,         % sets if automatic breaks should only happen at whitespace
  breaklines=true,                 % sets automatic line breaking
  captionpos=b,                    % sets the caption-position to bottom
  %commentstyle=\color{mygreen},    % comment style
  %deletekeywords={...},            % if you want to delete keywords from the given language
  %escapeinside={\%*}{*)},          % if you want to add LaTeX within your code
  %extendedchars=true,              % lets you use non-ASCII characters; for 8-bits encodings only, does not work with UTF-8
  frame=l,                    % adds a frame around the code
				    %you can control the rules at the top, right, bottom, and left directly by using the four initial 
				    %letters for single rules and their upper case versions for double rules. http://mirror.hmc.edu/ctan/macros/latex/contrib/listings/listings.pdf
				    % Es frame frame=trBL ha doppia linea a sinistra e sotto, e singola a destra e sopra
  keepspaces=true,                 % keeps spaces in text, useful for keeping indentation of code (possibly needs columns=flexible)
  %keywordstyle=\color{blue},       % keyword style
  %language=Octave,                 % the language of the code
  %morekeywords={*,...},            % if you want to add more keywords to the set
  numbers=left,                    % where to put the line-numbers; possible values are (none, left, right)
  numbersep=5pt,                   % how far the line-numbers are from the code
  %numberstyle=\tiny\color{mygray}, % the style that is used for the line-numbers
  %rulecolor=\color{black},         % if not set, the frame-color may be changed on line-breaks within not-black text (e.g. comments (green here))
  showspaces=false,                % show spaces everywhere adding particular underscores; it overrides 'showstringspaces'
  showstringspaces=false,          % underline spaces within strings only
  showtabs=false,                  % show tabs within strings adding particular underscores
  stepnumber=1,                    % the step between two line-numbers. If it's 1, each line will be numbered
  %stringstyle=\color{mymauve},     % string literal style
  tabsize=2,                       % sets default tabsize to 2 spaces
  title=\lstname                   % show the filename of files included with \lstinputlisting; also try caption instead of title
}

\usepackage{titlesec} % Allows customization of titles
%\renewcommand\thesection{\Roman{section}} % Roman numerals for the sections
%\renewcommand\thesubsection{\Roman{subsection}} % Roman numerals for subsections
% \usefont {encoding} {family} {series} {shape}
%\titleformat{\section}[block]{\AlegreyaSansSC \bfseries \LARGE}{\thesection.}{1em}{} % Change the look of the section titles. Pezzi spostati \scshape\centering\bfseries
%\titleformat{\subsection}[block]{\AlegreyaSans \bfseries \Large}{\thesection.\thesubsection }{1em}{} % Change the look of the section titles

\usepackage{fancyhdr} % Headers and footers
\pagestyle{fancy} % All pages have headers and footers
\fancyhead{} % Blank out the default header
\fancyfoot{} % Blank out the default footer
\headheight=14pt % Perchè sennò continua a lamentarsi che 12pt è troppo poco e la mette a 14 lo stesso
%\fancyhead[C]{Chiappara, Labanca, Forcher - \textit{Ottica geometrica} $\bullet$ \thesection}
\fancyhead[L]{\textit{Lusiani Enrico}} % Custom header text. \nouppercase{\leftmark} per sezione, ma non ci sta
\fancyhead[R]{\textsc{\nouppercase{\leftmark}}}
\fancyfoot[RO,LE]{\thepage} % Custom footer text 

\usepackage[hidelinks]{hyperref} % For hyperlinks in the PDF
\hypersetup{
    bookmarks=true,         % show bookmarks bar?
    unicode=false,          % non-Latin characters in Acrobat’s bookmarks
   % pdftoolbar=true,        % show Acrobat’s toolbar?
   % pdfmenubar=true,        % show Acrobat’s menu?
   % pdffitwindow=false,     % window fit to page when opened
    %pdfstartview={FitH},    % fits the width of the page to the window
    pdftitle={Relazione di Elettronica},    % title
    pdfauthor={E. Lusiani},     % author
    pdfsubject={Tesi di Laurea - Anno accademico 2015-2016 - },   % subject of the document
    %pdfcreator={Creator},   % creator of the document
    %pdfproducer={Producer}, % producer of the document
    pdfkeywords={photomultipliers} {tesi} {fotomoltiplicatori} {JUNO}, % list of keywords
    %pdfnewwindow=true,      % links in new PDF window
    %colorlinks=false,       % false: boxed links; true: colored links
    linkcolor=red,          % color of internal links (change box color with linkbordercolor)
    citecolor=green,        % color of links to bibliography
    filecolor=magenta,      % color of file links
    urlcolor=cyan           % color of external links
}

\usepackage{etoolbox}

\usepackage{textgreek}

\usepackage{standalone}

\usepackage{circuitikz}

\usepackage{siunitx}

\usepackage{subcaption}


\else
  \usepackage{tikz}
\usepackage{pgfplots}

\usepackage{lipsum} % Package to generate dummy text throughout this template

%\usepackage[sc]{mathpazo} % Use the Palatino font
\usepackage{tgpagella} % TeX Gyre Pagella, versione migliorata di Palatino. Si ma bo, no
\usepackage{inconsolata} % Font monospace
%\usepackage{textcomp}
\usepackage[scale=0.98,ttdefault]{AnonymousPro}

%%%%%
%\usepackage{Alegreya} %% Option 'black' gives heavier bold face 
\usepackage{AlegreyaSans} %% Option 'black' gives heavier bold face 
%\renewcommand*\oldstylenums[1]{{\AlegreyaOsF #1}}
%\usepackage{opensans}
%\usepackage[euler-digits,euler-hat-accent]{eulervm}
\usepackage[euler-hat-accent]{eulervm}
%%%%%%

\usepackage[T1]{fontenc} % Use 8-bit encoding that has 256 glyphs
%\usepackage{fontspec}
%\setmainfont{tgpagella}
%\setsansfont{AlegreyaSans}
%\setmonofont{inconsolata}

\usepackage[utf8]{inputenc} % Consente l'uso caratteri accentati italiani
\linespread{1.08} % Line spacing - Palatino needs more space between lines, messo a 1.08 da 1.11 che era per alegreya
\usepackage{amsmath, amsthm, amssymb, amsfonts}
\usepackage[italian]{babel}
%\usepackage[kerning,spacing,tracking,letterspace = 2,babel]{microtype} % Slightly tweak font spacing for aesthetics. Il tre è pensato per Alegreya
\usepackage[kerning,spacing,babel]{microtype}
\SetTracking[]{encoding = *,shape = *}{3} % Aumenta la distanza fra le lettere
							     % http://tex.stackexchange.com/questions/66494/new-command-for-spacing-letters-in-microtype
%%%%%%%%%%%%%%%%%%%%%%%%%%%%%%%%%%%%%%%%%%%%%
%Miei package



\usepackage{graphicx}		% Per le immagini
%\usepackage{color}		% COLORI!

%\definecolor{grigio-molto-scuro}{gray}{0.1}	%colore

\usepackage{tabularx}		% Per le tabelle con le colonne tutte uguali
\usepackage{tabulary}		% Tabelle migliorate, nelle celle il testo va a capo da solo...
%%%%%%%%%%%%%%%%%%%%%%%%%%%%%%%%%%%%%%%%%%%%%
\newlength{\alphabet}
\settowidth{\alphabet}{\normalfont abcdefghijklmnopqrstuvwxyz}
\usepackage[
	    %hmargin=0.18\paperwidth,% metti la larghezza del testo (margini orizzontali) al 18% del foglio
	    textwidth=3.3\alphabet,  % http://tex.stackexchange.com/questions/59626/nicely-force-66-characters-per-line
	    hmarginratio=1:1,       % margini destro e sinistro uguali
	    top=35mm,	            % margine sopra a 32mm...
	    vmarginratio=4:5,       % quello sotto uguale (default 2:3)
	    columnsep=20pt]         % Spazio tra le colonne?
	    {geometry} % Document margins
\usepackage{multicol} % Used for the two-column layout of the document
\usepackage[hang, small,labelfont=bf,up,textfont=it,up]{caption} % Custom captions under/above floats in tables or figures
\usepackage{booktabs} % Horizontal rules in tables
\usepackage{float} % Required for tables and figures in the multi-column environment - they need to be placed in specific locations with the [H] (e.g. \begin{table}[H])
%\usepackage{tocloft} % Per customizzare le liste di floats (per i custom float!)
\usepackage[titles]{tocloft} %Pare causi meno casini con fancyhdr
\usepackage{nicefrac} % Per le frazioni tipo ⅛
\usepackage{pdfpages} % Per includere pagine intere in pdf (per la copertina)

\usepackage{lettrine} % The lettrine is the first enlarged letter at the beginning of the text
\usepackage{paralist} % Used for the compactitem environment which makes bullet points with less space between them
\usepackage[section]{placeins} % Per \FloatBarrier. L'opzione section comporta che le sezioni siano floatbarriers

\usepackage{abstract} % Allows abstract customization
\renewcommand{\abstractnamefont}{\normalfont\bfseries} % Set the "Abstract" text to bold
\renewcommand{\abstracttextfont}{\normalfont\small\itshape} % Set the abstract itself to small italic text

\usepackage{caption} % Per captions avanzate

\usepackage{listingsutf8} % Per includere codice sorgente meglio che con verbatim (e con caratteri non inglesi)
\lstset{ 
  %Preso anche questo da http://en.wikibooks.org/wiki/LaTeX/Source_Code_Listings
  %backgroundcolor=\color{white},   % choose the background color; you must add \usepackage{color} or \usepackage{xcolor}
  basicstyle=\footnotesize\ttfamily,        % the size of the fonts that are used for the code E MESSO IN MONOSPACE
  breakatwhitespace=true,         % sets if automatic breaks should only happen at whitespace
  breaklines=true,                 % sets automatic line breaking
  captionpos=b,                    % sets the caption-position to bottom
  %commentstyle=\color{mygreen},    % comment style
  %deletekeywords={...},            % if you want to delete keywords from the given language
  %escapeinside={\%*}{*)},          % if you want to add LaTeX within your code
  %extendedchars=true,              % lets you use non-ASCII characters; for 8-bits encodings only, does not work with UTF-8
  frame=l,                    % adds a frame around the code
				    %you can control the rules at the top, right, bottom, and left directly by using the four initial 
				    %letters for single rules and their upper case versions for double rules. http://mirror.hmc.edu/ctan/macros/latex/contrib/listings/listings.pdf
				    % Es frame frame=trBL ha doppia linea a sinistra e sotto, e singola a destra e sopra
  keepspaces=true,                 % keeps spaces in text, useful for keeping indentation of code (possibly needs columns=flexible)
  %keywordstyle=\color{blue},       % keyword style
  %language=Octave,                 % the language of the code
  %morekeywords={*,...},            % if you want to add more keywords to the set
  numbers=left,                    % where to put the line-numbers; possible values are (none, left, right)
  numbersep=5pt,                   % how far the line-numbers are from the code
  %numberstyle=\tiny\color{mygray}, % the style that is used for the line-numbers
  %rulecolor=\color{black},         % if not set, the frame-color may be changed on line-breaks within not-black text (e.g. comments (green here))
  showspaces=false,                % show spaces everywhere adding particular underscores; it overrides 'showstringspaces'
  showstringspaces=false,          % underline spaces within strings only
  showtabs=false,                  % show tabs within strings adding particular underscores
  stepnumber=1,                    % the step between two line-numbers. If it's 1, each line will be numbered
  %stringstyle=\color{mymauve},     % string literal style
  tabsize=2,                       % sets default tabsize to 2 spaces
  title=\lstname                   % show the filename of files included with \lstinputlisting; also try caption instead of title
}


\usepackage{titlesec} % Allows customization of titles
%\renewcommand\thesection{\Roman{section}} % Roman numerals for the sections
%\renewcommand\thesubsection{\Roman{subsection}} % Roman numerals for subsections
% \usefont {encoding} {family} {series} {shape}
%\AlegreyaSansSC
\titleformat{\section}[block]{ \bfseries \LARGE}{\thesection.}{1em}{} % Change the look of the section titles. Pezzi spostati \scshape\centering\bfseries
\titleformat{\subsection}[block]{\bfseries \Large}{\thesection.\thesubsection }{1em}{} % Change the look of the section titles

\usepackage{fancyhdr} % Headers and footers
\pagestyle{fancy} % All pages have headers and footers
\fancyhead{} % Blank out the default header
\fancyfoot{} % Blank out the default footer
\headheight=14pt % Perchè sennò continua a lamentarsi che 12pt è troppo poco e la mette a 14 lo stesso
%\fancyhead[C]{Chiappara, Labanca, Forcher - \textit{Ottica geometrica} $\bullet$ \thesection}
\fancyhead[L]{\textit{Gruppo 8}} % Custom header text. \nouppercase{\leftmark} per sezione, ma non ci sta
\fancyhead[R]{\textsc{\nouppercase{\leftmark}}}
\fancyfoot[RO,LE]{\thepage} % Custom footer text

\usepackage[hidelinks]{hyperref} % For hyperlinks in the PDF
\hypersetup{
    bookmarks=true,         % show bookmarks bar?
    unicode=false,          % non-Latin characters in Acrobat’s bookmarks
   % pdftoolbar=true,        % show Acrobat’s toolbar?
   % pdfmenubar=true,        % show Acrobat’s menu?
   % pdffitwindow=false,     % window fit to page when opened
    %pdfstartview={FitH},    % fits the width of the page to the window
    pdftitle={Relazione di laboratorio: Positronio},    % title
    pdfauthor={D.Chiappara, I. Di Terlizzi, E. Lusiani},     % author
    pdfsubject={Relazione di laboratorio},   % subject of the document
    %pdfcreator={Creator},   % creator of the document
    %pdfproducer={Producer}, % producer of the document
    %pdfkeywords={photomultipliers} {tesi} {fotomoltiplicatori} {JUNO}, % list of keywords
    %pdfnewwindow=true,      % links in new PDF window
    %colorlinks=false,       % false: boxed links; true: colored links
    linkcolor=red,          % color of internal links (change box color with linkbordercolor)
    citecolor=green,        % color of links to bibliography
    filecolor=magenta,      % color of file links
    urlcolor=cyan           % color of external links
}

\usepackage{etoolbox}

\usepackage{textgreek}

\usepackage{standalone}

%\usepackage{circuitikz}

\usepackage{siunitx}

\usepackage{subcaption}

\usepackage{multirow} %added by Davide for multirow tables in positronium report

%\extrafloats{100}

\fi
\DeclareGraphicsExtensions{.pdf, .png, .jpg} % Se due immagini hanno lo stesso nome sceglile secondo l'ordine di filetype qui
\graphicspath{ {../img/nofloat/} }					 % Path delle immagini 

%%%%%%%%%%%%%%%%%%%%%%%%%%%%%%%%%%%%%%5%%%%%%%%%%%%%%%%%%%%%%%%%%%%%%%%%%%%%%%%%%%
%\usepackage{float}
%\usepackage{caption}
%\usepackage{multirow}
%\usepackage[top=3.6cm, bottom=1.5in, left=0.5in, right=0.5in]{geometry}

%%%%%%%%%%%%%%%%%
% Robe del package tocloft per fare gli indici delle mie tabelle e grafici
% texblog.org/2008/07/13/define-your-own-list-of/
%\newcommand{\listtabellaname}{Lista delle tabelle}
%\newlistof{tabella}{tab}{\listtabellaname}


%%%%%%%%%%%%%%%%%

% I miei stili di float, con le righe
\floatstyle{plaintop}
\newfloat{tabella}{tb}{lop} 
\floatname{tabella}{Tabella}

\floatstyle{ruled}
\newfloat{grafico}{tb}{loi} 
\floatname{grafico}{Grafico}

\newcommand{\tabellaautorefname}{\bfseries Tabella} % per \autoref del package hyperref
\newcommand{\graficoautorefname}{\bfseries Grafico} % Idem



%%%%%%%%%%%%%%%%%%%%%%%%%%%%%%%%%%%%%%%%%%%%%%%%%%%%%%%%%%%%%%%%%%%%%5%%%%%%%%%%%%%
% Comandi personalizzati

% \newcommand{\cm}{\,\mathrm{cm}}
% \DeclareMathOperator{\cov}{cov} % Covarianza
% \DeclareMathOperator{\var}{var} % Covarianza
% \newcommand{\mm}{\,\mathrm{mm}}
% \newcommand{\nm}{\,\mathrm{nm}}
% \newcommand{\usuq}{\nicefrac{1}{q}}
% \newcommand{\usup}{\nicefrac{1}{p}}


\newtoggle{draft}
\togglefalse{draft}

\newcommand{\toprof}[1]
{
  \iftoggle{draft}{\emph{#1}}{}
}

\newcommand{\dd}{\mathop{}\,\mathrm{d}}

\newcommand\blankpage{%
    \null
    \thispagestyle{empty}%
    \addtocounter{page}{-1}%
    \newpage}
    
\newcommand{\isotope}[2]
{%
	$^{#2}\mathrm{#1}$%
}



%////////////////////////////////////////////////////////////////////////////////////////////////////////////////////////////
%////////////////////////////////////////////////////////////////////////////////////////////////////////////////////////////
% Fine dei dati iniziali per il latex: il documento finale inizierà da qui
\begin{document}


\begin{titlepage}

\begin{center}
\LARGE{Università degli Studi di Padova}\\
\line(1,0){450}\\
\vspace{1em}
\Huge{\textsc{\textbf{Relazione di laboratorio:\\ positronio}}}\\
\vspace{2em}
\LARGE{\textit{Laboratorio di fisica, primo anno LM}}\\
\vspace{4em}
\huge{\textbf\textsc\textit{{{Davide Chiappara}}}}\\
\vspace{0.5em}
\normalsize{Università di Padova, facoltà di fisica,}\\
\normalsize{davide.chiappara@studenti.unipd.it}\\
\normalsize{Matricola: 1153465}\\
\vspace{1em}
\huge{\textbf\textsc\textit{{{Ivan Di Terlizzi}}}}\\
\vspace{0.5em}
\normalsize{Università di Padova, facoltà di fisica,}\\
\normalsize{ivan.diterlizzi@studenti.unipd.it}\\
\normalsize{Matricola: 1155188}\\
\vspace{1em}
\huge{\textbf\textsc\textit{{{Enrico Lusiani}}}}\\
\vspace{0.5em}
\normalsize{Università di Padova, facoltà di fisica,}\\
\normalsize{enrico.lusiani@studenti.unipd.it}\\
\normalsize{Matricola: 1153399}\\
%\vspace{7em}
\vfill
\line(1,0){450}\\
\LARGE{Anno accademico 2016-2017}
\end{center}

\end{titlepage}



{
  %\color{grigio-molto-scuro}
  %\lsstyle % Abilita il letterspacing personalizzato
  %\unclfamily % Cagata per il font Uncial
  \vspace{ \stretch{1}
}

%\blankpage


\vspace{ \stretch{1} }
\noindent
\begin{abstract}
La seguente è la relazione sull'esperimento di decadimento del positronio eseguito da Chiappara Davide, Di Terlizzi Ivan e Lusiani Enrico facenti parte del gruppo 8. I dati sono
stati raccolti presso il laboratorio di fisica in via Loredan in data 14-15-16 Novembre 2016, e sono stati successivamente analizzati durante lo stesso anno accademico.\\
L'esperienza consiste nello studio del decadimento del positrone prodotto da un atomo di $^{22}Na$ con la produzione di orto-positronio o di para-positronio, andando a verificare
il rapporto tra il decadimento a due fotoni e quello a tre fotoni e andando a misurare la distribuzione temporale degli eventi.
\end{abstract}

\blankpage


\microtypesetup{protrusion=false} % disables protrusion locally in the document
\tableofcontents % prints Table of Contents
%\listoftabella
%\listoffigures
%\listoftables
\microtypesetup{protrusion=true} % enables protrusion

\blankpage

%chapters

 

\FloatBarrier


\section{Esecuzione esperimento}
L'apparato strumentale consiste in quattro rivelatori a scintillazione NaI(Tl), di cui uno montato verticalmente sopra il supporto per la sorgente di $^{22}\mathrm{Na}$ e tre
montati orizzontalmente su un goniometro a bracci che ne permette lo spostamento. Inoltre si ha accesso ad una serie di moduli elettronici (un amplificatore ad alto voltaggio,
un fan in-out, un CFTD, uno shaping amplifier, una concidence unit, un TAC, una delay unit e un modulo scaler rate). Per quanto riguarda l'acquisizione, si ha accesso
ad un oscilloscopio e ad un ADC.\\

Il primo giorno ci si è focalizzati sullo studio dei segnali dei rivelatori e sulla calibrazione dell'apparato. Come prima cosa, si è visualizzato il segnale di ogni
rivelatore, separatamente, sull'oscilloscopio, e si è caratterizzato tale segnale.
Dopo aver preso nota di tali segnali si è collegato all'amplificatore il rivelatore, e si sono guardati i segnali sull'oscilloscopio, riconoscendo quali di questi fossero
associati a segnali del tipo fotone da 511 keV e quali invece
fotone da 1275 keV. Per fare questo si è semplicemente utilizzato il rate indicato dall'oscilloscopio stesso in fase di visualizzazione del segnale. Infatti, alzando il
trigger dell'oscilloscopio, si vedono solo i segnali più energetici, e quindi si vede la rate diminuire di netto quando si supera l'ampiezza corrispondente ai segnali di fotopicco; questo ha permesso di vedere effettivamente quale fosse l'ampiezza
caratteristica dei segnali associati ai diversi fotoni emessi dalla sorgente. Tale operazione è stata ripetuta per tutti e 4 i rivelatori per verificare che non ci fossero
problemi di alcuna natura nella rivelazione dei fotoni emessi dal sodio. Dopo aver visualizzato i segnali in uscita dall'amplificatore si è voluto comprendere il funzionamento
del CFTD: per farlo si è utilizzata un'uscita del fan in-out come input e si è visualizzata sull'oscilloscopio l'uscita prompt e, triggherando su di essa, l'uscita delayed,
dopodiché si è visto il funzionamento dei vari micro-switch andando a modificarli e guardando il segnale all'oscilloscopio. Successivamente si è settata la soglia di tutti
e quattro i rivelatori: per i tre rivelatori complanari (quindi i rivelatori 1, 2 e 3) si è impostata la soglia in modo da non vedere il rumore elettronico, mentre per
il quarto rivelatore la soglia è stata impostata in modo da non vedere nemmeno il fotone da 511 keV. Per fare questo si è collegato all'oscilloscopio l'uscita
amplificata del rivelatore e si è triggherato il segnale sul segnale prompt del CFTD con input sempre su un'uscita del rivelatore stesso. Andando a modificare un trimmer
sul CFTD stesso si è potuto vedere come alzando la soglia scomparissero dal rivelatore via via i segnali meno energetici. Continuando ad agire su tale trimmer è stato
possibile rimuovere il rumore elettronico per i tre rivelatori complanari e il fotone da 511 keV per il quarto (si era precedenemente riconosciuta l'ampiezza del segnale
legato alla rivelazione di tale fotone). Una volta compreso il funzionamento dei vari moduli e impostata la soglia del CFTD in modo che potesse visualizzare il segnale
a cui si è interessati, si è passati alla vera e propria calibrazione dei rivelatori. Per farlo si è attaccato il rivelatore in questione all'amplificatore e quest'ultimo
al sistema di acquisizione digitale e si è dato a tale sistema il trigger utilizzando l'uscita delayed del CFTD a cui è stato attaccato lo stesso rivelatore (per poter
calibrare è stata abbassata la soglia di tutti i rivelatori precedentemente impostata, poi è stata ripristinata per le misure successive alla calibrazione). Fatto questo si
sono cambiati i parametri dell'amplificatore in modo che i due fotopicchi fossero visualizzati il primo circa al canale 500 e il secondo circa al canale 1300. Dopo aver
calibrato, è risultato più semplice agire sul trimmer per la regolazione della soglia del quarto rivelatore (infatti è stato sufficiente alzare la soglia fino alla scomparsa
del primo fotopicco).
Sistemate tutte le soglie come richiesto si è passati alla vera e propria presa dati che sono stati successivamente analizzati, e per preparare l'apparato si sono
regolate le width in modo da essere circa di 100~ns e si sono messi i delay in modo che ci fosse sovrapposizione tra i segnali dei rivelatori 1 e 2 e ci fosse un
ritardo di circa 20~ns tra il segnale del rivelatore 1 e quello del rivelatore 2. Successivamente si è verificata la coincidenza tra i segnali del rivelatore 1 e del
rivelatore 3 e si è agito sui microswitch di quest'ultimo per fare in modo che questa fosse buona.\\

Durante la seconda sessione di laboratorio si sono presi i dati, sfruttando l'apparato sistemato e studiato durante la sessione precedente. Come prima cosa si è
certificato che i rivelatori fossero nella configurazione 1, cioè con i rivelatori 1 e 2 collineari e il rivelatore 3 che forma un angolo di $\pi/3$ con il
rivelatore 2. Fatto questo, si è collegata la prima uscita della coincidence unit, cioè quella che genera un segnale quando arrivano in coincidenza i segnali
dal primo e dal secondo rivelatore al Master Gate del segnale di acquisizione. Poi, verificato che tutti i rivelatori fossero collegati all'ADC, si
è preso un file di prova andando a vedere se effettivamente si sono visti solamente i picchi a 511~keV. Successicamente a questo si è calibrato il TAC, andando ad aggiungere
i ritardi tramite l'apposita cassetta; si è poi passati alla vera e propria acquisizione dati. Come prima cosa si è collegato il modulo scaler/rate per avere una visione il tempo
reale del rate di acquisizione dei rivelatori. Per questa prima misura si è utilizzato come master trigger la coincidenza nel rivelatore 1 e nel rivelatore 2, in modo
che venissero registrati esclusivamente gli eventi in cui il positronio è decaduto lungo la linea formata dai rivelatori, e che poi si potessero selezionare via software
i dati in cui il primo fotone da 1275 keV è entrato nel quarto rivelatore. Infine si è preparato il sistema per la misura del decadimento del positronio in tre fotoni (cioè
con i tre rivelatori complanari distanziati da angoli di 120 gradi) e si è messo il master gate di acquisizione sulla coincidenza tra i tre rivelatori complanari. Tale
misura è stata fatta per tutta la durata tra la seconda e la terza sessione (cioè per circa 20 ore).\\

Durante la terza sessione di laboratorio si sono prese le misure fisiche dei rivelatori (in particolare si è guardato il datasheet e si è misurata la distanza tra
i rivelatori e la sorgente e le dimensioni fisiche dei rivelatori stessi), dopodiché si è presa un'ulteriore misura nel caso dei due rivelatori collineari e si è iniziata
l'analisi dati. 

\FloatBarrier


\section{Analisi dati}
\subsection{Analisi preliminare dei segnali}

Come prima cosa si è voluto caratterizzare il segnale in uscita direttamente dai rivelatori, oppure dal fan in-out. Per fare ciò si sono semplicemente collegati
all'oscilloscopio i rivelatori e si è preso nota delle caratteristiche del segnale stesso. I valori misurati sono riassunti nella Tabella \ref{tab:preliminare}.
Tutti i segnali sono risultati
di polarità negativa e si è misurata l'ampiezza di un segnale tipico, se ne è misurato il tempo di salita e il tempo di discesa (visti come i tempi impiegati dal segnale
per raggiungere dal 10\% al 90\% dell'ampiezza massima e viceversa) e si è misurato il rumore utilizzando una latenza di 5 s per il segnale. Alle prime misure si è
associato un errore calcolato come errore dati i parametri costruttivi dell'oscilloscopio, mentre al rumore non si è voluto associare un errore in quanto si è interessati
solo all'entità di tale grandezza.
%
\begin{table}[h]
	\centering
	\input{tables/segnali_preliminari.tex}
	\caption{Misura preliminare dei segnali in uscita dai rivelatori}
	\label{tab:preliminare}
\end{table}
%
Collegando tra il rivelatore il fan in-out non si è vista alcuna differenza sostanziale nei segnali visti.
A scopo di semplice confronto dei rivelatori, si è voluto anche andare a leggere le ampiezze medie dei segnali in uscita dai rivelatori. Si è rivelato come, effettivamente, 
le ampiezze medie fossero tra loro molto vicine. Non si riportano i valori in quanto è stato fatto solamente un confronto qualitativo.\\

Andando a spostare il trigger dell'oscilloscopio è stato possibile andare ad indentificare le ampiezze caratteristiche dei segnali associati ai due diversi fotoni emessi dalla
sorgente di sodio. Alzando il trigger si è notato un primo abbassamento repentino nella rate appena si è superata la soglia del rumore elettronico, poi si è visto
un secondo abbassamento repentino superando il fotopicco da 511~keV e un terzo abbassamento molto rapido superato il fotopicco a 1275 keV, dopo il quale il rate si attestava
a qualche decina di Hertz, legati a fotoni di natura ambientale o cosmica.\\

\subsection{Calibrazione in energia dei rivelatori}

Successivamente si è passati alla vera e propria calibrazione dei quattro rivelatori che sono stati utilizzati, per effettuare la quale si è semplicemente fatta la fit
lineare dei centroidi dei fotopicchi rivelati tramite il programma di acquisizione. Nella Tabella \ref{tab:calibrazione} si possono vedere i valori dei parametri interpolanti
ottenuti, mentre si rimanda alle appendici per vedere le singole interpolazioni.\\
%
\begin{table}[h]
	\centering
	\input{tables/calibrazione.tex}
	\caption{Parametri dell'interpolazione dei singoli picchi degli spettri.}
	\label{tab:calibrazione}
\end{table}
%

Partendo da questi dati è stato possibile andare a stimare i parametri di calibrazione $m$ e $q$ (dato che si conoscono solo due punti per i quali questa retta passa,
non si è fatta una vera e propria interpolazione ma si sono calcolati i coefficienti delle rette passanti per tali punti, non computando alcun errore in quanto non necessario per la calibrazione dei grafici). I risultati si possono vedere nella Tabella \ref{tab:calib_parametri}.\\
%
\begin{table}[h]
	\centering
	\input{tables/calib_parametri.tex}
	\caption{Parametri di calibrazione dei rivelatori.}
	\label{tab:calib_parametri}
\end{table}
%

A scopo di presentare il lavoro fatto, nella Figura \ref{gr:quattrospettri} si possono vedere gli spettri calibrati di tutti e quattro i
rivelatori.\\
\input{./graphs/quattrospettri}
%Inserire gli spettri calibrati: 4 pad in un'unica canvas

Poi si è preso un primo file di prove con le coincidenze, e si è visto che effettivamente si sono visti solamente i picchi a 511~keV. Questi spettri si possono
vedere nella Figura \ref{gr:prova_coinc}. Nel prendere il file di prova non si era ancora regolata opportunamente la soglia. Effettivamente da questi spettri si vede che a meno di effetti ambientali si sono raccolti solamente i dati sul
primo fotopicco e la spalla Compton associata a tale picco.\\
\input{./graphs/prova_coinc}
%Inserire logli spettroi con due pad diversi che presentano il file di prova
\FloatBarrier

\subsection{Calibrazione in tempo del TAC}

Successivamente si è passati alla calibrazione temporale del sistema di acquisizione, andando ad utilizzare la cassetta dei ritardi per vedere lo spostamento del
segnale in uscita dal TAC al variare dei ritardi inseriti. Sono stati fatti dei fit gaussiani del segnale in uscita dal TAC e tutti i grafici trovati si possono
vedere nella Figura \ref{gr:tempo_colori}. Inoltre nella Tabella \ref{tab:calibtempo} si possono vedere i parametri interpolanti tali segnali
al variare del ritardo introdotto.\\
\input{./graphs/tempo_colori}
%inserire gaussiante temporali colorate
\begin{table}[h]
	\centering
	\input{tables/calibtempo.tex}
	\caption{Calibrazione in tempo del sistema di acquisizione.}
	\label{tab:calibtempo}
\end{table}

Una volta noti i parametri delle gaussiane, con una regressione lineare, che si può vedere nel Grafico \ref{gr:calib_time_retta}, si può andare a calibrare il sistema di
acquisizione, andando a trovare i coefficienti di tale retta. I risultati sono:
$$ m= 2.1256 \pm 0.0004\ 1/ns \hspace{3cm} q = 311.22 \pm 0.03$$\\
\input{./graphs/calib_time_retta}
%Inserire regressione lineare della calibrazione temporale dell'apparato strumentale


Una volta calibrato tutto il sistema di acquisizione, si è arrivati al vero e proprio studio del positronio: in particolare si vuole misurare il rapporto tra il decadimento
a due e a tre fotoni del positronio e la distribuzione temporale degli eventi.\\

\FloatBarrier
\subsection{Decadimento a due fotoni}

Si inizi dal decadimento a due fotoni: la misura è stata fatta andando a registrare gli spettri dei tre rivelatori complanari assieme allo spettro del TAC. Il
trigger è stato impostato sulla coincidenza tra R1 ed R2, ed a tale valore è stato impostato anche lo stop del TAC, mentre lo start del TAC è stato dato da R4 che visualizza
un segnale (si ricorda che la soglia di R4 è stata regolata in modo da vedere solo segnali di energia uguale o superiore al fotopicco da 1275~keV). Gli spettri calibrati che
si sono visti si possono vedere nella Figura \ref{gr:180_misura1_spettricalibrati_2}.\\
\input{./graphs/180_misura1_spettricalibrati_2}

Da questa Figura è possibile vedere come, effettivamente, negli spettri dei primi due rivelatori si abbiano solo i valori legati al primo fotopicco e all'effetto Compton ad
esso associato, mentre lo spettro del terzo rivelatore somiglia ad uno degli spettri inizialmente presentati. Per presentare questi spettri è stato rimosso il bin zero
dallo spettro del TAC, che è associato ad una coincidenza tra il primo ed il secondo rivelatore, ma quando il quarto rivelatore non ha visto il fotone da 1275~keV. A partire
da questi spettri si possono ricavare varie informazioni, come per esempio il rate di decadimento in due fotoni. Dato che tale spettro è stato regolato in modo che
l'acquisizione durasse un'ora, e il programma di acquisizione ha registrato 14399800 eventi, vuol dire che sono stati acquisiti in media eventi con una frequenza di 4~kHz.
Affinché sia possibile avere una stima attendibile del rate di acquisizione, però, è necessario considerare il dead time dei rivelatori: il modulo di conteggio ha mostrato
come il rate di acquisizione per i due rivelatori complanari fosse di circa 15 kHz, e il rate di coincidenze fosse di circa 5.5 kHz. Quindi ci aspettiamo l'ADC abbia
registrato 4 kHz su un input reale di circa 5.5 kHz. Effettivamente si sa che MCA per un segnale con un rate di circa 5.5 kHz registra solamente tra il 65 e l'80\% dei dati
effettivi, perciò è plausibile un simile rate.\\

Inoltre con questo campione di dati si può fare un'ulteriore analisi: si possono contare quanti degli eventi registrati sono successivi all'ingresso del fotone energetico
nel quarto rivelatore. Per contare questi fotoni è necessario andare a integrare il picco del TAC: infatti si è ottenuto un segnale diverso da zero in uscita dal TAC solo
se è partito lo \textit{start}, cioè solo se è entrato un fotone nel rivelatore quattro. Si ha che il TAC ha rivelato circa $2.685 \times 10^5$ eventi differenti, il che vuol
dire che, rispetto al totale, solo nell'1.86\% dei casi l'emissione del fotone più energetico è avvenuta in maniera da esser rivelata da R4. Tutti questi risultati sono presentati senza errore in quanto l'errore poissoniano
associato alle misure è molto basso (si ha un grande numero di dati a disposizione), ma sul dead time risulta difficile andare a stimare il reale errore.\\

\FloatBarrier
\subsection{Decadimento a tre fotoni}
Per il decadimento a tre fotoni, la conFigurazione dell'apparato strumentale è analoga a quella utilizzata per il decadimento in due fotoni, con l'unica differenza
che si sono spostati i tre rivelatori complanari in modo che formassero angoli di 120 gradi l'uno  con l'altro e il master trigger è stato impostato sulla coincidenza
tra tutti e tre i rivelatori al posto che sulla coincidenza solo tra il primo e il secondo. Il TAC è stato mantenuto analogamente (cioè come start si è usato un segnale
nel quarto rivelatore e come stop la coincidenza tra i tre rivelatori complanari). Lo spettro visualizzato si può vedere nella Figura \ref{gr:120_spettricalibrati_2}.\\
\input{./graphs/120_spettricalibrati_2}

Questi spettri però, non vanno bene, e questo si vede dal fatto che, a differenza da quelli a due fotoni, si continuano a distinguere tutti i picchi. Questo fenomeno
è conseguenza del fatto che nella versione con tre coincidenze hanno rilevanza statistica anche dei casi "spuri", come per esempio l'ingresso del fotone da 1275 keV in uno dei tre rivelatori,
di un fotone legato a un decadimento a due corpi in un secondo rivelatore e di un fotone ambientale nel terzo. Per pulire questi fenomeni si può, per esempio, andare a
richiedere che il fotone più energetico sia effettivamente stato rivelato da R4 (che si vede imponendo che il segnale del TAC riveli un valore maggiore di zero). Facendolo,
si trovano i quattro spettri che si possono vedere nella Figura \ref{gr:120_spettri_pulitir4_2}.\\
\input{./graphs/120_spettri_pulitir4_2}

Effettivamente da questi spettri è riconoscibile il fotopicco legato al decadimento in tre fotoni, ma il campione di dati risulta decisamente scarno, nonostante si abbia
preso dati per 1220 minuti. Inoltre sono ancora ben riconoscibili i fotopicchi legati al fotone da 1275 keV e al fotone da 511 keV, il che vuol dire che è necessaria 
un'ulteriore scrematura per poter vedere i decadimenti a tre fotoni a cui si è realmente interessati. Per farlo ci sono diversi modi possibili. Un primo tentativo consisteva
nel taglio in base al valore in uscita dal TAC, ma è risultato troppo difficile trovare effettivamente un intervallo che permettesse di vedere solamente il picco a cui
si è interessati, e si è perciò trascurato. Una seconda prova è stata fatta andando a imporre che sul rivelatore 4 fosse stato visto il fotone da 1275 keV, e che sui tre
rivelatori complanari l'energia fosse per ognuno inferiore al valore dopo il quale ci si aspetta di vedere i primi effetti degli altri due fotopicchi. I risultati
di questa scrematura si possono vedere nella Figura \ref{gr:120_spettri_pulitien_2}, dove effettivamente si nota come i tagli sulle energie siano al punto giusto e lo
spettro del TAC sia perfettamente in linea con ciò che si aspetta (gaussiana convoluta con decadimento esponenziale).\\
\input{./graphs/120_spettri_pulitien_2}

Una diversa scrematura può essere fatta richiedendo che la somma delle energie sui vari rivelatori sia inferiore all'energia data dall'annichilazione del positronio, cioè si
mette un vincolo a sulla somma totale affinché sia tra i 950 e i 1100 keV. In questa maniera si arriva agli spettri che si possono vedere nella 
Figura \ref{gr:120_spettri_pulitisom_2}.\\
\input{./graphs/120_spettri_pulitisom_2}

In questo grafico si vede che effettivamente si sono puliti abbastanza bene i dati, nonostante se ne siano persi una buona quantità, inoltre è quello fisicamente più
sensato tra i filtri inseriti. Questo vuol dire che si è ottenuto un rate di decadimento in tre fotoni decisamente molto basso (e quindi sul quale c'è un errore poissoniano
decisamente molto alto), cioè di circa $1.1 \times 10^{-2}$~Hz. Questo perché si è comunque chiesto che il quarto rivelatore rivelasse un
segnale valido per il TAC. Si è provato a mantenre i filtri
sulla somma delle energie rivelate e si sono aggiunti dei filtri affinché l'energia nei singoli rivelatori non sia nulla e non superi i 450 keVrimuovnedo il filtro sul
segnale valido del TAC. Questa selezione ha senso in quanto il filtro sul TAC era necessario ad escludere gli eventi in cui il fotone da 1275~keV era stato rilevato in uno dei rivelatori sul piano, ma i restanti filtri ci assicurano che questi eventi sono in ogni caso scartati. In questo modo si ottengono gli spettri che si possono vedere nella Figura \ref{gr:120_spettri_pulitisom_nor4_2}.\\
\input{./graphs/120_spettri_pulitisom_nor4_2}

Questi spettri si considerano in assoluto i più attendibili, e rivelano come ci sia una rate di circa 0.29 Hz. Non è necessario correggere questo valore per dead time perché
si vede dagli spettri nella Figura \ref{gr:120_spettricalibrati_2} che il rate è sufficientemente basso.
\FloatBarrier

\subsection{Confronto tra le rate}
Come si vedrà nella sezione successiva, per poter confrontare le rate di decadimento del positronio è necessario effettuare delle correzioni di natura geometrica. Per poterlo
fare, è necessario conoscere le grandezze fisiche caratteristiche dell'esperimento. Si sono misurati (e saranno definiti in seguito):
$$R = 17.9 \text{cm} \hspace{3cm} r =5.08 \text{cm}$$
Si sfrutti il calcolo teorico dell'accettanza (sezione successiva) per stimare il rapporto tra le rate dei due canali di decadimento. Per quanto riguarda il parapositronio, si sono visti eventi con una frequenza di 5.5~kHz, ma gli eventi visti sono circa il 4.03\% dei totali. Questo vuol dire che 
ci sono stati eventi con un rate di circa 136.5 kHz. Per l'ortopositronio, invece, si sono visti circa il 2.52\% degli eventi a 120 gradi,
perciò considerando che si è visto un rate
di 0.29 Hz, il rate effettivo è di circa 11,5 Hz. Considerando ora che nel decadimento dell'ortopositronio solamente circa il 3.5\% degli eventi decade con un
angolo che l'apparato strumentale vede\footnote{Si considera la distribuzione di probabilità trovata al link:
\cite{bib:pos_formula} (confrontata con il grafico trovato nell'articolo \cite{bib:pos_graph}).
Si trova che integrando numericamente tale funzione facendo variare l'angolo tra le traiettorie di 10 gradi attorno al valore 120 usato per la stima teorica},
si ha che si è creato l'ortopositronio con un rate di circa 328.8 Hz. Prima di poter calcolare il rapporto tra le due rate è necessario considerare l'effiecienza dei rivelatori. Sperimentalmente si vede che questa è di circa 0.4 (più avanti se ne vedrà il calcolo). Questo coefficiente entra nella stima della rate finale del positronio, al quadrato nel caso del p-Ps e al cubo nel caso dell'o-Ps. Questo vuol dire che il rapporto tra le due rate è di circa 1/166, abbastanza in linea con le previsioni teoriche\footnote{Nel considerare questo rapporto si sono fatte molte approssimazioni: il calcolo teorico è frutto di uno sviluppo al primo ordine, la correzione per il
dead time non è molto precisa e la pulizia degli eventi non è perfettamente univoca. Oltretutto gli intervalli di integrazione sono stati presi in modo da essere fisicamente
sensati, ma non per questo oggettivi, si è scelto 10 gradi per avere un'area simile a quella effettiva del rivelatore. La distribuzione di probabilità è quasi costante
attorno a quei valori.}\\

Sfruttando invece la simulazione Monte Carlo, facendo variare anche gli angoli tra le direzioni dei fotoni, si ha che il decadimento a 3 fotoni viene visto nello 0.06580\% dei casi. Andando a ricalcolare il rapporto tra le rate dei canali di decadimento, tenendo sempre conto della stima dell'efficienza e utilizzando il modello teorico per il fattore geometrico del p-Ps, si ottiene un rapporto di circa 1/124.

\FloatBarrier
\subsection{Il tempo di decadimento dell'ortopositronio}
Dato l'esperimento, si è voluto anche andare a stimare il tempo di vita medio delle specie prodotte. Per quanto riguarda il parapositronio, però, questo non è
possibile in quanto il tempo di vita è troppo basso, ed effettivamente si vede nello spettro del TAC della Figura \ref{gr:180_misura1_spettricalibrati_2} come si
abbia una gaussiana semplice, mentre dalla Figura \ref{gr:120_spettri_pulitien_2} si vede che c'è un decadimento esponenziale più marcato. Perciò si studi questo
grafico.\\

Si è provato a studiare i dati con molti filtri, ma in nessun caso si è arrivati a vedere un'esponenziale sufficientemente "buono" per essere interpolato. Questo
probabilmente è conseguenza del fatto che non si hanno molti dati una volta imposti i vari filtri per isolare l'ortopositronio. Quindi, in definitiva, si vede che
l'ortopositronio ha un tempo di vita più alto di quello del parapositronio (in quanto si hanno effettivamente più dati fuori e, più precisamente, a tempi maggiori, dalla
gaussiana), ma non è possibile, con i dati presi, andare a stimare la pendenza dell'esponenziale con cui decresce effetivamente lo spettro del TAC.



\FloatBarrier
\FloatBarrier


 

\FloatBarrier


\input{sections/chapters/PLACEHOLDER.txt}
\FloatBarrier


	
%appendix

Si presentano i grafici di interpolazione gaussiana che non sono stati riportati nella relazione per motivi di fluidità.

\begin{tikzpicture}
\pgfdeclareplotmark{cross} {
\pgfpathmoveto{\pgfpoint{-0.3\pgfplotmarksize}{\pgfplotmarksize}}
\pgfpathlineto{\pgfpoint{+0.3\pgfplotmarksize}{\pgfplotmarksize}}
\pgfpathlineto{\pgfpoint{+0.3\pgfplotmarksize}{0.3\pgfplotmarksize}}
\pgfpathlineto{\pgfpoint{+1\pgfplotmarksize}{0.3\pgfplotmarksize}}
\pgfpathlineto{\pgfpoint{+1\pgfplotmarksize}{-0.3\pgfplotmarksize}}
\pgfpathlineto{\pgfpoint{+0.3\pgfplotmarksize}{-0.3\pgfplotmarksize}}
\pgfpathlineto{\pgfpoint{+0.3\pgfplotmarksize}{-1.\pgfplotmarksize}}
\pgfpathlineto{\pgfpoint{-0.3\pgfplotmarksize}{-1.\pgfplotmarksize}}
\pgfpathlineto{\pgfpoint{-0.3\pgfplotmarksize}{-0.3\pgfplotmarksize}}
\pgfpathlineto{\pgfpoint{-1.\pgfplotmarksize}{-0.3\pgfplotmarksize}}
\pgfpathlineto{\pgfpoint{-1.\pgfplotmarksize}{0.3\pgfplotmarksize}}
\pgfpathlineto{\pgfpoint{-0.3\pgfplotmarksize}{0.3\pgfplotmarksize}}
\pgfpathclose
\pgfusepathqstroke
}
\pgfdeclareplotmark{cross*} {
\pgfpathmoveto{\pgfpoint{-0.3\pgfplotmarksize}{\pgfplotmarksize}}
\pgfpathlineto{\pgfpoint{+0.3\pgfplotmarksize}{\pgfplotmarksize}}
\pgfpathlineto{\pgfpoint{+0.3\pgfplotmarksize}{0.3\pgfplotmarksize}}
\pgfpathlineto{\pgfpoint{+1\pgfplotmarksize}{0.3\pgfplotmarksize}}
\pgfpathlineto{\pgfpoint{+1\pgfplotmarksize}{-0.3\pgfplotmarksize}}
\pgfpathlineto{\pgfpoint{+0.3\pgfplotmarksize}{-0.3\pgfplotmarksize}}
\pgfpathlineto{\pgfpoint{+0.3\pgfplotmarksize}{-1.\pgfplotmarksize}}
\pgfpathlineto{\pgfpoint{-0.3\pgfplotmarksize}{-1.\pgfplotmarksize}}
\pgfpathlineto{\pgfpoint{-0.3\pgfplotmarksize}{-0.3\pgfplotmarksize}}
\pgfpathlineto{\pgfpoint{-1.\pgfplotmarksize}{-0.3\pgfplotmarksize}}
\pgfpathlineto{\pgfpoint{-1.\pgfplotmarksize}{0.3\pgfplotmarksize}}
\pgfpathlineto{\pgfpoint{-0.3\pgfplotmarksize}{0.3\pgfplotmarksize}}
\pgfpathclose
\pgfusepathqfillstroke
}
\pgfdeclareplotmark{newstar} {
\pgfpathmoveto{\pgfqpoint{0pt}{\pgfplotmarksize}}
\pgfpathlineto{\pgfqpointpolar{44}{0.5\pgfplotmarksize}}
\pgfpathlineto{\pgfqpointpolar{18}{\pgfplotmarksize}}
\pgfpathlineto{\pgfqpointpolar{-20}{0.5\pgfplotmarksize}}
\pgfpathlineto{\pgfqpointpolar{-54}{\pgfplotmarksize}}
\pgfpathlineto{\pgfqpointpolar{-90}{0.5\pgfplotmarksize}}
\pgfpathlineto{\pgfqpointpolar{234}{\pgfplotmarksize}}
\pgfpathlineto{\pgfqpointpolar{198}{0.5\pgfplotmarksize}}
\pgfpathlineto{\pgfqpointpolar{162}{\pgfplotmarksize}}
\pgfpathlineto{\pgfqpointpolar{134}{0.5\pgfplotmarksize}}
\pgfpathclose
\pgfusepathqstroke
}
\pgfdeclareplotmark{newstar*} {
\pgfpathmoveto{\pgfqpoint{0pt}{\pgfplotmarksize}}
\pgfpathlineto{\pgfqpointpolar{44}{0.5\pgfplotmarksize}}
\pgfpathlineto{\pgfqpointpolar{18}{\pgfplotmarksize}}
\pgfpathlineto{\pgfqpointpolar{-20}{0.5\pgfplotmarksize}}
\pgfpathlineto{\pgfqpointpolar{-54}{\pgfplotmarksize}}
\pgfpathlineto{\pgfqpointpolar{-90}{0.5\pgfplotmarksize}}
\pgfpathlineto{\pgfqpointpolar{234}{\pgfplotmarksize}}
\pgfpathlineto{\pgfqpointpolar{198}{0.5\pgfplotmarksize}}
\pgfpathlineto{\pgfqpointpolar{162}{\pgfplotmarksize}}
\pgfpathlineto{\pgfqpointpolar{134}{0.5\pgfplotmarksize}}
\pgfpathclose
\pgfusepathqfillstroke
}
\definecolor{c}{rgb}{1,1,1};
\draw [color=c, fill=c] (0,0) rectangle (20,13.4957);
\draw [color=c, fill=c] (2,1.34957) rectangle (18,12.1461);
\definecolor{c}{rgb}{0,0,0};
\draw [c,line width=0.9] (2,1.34957) -- (2,12.1461) -- (18,12.1461) -- (18,1.34957) -- (2,1.34957);
\definecolor{c}{rgb}{1,1,1};
\draw [color=c, fill=c] (2,1.34957) rectangle (18,12.1461);
\definecolor{c}{rgb}{0,0,0};
\draw [c,line width=0.9] (2,1.34957) -- (2,12.1461) -- (18,12.1461) -- (18,1.34957) -- (2,1.34957);
\definecolor{c}{rgb}{0,0,0.6};
\draw [c,line width=0.9] (2,1.87057) -- (2.53333,1.87057) -- (2.53333,2.5399) -- (3.06667,2.5399) -- (3.06667,2.95658) -- (3.6,2.95658) -- (3.6,3.82792) -- (4.13333,3.82792) -- (4.13333,4.58831) -- (4.66667,4.58831) -- (4.66667,5.40659) --
 (5.2,5.40659) -- (5.2,6.34305) -- (5.73333,6.34305) -- (5.73333,7.32354) -- (6.26667,7.32354) -- (6.26667,8.298) -- (6.8,8.298) -- (6.8,8.91608) -- (7.33333,8.91608) -- (7.33333,9.59265) -- (7.86667,9.59265) -- (7.86667,10.5339) -- (8.4,10.5339) --
 (8.4,10.8915) -- (8.93333,10.8915) -- (8.93333,11.3209) -- (9.46667,11.3209) -- (9.46667,11.4342) -- (10,11.4342) -- (10,11.632) -- (10.5333,11.632) -- (10.5333,11.1942) -- (11.0667,11.1942) -- (11.0667,11.4125) -- (11.6,11.4125) -- (11.6,10.7444)
 -- (12.1333,10.7444) -- (12.1333,9.98822) -- (12.6667,9.98822) -- (12.6667,9.3822) -- (13.2,9.3822) -- (13.2,8.67668) -- (13.7333,8.67668) -- (13.7333,7.88675) -- (14.2667,7.88675) -- (14.2667,7.01179) -- (14.8,7.01179) -- (14.8,6.00175) --
 (15.3333,6.00175) -- (15.3333,5.15152) -- (15.8667,5.15152) -- (15.8667,4.19093) -- (16.4,4.19093) -- (16.4,3.32019) -- (16.9333,3.32019) -- (16.9333,2.55136) -- (17.4667,2.55136) -- (17.4667,1.83921) -- (18,1.83921);
\definecolor{c}{rgb}{1,1,1};
\draw [color=c, fill=c] (12.4,9.91934) rectangle (19.6,12.6185);
\definecolor{c}{rgb}{0,0,0};
\draw [c,line width=0.9] (12.4,9.91934) -- (19.6,9.91934);
\draw [c,line width=0.9] (19.6,9.91934) -- (19.6,12.6185);
\draw [c,line width=0.9] (19.6,12.6185) -- (12.4,12.6185);
\draw [c,line width=0.9] (12.4,12.6185) -- (12.4,9.91934);
\draw [anchor= west] (12.76,12.2811) node[scale=1.01821, color=c, rotate=0]{$\chi^{2} / ndf $};
\draw [anchor= east] (19.24,12.2811) node[scale=1.01821, color=c, rotate=0]{ 119.6 / 27};
\draw [anchor= west] (12.76,11.6063) node[scale=1.01821, color=c, rotate=0]{Constant };
\draw [anchor= east] (19.24,11.6063) node[scale=1.01821, color=c, rotate=0]{$ 2.256e+04 \pm 4.509e+01$};
\draw [anchor= west] (12.76,10.9315) node[scale=1.01821, color=c, rotate=0]{Mean     };
\draw [anchor= east] (19.24,10.9315) node[scale=1.01821, color=c, rotate=0]{$ 522.4 \pm 0.0$};
\draw [anchor= west] (12.76,10.2567) node[scale=1.01821, color=c, rotate=0]{Sigma    };
\draw [anchor= east] (19.24,10.2567) node[scale=1.01821, color=c, rotate=0]{$ 18.27 \pm 0.04$};
\definecolor{c}{rgb}{1,0,0};
\draw [c,line width=1.8] (2.08,1.3719) -- (2.24,1.56409) -- (2.4,1.76268) -- (2.56,1.96758) -- (2.72,2.17867) -- (2.88,2.39581) -- (3.04,2.61882) -- (3.2,2.84749) -- (3.36,3.0816) -- (3.52,3.32089) -- (3.68,3.56507) -- (3.84,3.81381) -- (4,4.06677)
 -- (4.16,4.32358) -- (4.32,4.58383) -- (4.48,4.84708) -- (4.64,5.11288) -- (4.8,5.38074) -- (4.96,5.65015) -- (5.12,5.92057) -- (5.28,6.19144) -- (5.44,6.46219) -- (5.6,6.73223) -- (5.76,7.00092) -- (5.92,7.26766) -- (6.08,7.53179) -- (6.24,7.79267)
 -- (6.4,8.04963) -- (6.56,8.30203) -- (6.72,8.54918) -- (6.88,8.79044) -- (7.04,9.02514) -- (7.2,9.25262) -- (7.36,9.47225) -- (7.52,9.68339) -- (7.68,9.88544) -- (7.84,10.0778) -- (8,10.2599) -- (8.16,10.4312) -- (8.32,10.5911) -- (8.48,10.7392) --
 (8.64,10.875) -- (8.8,10.9982) -- (8.96,11.1082) -- (9.12,11.2047) -- (9.28,11.2876) -- (9.44,11.3564) -- (9.6,11.4109) -- (9.76,11.4511) -- (9.92,11.4767);
\draw [c,line width=1.8] (9.92,11.4767) -- (10.08,11.4877) -- (10.24,11.484) -- (10.4,11.4657) -- (10.56,11.4328) -- (10.72,11.3854) -- (10.88,11.3237) -- (11.04,11.2478) -- (11.2,11.1581) -- (11.36,11.0547) -- (11.52,10.9381) -- (11.68,10.8086) --
 (11.84,10.6666) -- (12,10.5125) -- (12.16,10.3468) -- (12.32,10.17) -- (12.48,9.98274) -- (12.64,9.78546) -- (12.8,9.57878) -- (12.96,9.36331) -- (13.12,9.13967) -- (13.28,8.90849) -- (13.44,8.67043) -- (13.6,8.42614) -- (13.76,8.17628) --
 (13.92,7.92151) -- (14.08,7.6625) -- (14.24,7.39991) -- (14.4,7.1344) -- (14.56,6.8666) -- (14.72,6.59716) -- (14.88,6.32669) -- (15.04,6.05581) -- (15.2,5.78509) -- (15.36,5.51511) -- (15.52,5.24641) -- (15.68,4.97952) -- (15.84,4.71493) --
 (16,4.45313) -- (16.16,4.19455) -- (16.32,3.93962) -- (16.48,3.68872) -- (16.64,3.44222) -- (16.8,3.20046) -- (16.96,2.96373) -- (17.12,2.73231) -- (17.28,2.50645) -- (17.44,2.28635) -- (17.6,2.07222) -- (17.76,1.86421);
\draw [c,line width=1.8] (17.76,1.86421) -- (17.92,1.66246);
\definecolor{c}{rgb}{0,0,0};
\draw [c,line width=0.9] (2,1.34957) -- (18,1.34957);
\draw [c,line width=0.9] (4.13333,1.67347) -- (4.13333,1.34957);
\draw [c,line width=0.9] (4.66667,1.51152) -- (4.66667,1.34957);
\draw [c,line width=0.9] (5.2,1.51152) -- (5.2,1.34957);
\draw [c,line width=0.9] (5.73333,1.51152) -- (5.73333,1.34957);
\draw [c,line width=0.9] (6.26667,1.51152) -- (6.26667,1.34957);
\draw [c,line width=0.9] (6.8,1.67347) -- (6.8,1.34957);
\draw [c,line width=0.9] (7.33333,1.51152) -- (7.33333,1.34957);
\draw [c,line width=0.9] (7.86667,1.51152) -- (7.86667,1.34957);
\draw [c,line width=0.9] (8.4,1.51152) -- (8.4,1.34957);
\draw [c,line width=0.9] (8.93333,1.51152) -- (8.93333,1.34957);
\draw [c,line width=0.9] (9.46667,1.67347) -- (9.46667,1.34957);
\draw [c,line width=0.9] (10,1.51152) -- (10,1.34957);
\draw [c,line width=0.9] (10.5333,1.51152) -- (10.5333,1.34957);
\draw [c,line width=0.9] (11.0667,1.51152) -- (11.0667,1.34957);
\draw [c,line width=0.9] (11.6,1.51152) -- (11.6,1.34957);
\draw [c,line width=0.9] (12.1333,1.67347) -- (12.1333,1.34957);
\draw [c,line width=0.9] (12.6667,1.51152) -- (12.6667,1.34957);
\draw [c,line width=0.9] (13.2,1.51152) -- (13.2,1.34957);
\draw [c,line width=0.9] (13.7333,1.51152) -- (13.7333,1.34957);
\draw [c,line width=0.9] (14.2667,1.51152) -- (14.2667,1.34957);
\draw [c,line width=0.9] (14.8,1.67347) -- (14.8,1.34957);
\draw [c,line width=0.9] (15.3333,1.51152) -- (15.3333,1.34957);
\draw [c,line width=0.9] (15.8667,1.51152) -- (15.8667,1.34957);
\draw [c,line width=0.9] (16.4,1.51152) -- (16.4,1.34957);
\draw [c,line width=0.9] (16.9333,1.51152) -- (16.9333,1.34957);
\draw [c,line width=0.9] (17.4667,1.67347) -- (17.4667,1.34957);
\draw [c,line width=0.9] (4.13333,1.67347) -- (4.13333,1.34957);
\draw [c,line width=0.9] (3.6,1.51152) -- (3.6,1.34957);
\draw [c,line width=0.9] (3.06667,1.51152) -- (3.06667,1.34957);
\draw [c,line width=0.9] (2.53333,1.51152) -- (2.53333,1.34957);
\draw [c,line width=0.9] (17.4667,1.67347) -- (17.4667,1.34957);
\draw [c,line width=0.9] (18,1.51152) -- (18,1.34957);
\draw [anchor=base] (4.13333,0.904212) node[scale=1.01821, color=c, rotate=0]{500};
\draw [anchor=base] (6.8,0.904212) node[scale=1.01821, color=c, rotate=0]{510};
\draw [anchor=base] (9.46667,0.904212) node[scale=1.01821, color=c, rotate=0]{520};
\draw [anchor=base] (12.1333,0.904212) node[scale=1.01821, color=c, rotate=0]{530};
\draw [anchor=base] (14.8,0.904212) node[scale=1.01821, color=c, rotate=0]{540};
\draw [anchor=base] (17.4667,0.904212) node[scale=1.01821, color=c, rotate=0]{550};
\draw [c,line width=0.9] (2,1.34957) -- (2,12.1461);
\draw [c,line width=0.9] (2.48,1.50153) -- (2,1.50153);
\draw [c,line width=0.9] (2.24,1.80303) -- (2,1.80303);
\draw [c,line width=0.9] (2.24,2.10453) -- (2,2.10453);
\draw [c,line width=0.9] (2.24,2.40603) -- (2,2.40603);
\draw [c,line width=0.9] (2.48,2.70754) -- (2,2.70754);
\draw [c,line width=0.9] (2.24,3.00904) -- (2,3.00904);
\draw [c,line width=0.9] (2.24,3.31054) -- (2,3.31054);
\draw [c,line width=0.9] (2.24,3.61204) -- (2,3.61204);
\draw [c,line width=0.9] (2.48,3.91355) -- (2,3.91355);
\draw [c,line width=0.9] (2.24,4.21505) -- (2,4.21505);
\draw [c,line width=0.9] (2.24,4.51655) -- (2,4.51655);
\draw [c,line width=0.9] (2.24,4.81805) -- (2,4.81805);
\draw [c,line width=0.9] (2.48,5.11956) -- (2,5.11956);
\draw [c,line width=0.9] (2.24,5.42106) -- (2,5.42106);
\draw [c,line width=0.9] (2.24,5.72256) -- (2,5.72256);
\draw [c,line width=0.9] (2.24,6.02406) -- (2,6.02406);
\draw [c,line width=0.9] (2.48,6.32557) -- (2,6.32557);
\draw [c,line width=0.9] (2.24,6.62707) -- (2,6.62707);
\draw [c,line width=0.9] (2.24,6.92857) -- (2,6.92857);
\draw [c,line width=0.9] (2.24,7.23007) -- (2,7.23007);
\draw [c,line width=0.9] (2.48,7.53158) -- (2,7.53158);
\draw [c,line width=0.9] (2.24,7.83308) -- (2,7.83308);
\draw [c,line width=0.9] (2.24,8.13458) -- (2,8.13458);
\draw [c,line width=0.9] (2.24,8.43608) -- (2,8.43608);
\draw [c,line width=0.9] (2.48,8.73759) -- (2,8.73759);
\draw [c,line width=0.9] (2.24,9.03909) -- (2,9.03909);
\draw [c,line width=0.9] (2.24,9.34059) -- (2,9.34059);
\draw [c,line width=0.9] (2.24,9.64209) -- (2,9.64209);
\draw [c,line width=0.9] (2.48,9.9436) -- (2,9.9436);
\draw [c,line width=0.9] (2.24,10.2451) -- (2,10.2451);
\draw [c,line width=0.9] (2.24,10.5466) -- (2,10.5466);
\draw [c,line width=0.9] (2.24,10.8481) -- (2,10.8481);
\draw [c,line width=0.9] (2.48,11.1496) -- (2,11.1496);
\draw [c,line width=0.9] (2.48,1.50153) -- (2,1.50153);
\draw [c,line width=0.9] (2.48,11.1496) -- (2,11.1496);
\draw [c,line width=0.9] (2.24,11.4511) -- (2,11.4511);
\draw [c,line width=0.9] (2.24,11.7526) -- (2,11.7526);
\draw [c,line width=0.9] (2.24,12.0541) -- (2,12.0541);
\draw [anchor= east] (1.9,1.50153) node[scale=1.01821, color=c, rotate=0]{6000};
\draw [anchor= east] (1.9,2.70754) node[scale=1.01821, color=c, rotate=0]{8000};
\draw [anchor= east] (1.9,3.91355) node[scale=1.01821, color=c, rotate=0]{10000};
\draw [anchor= east] (1.9,5.11956) node[scale=1.01821, color=c, rotate=0]{12000};
\draw [anchor= east] (1.9,6.32557) node[scale=1.01821, color=c, rotate=0]{14000};
\draw [anchor= east] (1.9,7.53158) node[scale=1.01821, color=c, rotate=0]{16000};
\draw [anchor= east] (1.9,8.73759) node[scale=1.01821, color=c, rotate=0]{18000};
\draw [anchor= east] (1.9,9.9436) node[scale=1.01821, color=c, rotate=0]{20000};
\draw [anchor= east] (1.9,11.1496) node[scale=1.01821, color=c, rotate=0]{22000};
\definecolor{c}{rgb}{1,1,1};
\draw [color=c, fill=c] (12.4,9.91934) rectangle (19.6,12.6185);
\definecolor{c}{rgb}{0,0,0};
\draw [c,line width=0.9] (12.4,9.91934) -- (19.6,9.91934);
\draw [c,line width=0.9] (19.6,9.91934) -- (19.6,12.6185);
\draw [c,line width=0.9] (19.6,12.6185) -- (12.4,12.6185);
\draw [c,line width=0.9] (12.4,12.6185) -- (12.4,9.91934);
\draw [anchor= west] (12.76,12.2811) node[scale=1.01821, color=c, rotate=0]{$\chi^{2} / ndf $};
\draw [anchor= east] (19.24,12.2811) node[scale=1.01821, color=c, rotate=0]{ 119.6 / 27};
\draw [anchor= west] (12.76,11.6063) node[scale=1.01821, color=c, rotate=0]{Constant };
\draw [anchor= east] (19.24,11.6063) node[scale=1.01821, color=c, rotate=0]{$ 2.256e+04 \pm 4.509e+01$};
\draw [anchor= west] (12.76,10.9315) node[scale=1.01821, color=c, rotate=0]{Mean     };
\draw [anchor= east] (19.24,10.9315) node[scale=1.01821, color=c, rotate=0]{$ 522.4 \pm 0.0$};
\draw [anchor= west] (12.76,10.2567) node[scale=1.01821, color=c, rotate=0]{Sigma    };
\draw [anchor= east] (19.24,10.2567) node[scale=1.01821, color=c, rotate=0]{$ 18.27 \pm 0.04$};
\end{tikzpicture}
 
\begin{tikzpicture}
\pgfdeclareplotmark{cross} {
\pgfpathmoveto{\pgfpoint{-0.3\pgfplotmarksize}{\pgfplotmarksize}}
\pgfpathlineto{\pgfpoint{+0.3\pgfplotmarksize}{\pgfplotmarksize}}
\pgfpathlineto{\pgfpoint{+0.3\pgfplotmarksize}{0.3\pgfplotmarksize}}
\pgfpathlineto{\pgfpoint{+1\pgfplotmarksize}{0.3\pgfplotmarksize}}
\pgfpathlineto{\pgfpoint{+1\pgfplotmarksize}{-0.3\pgfplotmarksize}}
\pgfpathlineto{\pgfpoint{+0.3\pgfplotmarksize}{-0.3\pgfplotmarksize}}
\pgfpathlineto{\pgfpoint{+0.3\pgfplotmarksize}{-1.\pgfplotmarksize}}
\pgfpathlineto{\pgfpoint{-0.3\pgfplotmarksize}{-1.\pgfplotmarksize}}
\pgfpathlineto{\pgfpoint{-0.3\pgfplotmarksize}{-0.3\pgfplotmarksize}}
\pgfpathlineto{\pgfpoint{-1.\pgfplotmarksize}{-0.3\pgfplotmarksize}}
\pgfpathlineto{\pgfpoint{-1.\pgfplotmarksize}{0.3\pgfplotmarksize}}
\pgfpathlineto{\pgfpoint{-0.3\pgfplotmarksize}{0.3\pgfplotmarksize}}
\pgfpathclose
\pgfusepathqstroke
}
\pgfdeclareplotmark{cross*} {
\pgfpathmoveto{\pgfpoint{-0.3\pgfplotmarksize}{\pgfplotmarksize}}
\pgfpathlineto{\pgfpoint{+0.3\pgfplotmarksize}{\pgfplotmarksize}}
\pgfpathlineto{\pgfpoint{+0.3\pgfplotmarksize}{0.3\pgfplotmarksize}}
\pgfpathlineto{\pgfpoint{+1\pgfplotmarksize}{0.3\pgfplotmarksize}}
\pgfpathlineto{\pgfpoint{+1\pgfplotmarksize}{-0.3\pgfplotmarksize}}
\pgfpathlineto{\pgfpoint{+0.3\pgfplotmarksize}{-0.3\pgfplotmarksize}}
\pgfpathlineto{\pgfpoint{+0.3\pgfplotmarksize}{-1.\pgfplotmarksize}}
\pgfpathlineto{\pgfpoint{-0.3\pgfplotmarksize}{-1.\pgfplotmarksize}}
\pgfpathlineto{\pgfpoint{-0.3\pgfplotmarksize}{-0.3\pgfplotmarksize}}
\pgfpathlineto{\pgfpoint{-1.\pgfplotmarksize}{-0.3\pgfplotmarksize}}
\pgfpathlineto{\pgfpoint{-1.\pgfplotmarksize}{0.3\pgfplotmarksize}}
\pgfpathlineto{\pgfpoint{-0.3\pgfplotmarksize}{0.3\pgfplotmarksize}}
\pgfpathclose
\pgfusepathqfillstroke
}
\pgfdeclareplotmark{newstar} {
\pgfpathmoveto{\pgfqpoint{0pt}{\pgfplotmarksize}}
\pgfpathlineto{\pgfqpointpolar{44}{0.5\pgfplotmarksize}}
\pgfpathlineto{\pgfqpointpolar{18}{\pgfplotmarksize}}
\pgfpathlineto{\pgfqpointpolar{-20}{0.5\pgfplotmarksize}}
\pgfpathlineto{\pgfqpointpolar{-54}{\pgfplotmarksize}}
\pgfpathlineto{\pgfqpointpolar{-90}{0.5\pgfplotmarksize}}
\pgfpathlineto{\pgfqpointpolar{234}{\pgfplotmarksize}}
\pgfpathlineto{\pgfqpointpolar{198}{0.5\pgfplotmarksize}}
\pgfpathlineto{\pgfqpointpolar{162}{\pgfplotmarksize}}
\pgfpathlineto{\pgfqpointpolar{134}{0.5\pgfplotmarksize}}
\pgfpathclose
\pgfusepathqstroke
}
\pgfdeclareplotmark{newstar*} {
\pgfpathmoveto{\pgfqpoint{0pt}{\pgfplotmarksize}}
\pgfpathlineto{\pgfqpointpolar{44}{0.5\pgfplotmarksize}}
\pgfpathlineto{\pgfqpointpolar{18}{\pgfplotmarksize}}
\pgfpathlineto{\pgfqpointpolar{-20}{0.5\pgfplotmarksize}}
\pgfpathlineto{\pgfqpointpolar{-54}{\pgfplotmarksize}}
\pgfpathlineto{\pgfqpointpolar{-90}{0.5\pgfplotmarksize}}
\pgfpathlineto{\pgfqpointpolar{234}{\pgfplotmarksize}}
\pgfpathlineto{\pgfqpointpolar{198}{0.5\pgfplotmarksize}}
\pgfpathlineto{\pgfqpointpolar{162}{\pgfplotmarksize}}
\pgfpathlineto{\pgfqpointpolar{134}{0.5\pgfplotmarksize}}
\pgfpathclose
\pgfusepathqfillstroke
}
\definecolor{c}{rgb}{1,1,1};
\draw [color=c, fill=c] (0,0) rectangle (20,13.4957);
\draw [color=c, fill=c] (2,1.34957) rectangle (18,12.1461);
\definecolor{c}{rgb}{0,0,0};
\draw [c,line width=0.9] (2,1.34957) -- (2,12.1461) -- (18,12.1461) -- (18,1.34957) -- (2,1.34957);
\definecolor{c}{rgb}{1,1,1};
\draw [color=c, fill=c] (2,1.34957) rectangle (18,12.1461);
\definecolor{c}{rgb}{0,0,0};
\draw [c,line width=0.9] (2,1.34957) -- (2,12.1461) -- (18,12.1461) -- (18,1.34957) -- (2,1.34957);
\definecolor{c}{rgb}{0,0,0.6};
\draw [c,line width=0.9] (2,2.24592) -- (2.2963,2.24592) -- (2.2963,2.63992) -- (2.59259,2.63992) -- (2.59259,3.15466) -- (2.88889,3.15466) -- (2.88889,3.40568) -- (3.18519,3.40568) -- (3.18519,3.78061) -- (3.48148,3.78061) -- (3.48148,3.92995) --
 (3.77778,3.92995) -- (3.77778,4.56226) -- (4.07407,4.56226) -- (4.07407,5.07382) -- (4.37037,5.07382) -- (4.37037,5.67118) -- (4.66667,5.67118) -- (4.66667,5.84593) -- (4.96296,5.84593) -- (4.96296,6.20498) -- (5.25926,6.20498) -- (5.25926,6.78963)
 -- (5.55556,6.78963) -- (5.55556,8.41964) -- (5.85185,8.41964) -- (5.85185,9.51585) -- (6.14815,9.51585) -- (6.14815,8.67383) -- (6.44444,8.67383) -- (6.44444,9.01382) -- (6.74074,9.01382) -- (6.74074,9.14409) -- (7.03704,9.14409) --
 (7.03704,10.0941) -- (7.33333,10.0941) -- (7.33333,10.2498) -- (7.62963,10.2498) -- (7.62963,10.4754) -- (7.92593,10.4754) -- (7.92593,10.9171) -- (8.22222,10.9171) -- (8.22222,10.9425) -- (8.51852,10.9425) -- (8.51852,10.9616) -- (8.81481,10.9616)
 -- (8.81481,11.5494) -- (9.11111,11.5494) -- (9.11111,11.632) -- (9.40741,11.632) -- (9.40741,11.2094) -- (9.7037,11.2094) -- (9.7037,11.5875) -- (10,11.5875) -- (10,11.4477) -- (10.2963,11.4477) -- (10.2963,11.4414) -- (10.5926,11.4414) --
 (10.5926,11.0092) -- (10.8889,11.0092) -- (10.8889,10.8917) -- (11.1852,10.8917) -- (11.1852,10.5231) -- (11.4815,10.5231) -- (11.4815,9.78593) -- (11.7778,9.78593) -- (11.7778,9.69696) -- (12.0741,9.69696) -- (12.0741,9.77004) -- (12.3704,9.77004)
 -- (12.3704,9.09008) -- (12.6667,9.09008) -- (12.6667,8.93756) -- (12.963,8.93756) -- (12.963,8.37198) -- (13.2593,8.37198) -- (13.2593,8.0034) -- (13.5556,8.0034) -- (13.5556,7.3838) -- (13.8519,7.3838) -- (13.8519,6.7928) -- (14.1481,6.7928) --
 (14.1481,6.18274) -- (14.4444,6.18274) -- (14.4444,6.08742) -- (14.7407,6.08742) -- (14.7407,5.36297) -- (15.037,5.36297) -- (15.037,5.05158) -- (15.3333,5.05158) -- (15.3333,4.39068) -- (15.6296,4.39068) -- (15.6296,4.0221) -- (15.9259,4.0221) --
 (15.9259,3.49782) -- (16.2222,3.49782) -- (16.2222,3.4311) -- (16.5185,3.4311) -- (16.5185,2.90365) -- (16.8148,2.90365) -- (16.8148,2.62085) -- (17.1111,2.62085) -- (17.1111,2.18873) -- (17.4074,2.18873) -- (17.4074,1.94089) -- (17.7037,1.94089) --
 (17.7037,1.83921) -- (18,1.83921);
\definecolor{c}{rgb}{1,1,1};
\draw [color=c, fill=c] (12.4,9.91934) rectangle (19.6,12.6185);
\definecolor{c}{rgb}{0,0,0};
\draw [c,line width=0.9] (12.4,9.91934) -- (19.6,9.91934);
\draw [c,line width=0.9] (19.6,9.91934) -- (19.6,12.6185);
\draw [c,line width=0.9] (19.6,12.6185) -- (12.4,12.6185);
\draw [c,line width=0.9] (12.4,12.6185) -- (12.4,9.91934);
\draw [anchor= west] (12.76,12.2811) node[scale=1.40004, color=c, rotate=0]{$\chi^{2} / ndf $};
\draw [anchor= east] (19.24,12.2811) node[scale=1.40004, color=c, rotate=0]{ 167.5 / 51};
\draw [anchor= west] (12.76,11.6063) node[scale=1.40004, color=c, rotate=0]{Constant };
\draw [anchor= east] (19.24,11.6063) node[scale=1.40004, color=c, rotate=0]{$  3947 \pm 14.2$};
\draw [anchor= west] (12.76,10.9315) node[scale=1.40004, color=c, rotate=0]{Mean     };
\draw [anchor= east] (19.24,10.9315) node[scale=1.40004, color=c, rotate=0]{$  1286 \pm 0.1$};
\draw [anchor= west] (12.76,10.2567) node[scale=1.40004, color=c, rotate=0]{Sigma    };
\draw [anchor= east] (19.24,10.2567) node[scale=1.40004, color=c, rotate=0]{$ 31.25 \pm 0.11$};
\definecolor{c}{rgb}{1,0,0};
\draw [c,line width=1.8] (2.08,2.19136) -- (2.24,2.38092) -- (2.4,2.57696) -- (2.56,2.77939) -- (2.72,2.98805) -- (2.88,3.20276) -- (3.04,3.42332) -- (3.2,3.64949) -- (3.36,3.88099) -- (3.52,4.11752) -- (3.68,4.35873) -- (3.84,4.60426) -- (4,4.8537)
 -- (4.16,5.10661) -- (4.32,5.36252) -- (4.48,5.62093) -- (4.64,5.88132) -- (4.8,6.14312) -- (4.96,6.40575) -- (5.12,6.66861) -- (5.28,6.93106) -- (5.44,7.19245) -- (5.6,7.45212) -- (5.76,7.70938) -- (5.92,7.96353) -- (6.08,8.21387) -- (6.24,8.45969)
 -- (6.4,8.70027) -- (6.56,8.93491) -- (6.72,9.16289) -- (6.88,9.38351) -- (7.04,9.59608) -- (7.2,9.79992) -- (7.36,9.99438) -- (7.52,10.1788) -- (7.68,10.3526) -- (7.84,10.5152) -- (8,10.6661) -- (8.16,10.8047) -- (8.32,10.9305) -- (8.48,11.0431) --
 (8.64,11.1421) -- (8.8,11.2272) -- (8.96,11.2981) -- (9.12,11.3545) -- (9.28,11.3962) -- (9.44,11.4231) -- (9.6,11.4351) -- (9.76,11.432) -- (9.92,11.414);
\draw [c,line width=1.8] (9.92,11.414) -- (10.08,11.3812) -- (10.24,11.3335) -- (10.4,11.2712) -- (10.56,11.1946) -- (10.72,11.1038) -- (10.88,10.9992) -- (11.04,10.8812) -- (11.2,10.7502) -- (11.36,10.6066) -- (11.52,10.451) -- (11.68,10.2838) --
 (11.84,10.1056) -- (12,9.91705) -- (12.16,9.71872) -- (12.32,9.51128) -- (12.48,9.29538) -- (12.64,9.0717) -- (12.8,8.84095) -- (12.96,8.60383) -- (13.12,8.36105) -- (13.28,8.11332) -- (13.44,7.86135) -- (13.6,7.60586) -- (13.76,7.34755) --
 (13.92,7.0871) -- (14.08,6.8252) -- (14.24,6.56251) -- (14.4,6.29967) -- (14.56,6.0373) -- (14.72,5.776) -- (14.88,5.51635) -- (15.04,5.25888) -- (15.2,5.00413) -- (15.36,4.75256) -- (15.52,4.50465) -- (15.68,4.26082) -- (15.84,4.02145) --
 (16,3.78692) -- (16.16,3.55753) -- (16.32,3.3336) -- (16.48,3.11537) -- (16.64,2.90308) -- (16.8,2.69691) -- (16.96,2.49705) -- (17.12,2.30361) -- (17.28,2.1167) -- (17.44,1.93641) -- (17.6,1.76277) -- (17.76,1.59581);
\draw [c,line width=1.8] (17.76,1.59581) -- (17.92,1.43553);
\definecolor{c}{rgb}{0,0,0};
\draw [c,line width=0.9] (2,1.34957) -- (18,1.34957);
\draw [c,line width=0.9] (2.88889,1.67347) -- (2.88889,1.34957);
\draw [c,line width=0.9] (3.62963,1.51152) -- (3.62963,1.34957);
\draw [c,line width=0.9] (4.37037,1.51152) -- (4.37037,1.34957);
\draw [c,line width=0.9] (5.11111,1.51152) -- (5.11111,1.34957);
\draw [c,line width=0.9] (5.85185,1.67347) -- (5.85185,1.34957);
\draw [c,line width=0.9] (6.59259,1.51152) -- (6.59259,1.34957);
\draw [c,line width=0.9] (7.33333,1.51152) -- (7.33333,1.34957);
\draw [c,line width=0.9] (8.07407,1.51152) -- (8.07407,1.34957);
\draw [c,line width=0.9] (8.81481,1.67347) -- (8.81481,1.34957);
\draw [c,line width=0.9] (9.55556,1.51152) -- (9.55556,1.34957);
\draw [c,line width=0.9] (10.2963,1.51152) -- (10.2963,1.34957);
\draw [c,line width=0.9] (11.037,1.51152) -- (11.037,1.34957);
\draw [c,line width=0.9] (11.7778,1.67347) -- (11.7778,1.34957);
\draw [c,line width=0.9] (12.5185,1.51152) -- (12.5185,1.34957);
\draw [c,line width=0.9] (13.2593,1.51152) -- (13.2593,1.34957);
\draw [c,line width=0.9] (14,1.51152) -- (14,1.34957);
\draw [c,line width=0.9] (14.7407,1.67347) -- (14.7407,1.34957);
\draw [c,line width=0.9] (15.4815,1.51152) -- (15.4815,1.34957);
\draw [c,line width=0.9] (16.2222,1.51152) -- (16.2222,1.34957);
\draw [c,line width=0.9] (16.963,1.51152) -- (16.963,1.34957);
\draw [c,line width=0.9] (17.7037,1.67347) -- (17.7037,1.34957);
\draw [c,line width=0.9] (2.88889,1.67347) -- (2.88889,1.34957);
\draw [c,line width=0.9] (2.14815,1.51152) -- (2.14815,1.34957);
\draw [c,line width=0.9] (17.7037,1.67347) -- (17.7037,1.34957);
\draw [anchor=base] (2.88889,0.904212) node[scale=1.01821, color=c, rotate=0]{1240};
\draw [anchor=base] (5.85185,0.904212) node[scale=1.01821, color=c, rotate=0]{1260};
\draw [anchor=base] (8.81481,0.904212) node[scale=1.01821, color=c, rotate=0]{1280};
\draw [anchor=base] (11.7778,0.904212) node[scale=1.01821, color=c, rotate=0]{1300};
\draw [anchor=base] (14.7407,0.904212) node[scale=1.01821, color=c, rotate=0]{1320};
\draw [anchor=base] (17.7037,0.904212) node[scale=1.01821, color=c, rotate=0]{1340};
\draw [c,line width=0.9] (2,1.34957) -- (2,12.1461);
\draw [c,line width=0.9] (2.48,2.07116) -- (2,2.07116);
\draw [c,line width=0.9] (2.24,2.3889) -- (2,2.3889);
\draw [c,line width=0.9] (2.24,2.70665) -- (2,2.70665);
\draw [c,line width=0.9] (2.24,3.02439) -- (2,3.02439);
\draw [c,line width=0.9] (2.24,3.34213) -- (2,3.34213);
\draw [c,line width=0.9] (2.48,3.65987) -- (2,3.65987);
\draw [c,line width=0.9] (2.24,3.97761) -- (2,3.97761);
\draw [c,line width=0.9] (2.24,4.29535) -- (2,4.29535);
\draw [c,line width=0.9] (2.24,4.6131) -- (2,4.6131);
\draw [c,line width=0.9] (2.24,4.93084) -- (2,4.93084);
\draw [c,line width=0.9] (2.48,5.24858) -- (2,5.24858);
\draw [c,line width=0.9] (2.24,5.56632) -- (2,5.56632);
\draw [c,line width=0.9] (2.24,5.88406) -- (2,5.88406);
\draw [c,line width=0.9] (2.24,6.2018) -- (2,6.2018);
\draw [c,line width=0.9] (2.24,6.51955) -- (2,6.51955);
\draw [c,line width=0.9] (2.48,6.83729) -- (2,6.83729);
\draw [c,line width=0.9] (2.24,7.15503) -- (2,7.15503);
\draw [c,line width=0.9] (2.24,7.47277) -- (2,7.47277);
\draw [c,line width=0.9] (2.24,7.79051) -- (2,7.79051);
\draw [c,line width=0.9] (2.24,8.10825) -- (2,8.10825);
\draw [c,line width=0.9] (2.48,8.426) -- (2,8.426);
\draw [c,line width=0.9] (2.24,8.74374) -- (2,8.74374);
\draw [c,line width=0.9] (2.24,9.06148) -- (2,9.06148);
\draw [c,line width=0.9] (2.24,9.37922) -- (2,9.37922);
\draw [c,line width=0.9] (2.24,9.69696) -- (2,9.69696);
\draw [c,line width=0.9] (2.48,10.0147) -- (2,10.0147);
\draw [c,line width=0.9] (2.24,10.3324) -- (2,10.3324);
\draw [c,line width=0.9] (2.24,10.6502) -- (2,10.6502);
\draw [c,line width=0.9] (2.24,10.9679) -- (2,10.9679);
\draw [c,line width=0.9] (2.24,11.2857) -- (2,11.2857);
\draw [c,line width=0.9] (2.48,11.6034) -- (2,11.6034);
\draw [c,line width=0.9] (2.48,2.07116) -- (2,2.07116);
\draw [c,line width=0.9] (2.24,1.75342) -- (2,1.75342);
\draw [c,line width=0.9] (2.24,1.43568) -- (2,1.43568);
\draw [c,line width=0.9] (2.48,11.6034) -- (2,11.6034);
\draw [c,line width=0.9] (2.24,11.9212) -- (2,11.9212);
\draw [anchor= east] (1.9,2.07116) node[scale=1.01821, color=c, rotate=0]{1000};
\draw [anchor= east] (1.9,3.65987) node[scale=1.01821, color=c, rotate=0]{1500};
\draw [anchor= east] (1.9,5.24858) node[scale=1.01821, color=c, rotate=0]{2000};
\draw [anchor= east] (1.9,6.83729) node[scale=1.01821, color=c, rotate=0]{2500};
\draw [anchor= east] (1.9,8.426) node[scale=1.01821, color=c, rotate=0]{3000};
\draw [anchor= east] (1.9,10.0147) node[scale=1.01821, color=c, rotate=0]{3500};
\draw [anchor= east] (1.9,11.6034) node[scale=1.01821, color=c, rotate=0]{4000};
\definecolor{c}{rgb}{1,1,1};
\draw [color=c, fill=c] (12.4,9.91934) rectangle (19.6,12.6185);
\definecolor{c}{rgb}{0,0,0};
\draw [c,line width=0.9] (12.4,9.91934) -- (19.6,9.91934);
\draw [c,line width=0.9] (19.6,9.91934) -- (19.6,12.6185);
\draw [c,line width=0.9] (19.6,12.6185) -- (12.4,12.6185);
\draw [c,line width=0.9] (12.4,12.6185) -- (12.4,9.91934);
\draw [anchor= west] (12.76,12.2811) node[scale=1.40004, color=c, rotate=0]{$\chi^{2} / ndf $};
\draw [anchor= east] (19.24,12.2811) node[scale=1.40004, color=c, rotate=0]{ 167.5 / 51};
\draw [anchor= west] (12.76,11.6063) node[scale=1.40004, color=c, rotate=0]{Constant };
\draw [anchor= east] (19.24,11.6063) node[scale=1.40004, color=c, rotate=0]{$  3947 \pm 14.2$};
\draw [anchor= west] (12.76,10.9315) node[scale=1.40004, color=c, rotate=0]{Mean     };
\draw [anchor= east] (19.24,10.9315) node[scale=1.40004, color=c, rotate=0]{$  1286 \pm 0.1$};
\draw [anchor= west] (12.76,10.2567) node[scale=1.40004, color=c, rotate=0]{Sigma    };
\draw [anchor= east] (19.24,10.2567) node[scale=1.40004, color=c, rotate=0]{$ 31.25 \pm 0.11$};
\end{tikzpicture}
 
\begin{tikzpicture}
\pgfdeclareplotmark{cross} {
\pgfpathmoveto{\pgfpoint{-0.3\pgfplotmarksize}{\pgfplotmarksize}}
\pgfpathlineto{\pgfpoint{+0.3\pgfplotmarksize}{\pgfplotmarksize}}
\pgfpathlineto{\pgfpoint{+0.3\pgfplotmarksize}{0.3\pgfplotmarksize}}
\pgfpathlineto{\pgfpoint{+1\pgfplotmarksize}{0.3\pgfplotmarksize}}
\pgfpathlineto{\pgfpoint{+1\pgfplotmarksize}{-0.3\pgfplotmarksize}}
\pgfpathlineto{\pgfpoint{+0.3\pgfplotmarksize}{-0.3\pgfplotmarksize}}
\pgfpathlineto{\pgfpoint{+0.3\pgfplotmarksize}{-1.\pgfplotmarksize}}
\pgfpathlineto{\pgfpoint{-0.3\pgfplotmarksize}{-1.\pgfplotmarksize}}
\pgfpathlineto{\pgfpoint{-0.3\pgfplotmarksize}{-0.3\pgfplotmarksize}}
\pgfpathlineto{\pgfpoint{-1.\pgfplotmarksize}{-0.3\pgfplotmarksize}}
\pgfpathlineto{\pgfpoint{-1.\pgfplotmarksize}{0.3\pgfplotmarksize}}
\pgfpathlineto{\pgfpoint{-0.3\pgfplotmarksize}{0.3\pgfplotmarksize}}
\pgfpathclose
\pgfusepathqstroke
}
\pgfdeclareplotmark{cross*} {
\pgfpathmoveto{\pgfpoint{-0.3\pgfplotmarksize}{\pgfplotmarksize}}
\pgfpathlineto{\pgfpoint{+0.3\pgfplotmarksize}{\pgfplotmarksize}}
\pgfpathlineto{\pgfpoint{+0.3\pgfplotmarksize}{0.3\pgfplotmarksize}}
\pgfpathlineto{\pgfpoint{+1\pgfplotmarksize}{0.3\pgfplotmarksize}}
\pgfpathlineto{\pgfpoint{+1\pgfplotmarksize}{-0.3\pgfplotmarksize}}
\pgfpathlineto{\pgfpoint{+0.3\pgfplotmarksize}{-0.3\pgfplotmarksize}}
\pgfpathlineto{\pgfpoint{+0.3\pgfplotmarksize}{-1.\pgfplotmarksize}}
\pgfpathlineto{\pgfpoint{-0.3\pgfplotmarksize}{-1.\pgfplotmarksize}}
\pgfpathlineto{\pgfpoint{-0.3\pgfplotmarksize}{-0.3\pgfplotmarksize}}
\pgfpathlineto{\pgfpoint{-1.\pgfplotmarksize}{-0.3\pgfplotmarksize}}
\pgfpathlineto{\pgfpoint{-1.\pgfplotmarksize}{0.3\pgfplotmarksize}}
\pgfpathlineto{\pgfpoint{-0.3\pgfplotmarksize}{0.3\pgfplotmarksize}}
\pgfpathclose
\pgfusepathqfillstroke
}
\pgfdeclareplotmark{newstar} {
\pgfpathmoveto{\pgfqpoint{0pt}{\pgfplotmarksize}}
\pgfpathlineto{\pgfqpointpolar{44}{0.5\pgfplotmarksize}}
\pgfpathlineto{\pgfqpointpolar{18}{\pgfplotmarksize}}
\pgfpathlineto{\pgfqpointpolar{-20}{0.5\pgfplotmarksize}}
\pgfpathlineto{\pgfqpointpolar{-54}{\pgfplotmarksize}}
\pgfpathlineto{\pgfqpointpolar{-90}{0.5\pgfplotmarksize}}
\pgfpathlineto{\pgfqpointpolar{234}{\pgfplotmarksize}}
\pgfpathlineto{\pgfqpointpolar{198}{0.5\pgfplotmarksize}}
\pgfpathlineto{\pgfqpointpolar{162}{\pgfplotmarksize}}
\pgfpathlineto{\pgfqpointpolar{134}{0.5\pgfplotmarksize}}
\pgfpathclose
\pgfusepathqstroke
}
\pgfdeclareplotmark{newstar*} {
\pgfpathmoveto{\pgfqpoint{0pt}{\pgfplotmarksize}}
\pgfpathlineto{\pgfqpointpolar{44}{0.5\pgfplotmarksize}}
\pgfpathlineto{\pgfqpointpolar{18}{\pgfplotmarksize}}
\pgfpathlineto{\pgfqpointpolar{-20}{0.5\pgfplotmarksize}}
\pgfpathlineto{\pgfqpointpolar{-54}{\pgfplotmarksize}}
\pgfpathlineto{\pgfqpointpolar{-90}{0.5\pgfplotmarksize}}
\pgfpathlineto{\pgfqpointpolar{234}{\pgfplotmarksize}}
\pgfpathlineto{\pgfqpointpolar{198}{0.5\pgfplotmarksize}}
\pgfpathlineto{\pgfqpointpolar{162}{\pgfplotmarksize}}
\pgfpathlineto{\pgfqpointpolar{134}{0.5\pgfplotmarksize}}
\pgfpathclose
\pgfusepathqfillstroke
}
\definecolor{c}{rgb}{1,1,1};
\draw [color=c, fill=c] (0,0) rectangle (20,13.4957);
\draw [color=c, fill=c] (2,1.34957) rectangle (18,12.1461);
\definecolor{c}{rgb}{0,0,0};
\draw [c,line width=0.9] (2,1.34957) -- (2,12.1461) -- (18,12.1461) -- (18,1.34957) -- (2,1.34957);
\definecolor{c}{rgb}{1,1,1};
\draw [color=c, fill=c] (2,1.34957) rectangle (18,12.1461);
\definecolor{c}{rgb}{0,0,0};
\draw [c,line width=0.9] (2,1.34957) -- (2,12.1461) -- (18,12.1461) -- (18,1.34957) -- (2,1.34957);
\definecolor{c}{rgb}{0,0,0.6};
\draw [c,line width=0.9] (2,3.02763) -- (2.48485,3.02763) -- (2.48485,3.432) -- (2.9697,3.432) -- (2.9697,3.99883) -- (3.45455,3.99883) -- (3.45455,4.39243) -- (3.93939,4.39243) -- (3.93939,5.10916) -- (4.42424,5.10916) -- (4.42424,5.80211) --
 (4.90909,5.80211) -- (4.90909,6.58436) -- (5.39394,6.58436) -- (5.39394,7.49184) -- (5.87879,7.49184) -- (5.87879,8.30596) -- (6.36364,8.30596) -- (6.36364,8.92799) -- (6.84848,8.92799) -- (6.84848,9.64697) -- (7.33333,9.64697) -- (7.33333,10.2641)
 -- (7.81818,10.2641) -- (7.81818,10.8515) -- (8.30303,10.8515) -- (8.30303,11.0607) -- (8.78788,11.0607) -- (8.78788,11.3748) -- (9.27273,11.3748) -- (9.27273,11.5674) -- (9.75758,11.5674) -- (9.75758,11.632) -- (10.2424,11.632) -- (10.2424,11.4731)
 -- (10.7273,11.4731) -- (10.7273,11.2537) -- (11.2121,11.2537) -- (11.2121,10.4449) -- (11.697,10.4449) -- (11.697,9.85791) -- (12.1818,9.85791) -- (12.1818,9.12906) -- (12.6667,9.12906) -- (12.6667,8.51375) -- (13.1515,8.51375) -- (13.1515,7.65296)
 -- (13.6364,7.65296) -- (13.6364,6.85454) -- (14.1212,6.85454) -- (14.1212,5.86808) -- (14.6061,5.86808) -- (14.6061,5.1316) -- (15.0909,5.1316) -- (15.0909,4.4014) -- (15.5758,4.4014) -- (15.5758,3.77129) -- (16.0606,3.77129) -- (16.0606,3.22151)
 -- (16.5455,3.22151) -- (16.5455,2.67218) -- (17.0303,2.67218) -- (17.0303,2.21306) -- (17.5152,2.21306) -- (17.5152,1.83921) -- (18,1.83921);
\definecolor{c}{rgb}{1,0,0};
\draw [c,line width=1.8] (2.08,2.56914) -- (2.24,2.72499) -- (2.4,2.88843) -- (2.56,3.05947) -- (2.72,3.2381) -- (2.88,3.42428) -- (3.04,3.61792) -- (3.2,3.81888) -- (3.36,4.027) -- (3.52,4.24205) -- (3.68,4.46379) -- (3.84,4.69188) -- (4,4.92599) --
 (4.16,5.16571) -- (4.32,5.41057) -- (4.48,5.66009) -- (4.64,5.91371) -- (4.8,6.17083) -- (4.96,6.43082) -- (5.12,6.69299) -- (5.28,6.95662) -- (5.44,7.22094) -- (5.6,7.48517) -- (5.76,7.74848) -- (5.92,8.01) -- (6.08,8.26887) -- (6.24,8.52419) --
 (6.4,8.77504) -- (6.56,9.02052) -- (6.72,9.2597) -- (6.88,9.49166) -- (7.04,9.7155) -- (7.2,9.93033) -- (7.36,10.1353) -- (7.52,10.3295) -- (7.68,10.5121) -- (7.84,10.6825) -- (8,10.8397) -- (8.16,10.9833) -- (8.32,11.1124) -- (8.48,11.2266) --
 (8.64,11.3254) -- (8.8,11.4082) -- (8.96,11.4748) -- (9.12,11.5248) -- (9.28,11.5579) -- (9.44,11.5742) -- (9.6,11.5733) -- (9.76,11.5555) -- (9.92,11.5207);
\draw [c,line width=1.8] (9.92,11.5207) -- (10.08,11.4691) -- (10.24,11.401) -- (10.4,11.3166) -- (10.56,11.2163) -- (10.72,11.1006) -- (10.88,10.9701) -- (11.04,10.8252) -- (11.2,10.6667) -- (11.36,10.4951) -- (11.52,10.3113) -- (11.68,10.116) --
 (11.84,9.91011) -- (12,9.69438) -- (12.16,9.46972) -- (12.32,9.23702) -- (12.48,8.9972) -- (12.64,8.75117) -- (12.8,8.49984) -- (12.96,8.24415) -- (13.12,7.98498) -- (13.28,7.72325) -- (13.44,7.45982) -- (13.6,7.19555) -- (13.76,6.93125) --
 (13.92,6.66773) -- (14.08,6.40574) -- (14.24,6.146) -- (14.4,5.88918) -- (14.56,5.63594) -- (14.72,5.38684) -- (14.88,5.14245) -- (15.04,4.90326) -- (15.2,4.66971) -- (15.36,4.44221) -- (15.52,4.2211) -- (15.68,4.0067) -- (15.84,3.79926) --
 (16,3.599) -- (16.16,3.40608) -- (16.32,3.22062) -- (16.48,3.04271) -- (16.64,2.8724) -- (16.8,2.70969) -- (16.96,2.55456) -- (17.12,2.40696) -- (17.28,2.26679) -- (17.44,2.13394) -- (17.6,2.00827) -- (17.76,1.88963);
\draw [c,line width=1.8] (17.76,1.88963) -- (17.92,1.77783);
\definecolor{c}{rgb}{1,1,1};
\draw [color=c, fill=c] (12.4,9.91934) rectangle (19.6,12.6185);
\definecolor{c}{rgb}{0,0,0};
\draw [c,line width=0.9] (12.4,9.91934) -- (19.6,9.91934);
\draw [c,line width=0.9] (19.6,9.91934) -- (19.6,12.6185);
\draw [c,line width=0.9] (19.6,12.6185) -- (12.4,12.6185);
\draw [c,line width=0.9] (12.4,12.6185) -- (12.4,9.91934);
\draw [anchor= west] (12.76,12.2811) node[scale=1.01821, color=c, rotate=0]{$\chi^{2} / ndf $};
\draw [anchor= east] (19.24,12.2811) node[scale=1.01821, color=c, rotate=0]{ 259.3 / 30};
\draw [anchor= west] (12.76,11.6063) node[scale=1.01821, color=c, rotate=0]{Constant };
\draw [anchor= east] (19.24,11.6063) node[scale=1.01821, color=c, rotate=0]{$ 2.484e+04 \pm 4.658e+01$};
\draw [anchor= west] (12.76,10.9315) node[scale=1.01821, color=c, rotate=0]{Mean     };
\draw [anchor= east] (19.24,10.9315) node[scale=1.01821, color=c, rotate=0]{$   521 \pm 0.0$};
\draw [anchor= west] (12.76,10.2567) node[scale=1.01821, color=c, rotate=0]{Sigma    };
\draw [anchor= east] (19.24,10.2567) node[scale=1.01821, color=c, rotate=0]{$ 16.88 \pm 0.03$};
\draw [c,line width=0.9] (2,1.34957) -- (18,1.34957);
\draw [c,line width=0.9] (2,1.67347) -- (2,1.34957);
\draw [c,line width=0.9] (2.48485,1.51152) -- (2.48485,1.34957);
\draw [c,line width=0.9] (2.9697,1.51152) -- (2.9697,1.34957);
\draw [c,line width=0.9] (3.45455,1.51152) -- (3.45455,1.34957);
\draw [c,line width=0.9] (3.93939,1.51152) -- (3.93939,1.34957);
\draw [c,line width=0.9] (4.42424,1.67347) -- (4.42424,1.34957);
\draw [c,line width=0.9] (4.90909,1.51152) -- (4.90909,1.34957);
\draw [c,line width=0.9] (5.39394,1.51152) -- (5.39394,1.34957);
\draw [c,line width=0.9] (5.87879,1.51152) -- (5.87879,1.34957);
\draw [c,line width=0.9] (6.36364,1.51152) -- (6.36364,1.34957);
\draw [c,line width=0.9] (6.84848,1.67347) -- (6.84848,1.34957);
\draw [c,line width=0.9] (7.33333,1.51152) -- (7.33333,1.34957);
\draw [c,line width=0.9] (7.81818,1.51152) -- (7.81818,1.34957);
\draw [c,line width=0.9] (8.30303,1.51152) -- (8.30303,1.34957);
\draw [c,line width=0.9] (8.78788,1.51152) -- (8.78788,1.34957);
\draw [c,line width=0.9] (9.27273,1.67347) -- (9.27273,1.34957);
\draw [c,line width=0.9] (9.75758,1.51152) -- (9.75758,1.34957);
\draw [c,line width=0.9] (10.2424,1.51152) -- (10.2424,1.34957);
\draw [c,line width=0.9] (10.7273,1.51152) -- (10.7273,1.34957);
\draw [c,line width=0.9] (11.2121,1.51152) -- (11.2121,1.34957);
\draw [c,line width=0.9] (11.697,1.67347) -- (11.697,1.34957);
\draw [c,line width=0.9] (12.1818,1.51152) -- (12.1818,1.34957);
\draw [c,line width=0.9] (12.6667,1.51152) -- (12.6667,1.34957);
\draw [c,line width=0.9] (13.1515,1.51152) -- (13.1515,1.34957);
\draw [c,line width=0.9] (13.6364,1.51152) -- (13.6364,1.34957);
\draw [c,line width=0.9] (14.1212,1.67347) -- (14.1212,1.34957);
\draw [c,line width=0.9] (14.6061,1.51152) -- (14.6061,1.34957);
\draw [c,line width=0.9] (15.0909,1.51152) -- (15.0909,1.34957);
\draw [c,line width=0.9] (15.5758,1.51152) -- (15.5758,1.34957);
\draw [c,line width=0.9] (16.0606,1.51152) -- (16.0606,1.34957);
\draw [c,line width=0.9] (16.5455,1.67347) -- (16.5455,1.34957);
\draw [c,line width=0.9] (16.5455,1.67347) -- (16.5455,1.34957);
\draw [c,line width=0.9] (17.0303,1.51152) -- (17.0303,1.34957);
\draw [c,line width=0.9] (17.5152,1.51152) -- (17.5152,1.34957);
\draw [c,line width=0.9] (18,1.51152) -- (18,1.34957);
\draw [anchor=base] (2,0.904212) node[scale=1.01821, color=c, rotate=0]{490};
\draw [anchor=base] (4.42424,0.904212) node[scale=1.01821, color=c, rotate=0]{500};
\draw [anchor=base] (6.84848,0.904212) node[scale=1.01821, color=c, rotate=0]{510};
\draw [anchor=base] (9.27273,0.904212) node[scale=1.01821, color=c, rotate=0]{520};
\draw [anchor=base] (11.697,0.904212) node[scale=1.01821, color=c, rotate=0]{530};
\draw [anchor=base] (14.1212,0.904212) node[scale=1.01821, color=c, rotate=0]{540};
\draw [anchor=base] (16.5455,0.904212) node[scale=1.01821, color=c, rotate=0]{550};
\draw [c,line width=0.9] (2,1.34957) -- (2,12.1461);
\draw [c,line width=0.9] (2.48,2.67263) -- (2,2.67263);
\draw [c,line width=0.9] (2.24,3.12143) -- (2,3.12143);
\draw [c,line width=0.9] (2.24,3.57023) -- (2,3.57023);
\draw [c,line width=0.9] (2.24,4.01903) -- (2,4.01903);
\draw [c,line width=0.9] (2.24,4.46783) -- (2,4.46783);
\draw [c,line width=0.9] (2.48,4.91663) -- (2,4.91663);
\draw [c,line width=0.9] (2.24,5.36543) -- (2,5.36543);
\draw [c,line width=0.9] (2.24,5.81423) -- (2,5.81423);
\draw [c,line width=0.9] (2.24,6.26302) -- (2,6.26302);
\draw [c,line width=0.9] (2.24,6.71182) -- (2,6.71182);
\draw [c,line width=0.9] (2.48,7.16062) -- (2,7.16062);
\draw [c,line width=0.9] (2.24,7.60942) -- (2,7.60942);
\draw [c,line width=0.9] (2.24,8.05822) -- (2,8.05822);
\draw [c,line width=0.9] (2.24,8.50702) -- (2,8.50702);
\draw [c,line width=0.9] (2.24,8.95582) -- (2,8.95582);
\draw [c,line width=0.9] (2.48,9.40462) -- (2,9.40462);
\draw [c,line width=0.9] (2.24,9.85342) -- (2,9.85342);
\draw [c,line width=0.9] (2.24,10.3022) -- (2,10.3022);
\draw [c,line width=0.9] (2.24,10.751) -- (2,10.751);
\draw [c,line width=0.9] (2.24,11.1998) -- (2,11.1998);
\draw [c,line width=0.9] (2.48,11.6486) -- (2,11.6486);
\draw [c,line width=0.9] (2.48,2.67263) -- (2,2.67263);
\draw [c,line width=0.9] (2.24,2.22383) -- (2,2.22383);
\draw [c,line width=0.9] (2.24,1.77503) -- (2,1.77503);
\draw [c,line width=0.9] (2.48,11.6486) -- (2,11.6486);
\draw [c,line width=0.9] (2.24,12.0974) -- (2,12.0974);
\draw [anchor= east] (1.9,2.67263) node[scale=1.01821, color=c, rotate=0]{5000};
\draw [anchor= east] (1.9,4.91663) node[scale=1.01821, color=c, rotate=0]{10000};
\draw [anchor= east] (1.9,7.16062) node[scale=1.01821, color=c, rotate=0]{15000};
\draw [anchor= east] (1.9,9.40462) node[scale=1.01821, color=c, rotate=0]{20000};
\draw [anchor= east] (1.9,11.6486) node[scale=1.01821, color=c, rotate=0]{25000};
\definecolor{c}{rgb}{1,1,1};
\draw [color=c, fill=c] (12.4,9.91934) rectangle (19.6,12.6185);
\definecolor{c}{rgb}{0,0,0};
\draw [c,line width=0.9] (12.4,9.91934) -- (19.6,9.91934);
\draw [c,line width=0.9] (19.6,9.91934) -- (19.6,12.6185);
\draw [c,line width=0.9] (19.6,12.6185) -- (12.4,12.6185);
\draw [c,line width=0.9] (12.4,12.6185) -- (12.4,9.91934);
\draw [anchor= west] (12.76,12.2811) node[scale=1.01821, color=c, rotate=0]{$\chi^{2} / ndf $};
\draw [anchor= east] (19.24,12.2811) node[scale=1.01821, color=c, rotate=0]{ 259.3 / 30};
\draw [anchor= west] (12.76,11.6063) node[scale=1.01821, color=c, rotate=0]{Constant };
\draw [anchor= east] (19.24,11.6063) node[scale=1.01821, color=c, rotate=0]{$ 2.484e+04 \pm 4.658e+01$};
\draw [anchor= west] (12.76,10.9315) node[scale=1.01821, color=c, rotate=0]{Mean     };
\draw [anchor= east] (19.24,10.9315) node[scale=1.01821, color=c, rotate=0]{$   521 \pm 0.0$};
\draw [anchor= west] (12.76,10.2567) node[scale=1.01821, color=c, rotate=0]{Sigma    };
\draw [anchor= east] (19.24,10.2567) node[scale=1.01821, color=c, rotate=0]{$ 16.88 \pm 0.03$};
\end{tikzpicture}
 
\begin{tikzpicture}
\pgfdeclareplotmark{cross} {
\pgfpathmoveto{\pgfpoint{-0.3\pgfplotmarksize}{\pgfplotmarksize}}
\pgfpathlineto{\pgfpoint{+0.3\pgfplotmarksize}{\pgfplotmarksize}}
\pgfpathlineto{\pgfpoint{+0.3\pgfplotmarksize}{0.3\pgfplotmarksize}}
\pgfpathlineto{\pgfpoint{+1\pgfplotmarksize}{0.3\pgfplotmarksize}}
\pgfpathlineto{\pgfpoint{+1\pgfplotmarksize}{-0.3\pgfplotmarksize}}
\pgfpathlineto{\pgfpoint{+0.3\pgfplotmarksize}{-0.3\pgfplotmarksize}}
\pgfpathlineto{\pgfpoint{+0.3\pgfplotmarksize}{-1.\pgfplotmarksize}}
\pgfpathlineto{\pgfpoint{-0.3\pgfplotmarksize}{-1.\pgfplotmarksize}}
\pgfpathlineto{\pgfpoint{-0.3\pgfplotmarksize}{-0.3\pgfplotmarksize}}
\pgfpathlineto{\pgfpoint{-1.\pgfplotmarksize}{-0.3\pgfplotmarksize}}
\pgfpathlineto{\pgfpoint{-1.\pgfplotmarksize}{0.3\pgfplotmarksize}}
\pgfpathlineto{\pgfpoint{-0.3\pgfplotmarksize}{0.3\pgfplotmarksize}}
\pgfpathclose
\pgfusepathqstroke
}
\pgfdeclareplotmark{cross*} {
\pgfpathmoveto{\pgfpoint{-0.3\pgfplotmarksize}{\pgfplotmarksize}}
\pgfpathlineto{\pgfpoint{+0.3\pgfplotmarksize}{\pgfplotmarksize}}
\pgfpathlineto{\pgfpoint{+0.3\pgfplotmarksize}{0.3\pgfplotmarksize}}
\pgfpathlineto{\pgfpoint{+1\pgfplotmarksize}{0.3\pgfplotmarksize}}
\pgfpathlineto{\pgfpoint{+1\pgfplotmarksize}{-0.3\pgfplotmarksize}}
\pgfpathlineto{\pgfpoint{+0.3\pgfplotmarksize}{-0.3\pgfplotmarksize}}
\pgfpathlineto{\pgfpoint{+0.3\pgfplotmarksize}{-1.\pgfplotmarksize}}
\pgfpathlineto{\pgfpoint{-0.3\pgfplotmarksize}{-1.\pgfplotmarksize}}
\pgfpathlineto{\pgfpoint{-0.3\pgfplotmarksize}{-0.3\pgfplotmarksize}}
\pgfpathlineto{\pgfpoint{-1.\pgfplotmarksize}{-0.3\pgfplotmarksize}}
\pgfpathlineto{\pgfpoint{-1.\pgfplotmarksize}{0.3\pgfplotmarksize}}
\pgfpathlineto{\pgfpoint{-0.3\pgfplotmarksize}{0.3\pgfplotmarksize}}
\pgfpathclose
\pgfusepathqfillstroke
}
\pgfdeclareplotmark{newstar} {
\pgfpathmoveto{\pgfqpoint{0pt}{\pgfplotmarksize}}
\pgfpathlineto{\pgfqpointpolar{44}{0.5\pgfplotmarksize}}
\pgfpathlineto{\pgfqpointpolar{18}{\pgfplotmarksize}}
\pgfpathlineto{\pgfqpointpolar{-20}{0.5\pgfplotmarksize}}
\pgfpathlineto{\pgfqpointpolar{-54}{\pgfplotmarksize}}
\pgfpathlineto{\pgfqpointpolar{-90}{0.5\pgfplotmarksize}}
\pgfpathlineto{\pgfqpointpolar{234}{\pgfplotmarksize}}
\pgfpathlineto{\pgfqpointpolar{198}{0.5\pgfplotmarksize}}
\pgfpathlineto{\pgfqpointpolar{162}{\pgfplotmarksize}}
\pgfpathlineto{\pgfqpointpolar{134}{0.5\pgfplotmarksize}}
\pgfpathclose
\pgfusepathqstroke
}
\pgfdeclareplotmark{newstar*} {
\pgfpathmoveto{\pgfqpoint{0pt}{\pgfplotmarksize}}
\pgfpathlineto{\pgfqpointpolar{44}{0.5\pgfplotmarksize}}
\pgfpathlineto{\pgfqpointpolar{18}{\pgfplotmarksize}}
\pgfpathlineto{\pgfqpointpolar{-20}{0.5\pgfplotmarksize}}
\pgfpathlineto{\pgfqpointpolar{-54}{\pgfplotmarksize}}
\pgfpathlineto{\pgfqpointpolar{-90}{0.5\pgfplotmarksize}}
\pgfpathlineto{\pgfqpointpolar{234}{\pgfplotmarksize}}
\pgfpathlineto{\pgfqpointpolar{198}{0.5\pgfplotmarksize}}
\pgfpathlineto{\pgfqpointpolar{162}{\pgfplotmarksize}}
\pgfpathlineto{\pgfqpointpolar{134}{0.5\pgfplotmarksize}}
\pgfpathclose
\pgfusepathqfillstroke
}
\definecolor{c}{rgb}{1,1,1};
\draw [color=c, fill=c] (0,0) rectangle (20,13.4957);
\draw [color=c, fill=c] (2,1.34957) rectangle (18,12.1461);
\definecolor{c}{rgb}{0,0,0};
\draw [c,line width=0.9] (2,1.34957) -- (2,12.1461) -- (18,12.1461) -- (18,1.34957) -- (2,1.34957);
\definecolor{c}{rgb}{1,1,1};
\draw [color=c, fill=c] (2,1.34957) rectangle (18,12.1461);
\definecolor{c}{rgb}{0,0,0};
\draw [c,line width=0.9] (2,1.34957) -- (2,12.1461) -- (18,12.1461) -- (18,1.34957) -- (2,1.34957);
\definecolor{c}{rgb}{0,0,0.6};
\draw [c,line width=0.9] (2,1.85425) -- (2.27586,1.85425) -- (2.27586,2.12495) -- (2.55172,2.12495) -- (2.55172,2.2979) -- (2.82759,2.2979) -- (2.82759,2.48087) -- (3.10345,2.48087) -- (3.10345,2.83177) -- (3.37931,2.83177) -- (3.37931,3.10247) --
 (3.65517,3.10247) -- (3.65517,3.43584) -- (3.93103,3.43584) -- (3.93103,3.78674) -- (4.2069,3.78674) -- (4.2069,4.0424) -- (4.48276,4.0424) -- (4.48276,4.47602) -- (4.75862,4.47602) -- (4.75862,4.7367) -- (5.03448,4.7367) -- (5.03448,5.10013) --
 (5.31034,5.10013) -- (5.31034,5.19037) -- (5.58621,5.19037) -- (5.58621,5.98241) -- (5.86207,5.98241) -- (5.86207,6.34335) -- (6.13793,6.34335) -- (6.13793,6.88725) -- (6.41379,6.88725) -- (6.41379,7.13289) -- (6.68966,7.13289) -- (6.68966,7.56149)
 -- (6.96552,7.56149) -- (6.96552,7.75199) -- (7.24138,7.75199) -- (7.24138,8.61923) -- (7.51724,8.61923) -- (7.51724,8.80721) -- (7.7931,8.80721) -- (7.7931,9.12052) -- (8.06897,9.12052) -- (8.06897,9.69701) -- (8.34483,9.69701) -- (8.34483,11.632)
 -- (8.62069,11.632) -- (8.62069,10.7222) -- (8.89655,10.7222) -- (8.89655,10.2309) -- (9.17241,10.2309) -- (9.17241,10.4364) -- (9.44828,10.4364) -- (9.44828,10.87) -- (9.72414,10.87) -- (9.72414,10.6069) -- (10,10.6069) -- (10,10.5693) --
 (10.2759,10.5693) -- (10.2759,10.6244) -- (10.5517,10.6244) -- (10.5517,10.3562) -- (10.8276,10.3562) -- (10.8276,10.5216) -- (11.1034,10.5216) -- (11.1034,10.7146) -- (11.3793,10.7146) -- (11.3793,10.1482) -- (11.6552,10.1482) -- (11.6552,9.79476)
 -- (11.931,9.79476) -- (11.931,9.40125) -- (12.2069,9.40125) -- (12.2069,9.14809) -- (12.4828,9.14809) -- (12.4828,8.82726) -- (12.7586,8.82726) -- (12.7586,8.81473) -- (13.0345,8.81473) -- (13.0345,7.91491) -- (13.3103,7.91491) -- (13.3103,7.86728)
 -- (13.5862,7.86728) -- (13.5862,7.20056) -- (13.8621,7.20056) -- (13.8621,6.68924) -- (14.1379,6.68924) -- (14.1379,6.29071) -- (14.4138,6.29071) -- (14.4138,5.73427) -- (14.6897,5.73427) -- (14.6897,5.34577) -- (14.9655,5.34577) --
 (14.9655,5.12019) -- (15.2414,5.12019) -- (15.2414,4.44845) -- (15.5172,4.44845) -- (15.5172,4.22788) -- (15.7931,4.22788) -- (15.7931,3.7717) -- (16.069,3.7717) -- (16.069,3.56617) -- (16.3448,3.56617) -- (16.3448,3.06237) -- (16.6207,3.06237) --
 (16.6207,2.69642) -- (16.8966,2.69642) -- (16.8966,2.5711) -- (17.1724,2.5711) -- (17.1724,2.16505) -- (17.4483,2.16505) -- (17.4483,2.01466) -- (17.7241,2.01466) -- (17.7241,1.83921) -- (18,1.83921);
\definecolor{c}{rgb}{1,1,1};
\draw [color=c, fill=c] (12.4,9.91934) rectangle (19.6,12.6185);
\definecolor{c}{rgb}{0,0,0};
\draw [c,line width=0.9] (12.4,9.91934) -- (19.6,9.91934);
\draw [c,line width=0.9] (19.6,9.91934) -- (19.6,12.6185);
\draw [c,line width=0.9] (19.6,12.6185) -- (12.4,12.6185);
\draw [c,line width=0.9] (12.4,12.6185) -- (12.4,9.91934);
\draw [anchor= west] (12.76,12.2811) node[scale=1.40004, color=c, rotate=0]{$\chi^{2} / ndf $};
\draw [anchor= east] (19.24,12.2811) node[scale=1.40004, color=c, rotate=0]{ 217.9 / 55};
\draw [anchor= west] (12.76,11.6063) node[scale=1.40004, color=c, rotate=0]{Constant };
\draw [anchor= east] (19.24,11.6063) node[scale=1.40004, color=c, rotate=0]{$  4189 \pm 14.5$};
\draw [anchor= west] (12.76,10.9315) node[scale=1.40004, color=c, rotate=0]{Mean     };
\draw [anchor= east] (19.24,10.9315) node[scale=1.40004, color=c, rotate=0]{$  1270 \pm 0.1$};
\draw [anchor= west] (12.76,10.2567) node[scale=1.40004, color=c, rotate=0]{Sigma    };
\draw [anchor= east] (19.24,10.2567) node[scale=1.40004, color=c, rotate=0]{$ 29.05 \pm 0.09$};
\definecolor{c}{rgb}{1,0,0};
\draw [c,line width=1.8] (2.08,1.68857) -- (2.24,1.80701) -- (2.4,1.93245) -- (2.56,2.06503) -- (2.72,2.20489) -- (2.88,2.35212) -- (3.04,2.50678) -- (3.2,2.6689) -- (3.36,2.83849) -- (3.52,3.0155) -- (3.68,3.19987) -- (3.84,3.39147) -- (4,3.59013)
 -- (4.16,3.79566) -- (4.32,4.0078) -- (4.48,4.22625) -- (4.64,4.45065) -- (4.8,4.68061) -- (4.96,4.91568) -- (5.12,5.15535) -- (5.28,5.39909) -- (5.44,5.6463) -- (5.6,5.89633) -- (5.76,6.1485) -- (5.92,6.40209) -- (6.08,6.65633) -- (6.24,6.91042) --
 (6.4,7.16353) -- (6.56,7.41481) -- (6.72,7.66335) -- (6.88,7.90828) -- (7.04,8.14867) -- (7.2,8.3836) -- (7.36,8.61216) -- (7.52,8.83342) -- (7.68,9.04647) -- (7.84,9.25043) -- (8,9.44443) -- (8.16,9.62763) -- (8.32,9.79922) -- (8.48,9.95844) --
 (8.64,10.1046) -- (8.8,10.2369) -- (8.96,10.3549) -- (9.12,10.458) -- (9.28,10.5457) -- (9.44,10.6176) -- (9.6,10.6734) -- (9.76,10.7127) -- (9.92,10.7355);
\draw [c,line width=1.8] (9.92,10.7355) -- (10.08,10.7415) -- (10.24,10.7308) -- (10.4,10.7034) -- (10.56,10.6594) -- (10.72,10.5991) -- (10.88,10.5227) -- (11.04,10.4307) -- (11.2,10.3234) -- (11.36,10.2013) -- (11.52,10.065) -- (11.68,9.9151) --
 (11.84,9.75234) -- (12,9.57742) -- (12.16,9.39111) -- (12.32,9.19423) -- (12.48,8.98764) -- (12.64,8.77219) -- (12.8,8.54879) -- (12.96,8.31836) -- (13.12,8.0818) -- (13.28,7.84005) -- (13.44,7.59402) -- (13.6,7.34461) -- (13.76,7.09274) --
 (13.92,6.83927) -- (14.08,6.58505) -- (14.24,6.33091) -- (14.4,6.07765) -- (14.56,5.826) -- (14.72,5.57669) -- (14.88,5.3304) -- (15.04,5.08774) -- (15.2,4.8493) -- (15.36,4.61562) -- (15.52,4.38717) -- (15.68,4.1644) -- (15.84,3.94769) --
 (16,3.73738) -- (16.16,3.53375) -- (16.32,3.33704) -- (16.48,3.14746) -- (16.64,2.96514) -- (16.8,2.7902) -- (16.96,2.6227) -- (17.12,2.46268) -- (17.28,2.31011) -- (17.44,2.16495) -- (17.6,2.02714) -- (17.76,1.89657);
\draw [c,line width=1.8] (17.76,1.89657) -- (17.92,1.77311);
\definecolor{c}{rgb}{0,0,0};
\draw [c,line width=0.9] (2,1.34957) -- (18,1.34957);
\draw [c,line width=0.9] (3.10345,1.67347) -- (3.10345,1.34957);
\draw [c,line width=0.9] (3.7931,1.51152) -- (3.7931,1.34957);
\draw [c,line width=0.9] (4.48276,1.51152) -- (4.48276,1.34957);
\draw [c,line width=0.9] (5.17241,1.51152) -- (5.17241,1.34957);
\draw [c,line width=0.9] (5.86207,1.67347) -- (5.86207,1.34957);
\draw [c,line width=0.9] (6.55172,1.51152) -- (6.55172,1.34957);
\draw [c,line width=0.9] (7.24138,1.51152) -- (7.24138,1.34957);
\draw [c,line width=0.9] (7.93103,1.51152) -- (7.93103,1.34957);
\draw [c,line width=0.9] (8.62069,1.67347) -- (8.62069,1.34957);
\draw [c,line width=0.9] (9.31034,1.51152) -- (9.31034,1.34957);
\draw [c,line width=0.9] (10,1.51152) -- (10,1.34957);
\draw [c,line width=0.9] (10.6897,1.51152) -- (10.6897,1.34957);
\draw [c,line width=0.9] (11.3793,1.67347) -- (11.3793,1.34957);
\draw [c,line width=0.9] (12.069,1.51152) -- (12.069,1.34957);
\draw [c,line width=0.9] (12.7586,1.51152) -- (12.7586,1.34957);
\draw [c,line width=0.9] (13.4483,1.51152) -- (13.4483,1.34957);
\draw [c,line width=0.9] (14.1379,1.67347) -- (14.1379,1.34957);
\draw [c,line width=0.9] (14.8276,1.51152) -- (14.8276,1.34957);
\draw [c,line width=0.9] (15.5172,1.51152) -- (15.5172,1.34957);
\draw [c,line width=0.9] (16.2069,1.51152) -- (16.2069,1.34957);
\draw [c,line width=0.9] (16.8966,1.67347) -- (16.8966,1.34957);
\draw [c,line width=0.9] (3.10345,1.67347) -- (3.10345,1.34957);
\draw [c,line width=0.9] (2.41379,1.51152) -- (2.41379,1.34957);
\draw [c,line width=0.9] (16.8966,1.67347) -- (16.8966,1.34957);
\draw [c,line width=0.9] (17.5862,1.51152) -- (17.5862,1.34957);
\draw [anchor=base] (3.10345,0.904212) node[scale=1.01821, color=c, rotate=0]{1220};
\draw [anchor=base] (5.86207,0.904212) node[scale=1.01821, color=c, rotate=0]{1240};
\draw [anchor=base] (8.62069,0.904212) node[scale=1.01821, color=c, rotate=0]{1260};
\draw [anchor=base] (11.3793,0.904212) node[scale=1.01821, color=c, rotate=0]{1280};
\draw [anchor=base] (14.1379,0.904212) node[scale=1.01821, color=c, rotate=0]{1300};
\draw [anchor=base] (16.8966,0.904212) node[scale=1.01821, color=c, rotate=0]{1320};
\draw [c,line width=0.9] (2,1.34957) -- (2,12.1461);
\draw [c,line width=0.9] (2.48,1.49582) -- (2,1.49582);
\draw [c,line width=0.9] (2.24,1.74647) -- (2,1.74647);
\draw [c,line width=0.9] (2.24,1.99712) -- (2,1.99712);
\draw [c,line width=0.9] (2.24,2.24777) -- (2,2.24777);
\draw [c,line width=0.9] (2.24,2.49841) -- (2,2.49841);
\draw [c,line width=0.9] (2.48,2.74906) -- (2,2.74906);
\draw [c,line width=0.9] (2.24,2.99971) -- (2,2.99971);
\draw [c,line width=0.9] (2.24,3.25036) -- (2,3.25036);
\draw [c,line width=0.9] (2.24,3.501) -- (2,3.501);
\draw [c,line width=0.9] (2.24,3.75165) -- (2,3.75165);
\draw [c,line width=0.9] (2.48,4.0023) -- (2,4.0023);
\draw [c,line width=0.9] (2.24,4.25295) -- (2,4.25295);
\draw [c,line width=0.9] (2.24,4.50359) -- (2,4.50359);
\draw [c,line width=0.9] (2.24,4.75424) -- (2,4.75424);
\draw [c,line width=0.9] (2.24,5.00489) -- (2,5.00489);
\draw [c,line width=0.9] (2.48,5.25554) -- (2,5.25554);
\draw [c,line width=0.9] (2.24,5.50618) -- (2,5.50618);
\draw [c,line width=0.9] (2.24,5.75683) -- (2,5.75683);
\draw [c,line width=0.9] (2.24,6.00748) -- (2,6.00748);
\draw [c,line width=0.9] (2.24,6.25813) -- (2,6.25813);
\draw [c,line width=0.9] (2.48,6.50877) -- (2,6.50877);
\draw [c,line width=0.9] (2.24,6.75942) -- (2,6.75942);
\draw [c,line width=0.9] (2.24,7.01007) -- (2,7.01007);
\draw [c,line width=0.9] (2.24,7.26072) -- (2,7.26072);
\draw [c,line width=0.9] (2.24,7.51136) -- (2,7.51136);
\draw [c,line width=0.9] (2.48,7.76201) -- (2,7.76201);
\draw [c,line width=0.9] (2.24,8.01266) -- (2,8.01266);
\draw [c,line width=0.9] (2.24,8.26331) -- (2,8.26331);
\draw [c,line width=0.9] (2.24,8.51395) -- (2,8.51395);
\draw [c,line width=0.9] (2.24,8.7646) -- (2,8.7646);
\draw [c,line width=0.9] (2.48,9.01525) -- (2,9.01525);
\draw [c,line width=0.9] (2.24,9.2659) -- (2,9.2659);
\draw [c,line width=0.9] (2.24,9.51654) -- (2,9.51654);
\draw [c,line width=0.9] (2.24,9.76719) -- (2,9.76719);
\draw [c,line width=0.9] (2.24,10.0178) -- (2,10.0178);
\draw [c,line width=0.9] (2.48,10.2685) -- (2,10.2685);
\draw [c,line width=0.9] (2.24,10.5191) -- (2,10.5191);
\draw [c,line width=0.9] (2.24,10.7698) -- (2,10.7698);
\draw [c,line width=0.9] (2.24,11.0204) -- (2,11.0204);
\draw [c,line width=0.9] (2.24,11.2711) -- (2,11.2711);
\draw [c,line width=0.9] (2.48,11.5217) -- (2,11.5217);
\draw [c,line width=0.9] (2.48,1.49582) -- (2,1.49582);
\draw [c,line width=0.9] (2.48,11.5217) -- (2,11.5217);
\draw [c,line width=0.9] (2.24,11.7724) -- (2,11.7724);
\draw [c,line width=0.9] (2.24,12.023) -- (2,12.023);
\draw [anchor= east] (1.9,1.49582) node[scale=1.01821, color=c, rotate=0]{500};
\draw [anchor= east] (1.9,2.74906) node[scale=1.01821, color=c, rotate=0]{1000};
\draw [anchor= east] (1.9,4.0023) node[scale=1.01821, color=c, rotate=0]{1500};
\draw [anchor= east] (1.9,5.25554) node[scale=1.01821, color=c, rotate=0]{2000};
\draw [anchor= east] (1.9,6.50877) node[scale=1.01821, color=c, rotate=0]{2500};
\draw [anchor= east] (1.9,7.76201) node[scale=1.01821, color=c, rotate=0]{3000};
\draw [anchor= east] (1.9,9.01525) node[scale=1.01821, color=c, rotate=0]{3500};
\draw [anchor= east] (1.9,10.2685) node[scale=1.01821, color=c, rotate=0]{4000};
\draw [anchor= east] (1.9,11.5217) node[scale=1.01821, color=c, rotate=0]{4500};
\definecolor{c}{rgb}{1,1,1};
\draw [color=c, fill=c] (12.4,9.91934) rectangle (19.6,12.6185);
\definecolor{c}{rgb}{0,0,0};
\draw [c,line width=0.9] (12.4,9.91934) -- (19.6,9.91934);
\draw [c,line width=0.9] (19.6,9.91934) -- (19.6,12.6185);
\draw [c,line width=0.9] (19.6,12.6185) -- (12.4,12.6185);
\draw [c,line width=0.9] (12.4,12.6185) -- (12.4,9.91934);
\draw [anchor= west] (12.76,12.2811) node[scale=1.40004, color=c, rotate=0]{$\chi^{2} / ndf $};
\draw [anchor= east] (19.24,12.2811) node[scale=1.40004, color=c, rotate=0]{ 217.9 / 55};
\draw [anchor= west] (12.76,11.6063) node[scale=1.40004, color=c, rotate=0]{Constant };
\draw [anchor= east] (19.24,11.6063) node[scale=1.40004, color=c, rotate=0]{$  4189 \pm 14.5$};
\draw [anchor= west] (12.76,10.9315) node[scale=1.40004, color=c, rotate=0]{Mean     };
\draw [anchor= east] (19.24,10.9315) node[scale=1.40004, color=c, rotate=0]{$  1270 \pm 0.1$};
\draw [anchor= west] (12.76,10.2567) node[scale=1.40004, color=c, rotate=0]{Sigma    };
\draw [anchor= east] (19.24,10.2567) node[scale=1.40004, color=c, rotate=0]{$ 29.05 \pm 0.09$};
\end{tikzpicture}
 
\begin{tikzpicture}
\pgfdeclareplotmark{cross} {
\pgfpathmoveto{\pgfpoint{-0.3\pgfplotmarksize}{\pgfplotmarksize}}
\pgfpathlineto{\pgfpoint{+0.3\pgfplotmarksize}{\pgfplotmarksize}}
\pgfpathlineto{\pgfpoint{+0.3\pgfplotmarksize}{0.3\pgfplotmarksize}}
\pgfpathlineto{\pgfpoint{+1\pgfplotmarksize}{0.3\pgfplotmarksize}}
\pgfpathlineto{\pgfpoint{+1\pgfplotmarksize}{-0.3\pgfplotmarksize}}
\pgfpathlineto{\pgfpoint{+0.3\pgfplotmarksize}{-0.3\pgfplotmarksize}}
\pgfpathlineto{\pgfpoint{+0.3\pgfplotmarksize}{-1.\pgfplotmarksize}}
\pgfpathlineto{\pgfpoint{-0.3\pgfplotmarksize}{-1.\pgfplotmarksize}}
\pgfpathlineto{\pgfpoint{-0.3\pgfplotmarksize}{-0.3\pgfplotmarksize}}
\pgfpathlineto{\pgfpoint{-1.\pgfplotmarksize}{-0.3\pgfplotmarksize}}
\pgfpathlineto{\pgfpoint{-1.\pgfplotmarksize}{0.3\pgfplotmarksize}}
\pgfpathlineto{\pgfpoint{-0.3\pgfplotmarksize}{0.3\pgfplotmarksize}}
\pgfpathclose
\pgfusepathqstroke
}
\pgfdeclareplotmark{cross*} {
\pgfpathmoveto{\pgfpoint{-0.3\pgfplotmarksize}{\pgfplotmarksize}}
\pgfpathlineto{\pgfpoint{+0.3\pgfplotmarksize}{\pgfplotmarksize}}
\pgfpathlineto{\pgfpoint{+0.3\pgfplotmarksize}{0.3\pgfplotmarksize}}
\pgfpathlineto{\pgfpoint{+1\pgfplotmarksize}{0.3\pgfplotmarksize}}
\pgfpathlineto{\pgfpoint{+1\pgfplotmarksize}{-0.3\pgfplotmarksize}}
\pgfpathlineto{\pgfpoint{+0.3\pgfplotmarksize}{-0.3\pgfplotmarksize}}
\pgfpathlineto{\pgfpoint{+0.3\pgfplotmarksize}{-1.\pgfplotmarksize}}
\pgfpathlineto{\pgfpoint{-0.3\pgfplotmarksize}{-1.\pgfplotmarksize}}
\pgfpathlineto{\pgfpoint{-0.3\pgfplotmarksize}{-0.3\pgfplotmarksize}}
\pgfpathlineto{\pgfpoint{-1.\pgfplotmarksize}{-0.3\pgfplotmarksize}}
\pgfpathlineto{\pgfpoint{-1.\pgfplotmarksize}{0.3\pgfplotmarksize}}
\pgfpathlineto{\pgfpoint{-0.3\pgfplotmarksize}{0.3\pgfplotmarksize}}
\pgfpathclose
\pgfusepathqfillstroke
}
\pgfdeclareplotmark{newstar} {
\pgfpathmoveto{\pgfqpoint{0pt}{\pgfplotmarksize}}
\pgfpathlineto{\pgfqpointpolar{44}{0.5\pgfplotmarksize}}
\pgfpathlineto{\pgfqpointpolar{18}{\pgfplotmarksize}}
\pgfpathlineto{\pgfqpointpolar{-20}{0.5\pgfplotmarksize}}
\pgfpathlineto{\pgfqpointpolar{-54}{\pgfplotmarksize}}
\pgfpathlineto{\pgfqpointpolar{-90}{0.5\pgfplotmarksize}}
\pgfpathlineto{\pgfqpointpolar{234}{\pgfplotmarksize}}
\pgfpathlineto{\pgfqpointpolar{198}{0.5\pgfplotmarksize}}
\pgfpathlineto{\pgfqpointpolar{162}{\pgfplotmarksize}}
\pgfpathlineto{\pgfqpointpolar{134}{0.5\pgfplotmarksize}}
\pgfpathclose
\pgfusepathqstroke
}
\pgfdeclareplotmark{newstar*} {
\pgfpathmoveto{\pgfqpoint{0pt}{\pgfplotmarksize}}
\pgfpathlineto{\pgfqpointpolar{44}{0.5\pgfplotmarksize}}
\pgfpathlineto{\pgfqpointpolar{18}{\pgfplotmarksize}}
\pgfpathlineto{\pgfqpointpolar{-20}{0.5\pgfplotmarksize}}
\pgfpathlineto{\pgfqpointpolar{-54}{\pgfplotmarksize}}
\pgfpathlineto{\pgfqpointpolar{-90}{0.5\pgfplotmarksize}}
\pgfpathlineto{\pgfqpointpolar{234}{\pgfplotmarksize}}
\pgfpathlineto{\pgfqpointpolar{198}{0.5\pgfplotmarksize}}
\pgfpathlineto{\pgfqpointpolar{162}{\pgfplotmarksize}}
\pgfpathlineto{\pgfqpointpolar{134}{0.5\pgfplotmarksize}}
\pgfpathclose
\pgfusepathqfillstroke
}
\definecolor{c}{rgb}{1,1,1};
\draw [color=c, fill=c] (0,0) rectangle (20,13.4957);
\draw [color=c, fill=c] (2,1.34957) rectangle (18,12.1461);
\definecolor{c}{rgb}{0,0,0};
\draw [c,line width=0.9] (2,1.34957) -- (2,12.1461) -- (18,12.1461) -- (18,1.34957) -- (2,1.34957);
\definecolor{c}{rgb}{1,1,1};
\draw [color=c, fill=c] (2,1.34957) rectangle (18,12.1461);
\definecolor{c}{rgb}{0,0,0};
\draw [c,line width=0.9] (2,1.34957) -- (2,12.1461) -- (18,12.1461) -- (18,1.34957) -- (2,1.34957);
\definecolor{c}{rgb}{0,0,0.6};
\draw [c,line width=0.9] (2,2.29722) -- (2.48485,2.29722) -- (2.48485,2.78025) -- (2.9697,2.78025) -- (2.9697,3.38609) -- (3.45455,3.38609) -- (3.45455,3.9698) -- (3.93939,3.9698) -- (3.93939,4.54537) -- (4.42424,4.54537) -- (4.42424,5.25071) --
 (4.90909,5.25071) -- (4.90909,5.98574) -- (5.39394,5.98574) -- (5.39394,6.77548) -- (5.87879,6.77548) -- (5.87879,7.48082) -- (6.36364,7.48082) -- (6.36364,8.44107) -- (6.84848,8.44107) -- (6.84848,9.05563) -- (7.33333,9.05563) -- (7.33333,9.84653)
 -- (7.81818,9.84653) -- (7.81818,10.3185) -- (8.30303,10.3185) -- (8.30303,10.7986) -- (8.78788,10.7986) -- (8.78788,11.27) -- (9.27273,11.27) -- (9.27273,11.323) -- (9.75758,11.323) -- (9.75758,11.632) -- (10.2424,11.632) -- (10.2424,11.2619) --
 (10.7273,11.2619) -- (10.7273,11.2671) -- (11.2121,11.2671) -- (11.2121,11.11) -- (11.697,11.11) -- (11.697,10.8196) -- (12.1818,10.8196) -- (12.1818,10.1171) -- (12.6667,10.1171) -- (12.6667,9.27503) -- (13.1515,9.27503) -- (13.1515,8.69947) --
 (13.6364,8.69947) -- (13.6364,8.00285) -- (14.1212,8.00285) -- (14.1212,7.02398) -- (14.6061,7.02398) -- (14.6061,6.26742) -- (15.0909,6.26742) -- (15.0909,5.43694) -- (15.5758,5.43694) -- (15.5758,4.6379) -- (16.0606,4.6379) -- (16.0606,3.84351) --
 (16.5455,3.84351) -- (16.5455,2.99442) -- (17.0303,2.99442) -- (17.0303,2.63302) -- (17.5152,2.63302) -- (17.5152,1.83921) -- (18,1.83921);
\definecolor{c}{rgb}{1,0,0};
\draw [c,line width=1.8] (2.08,1.84973) -- (2.24,2.02159) -- (2.4,2.2002) -- (2.56,2.38553) -- (2.72,2.5775) -- (2.88,2.77601) -- (3.04,2.98095) -- (3.2,3.19214) -- (3.36,3.4094) -- (3.52,3.63251) -- (3.68,3.8612) -- (3.84,4.09519) -- (4,4.33415) --
 (4.16,4.57773) -- (4.32,4.82551) -- (4.48,5.07709) -- (4.64,5.332) -- (4.8,5.58974) -- (4.96,5.84979) -- (5.12,6.11161) -- (5.28,6.3746) -- (5.44,6.63816) -- (5.6,6.90165) -- (5.76,7.16442) -- (5.92,7.4258) -- (6.08,7.6851) -- (6.24,7.9416) --
 (6.4,8.1946) -- (6.56,8.44336) -- (6.72,8.68716) -- (6.88,8.92527) -- (7.04,9.15696) -- (7.2,9.38151) -- (7.36,9.59821) -- (7.52,9.80635) -- (7.68,10.0053) -- (7.84,10.1943) -- (8,10.3728) -- (8.16,10.5402) -- (8.32,10.6958) -- (8.48,10.8392) --
 (8.64,10.9699) -- (8.8,11.0873) -- (8.96,11.1912) -- (9.12,11.281) -- (9.28,11.3565) -- (9.44,11.4174) -- (9.6,11.4635) -- (9.76,11.4946) -- (9.92,11.5106);
\draw [c,line width=1.8] (9.92,11.5106) -- (10.08,11.5114) -- (10.24,11.4971) -- (10.4,11.4677) -- (10.56,11.4232) -- (10.72,11.364) -- (10.88,11.2901) -- (11.04,11.2018) -- (11.2,11.0995) -- (11.36,10.9836) -- (11.52,10.8544) -- (11.68,10.7124) --
 (11.84,10.558) -- (12,10.3919) -- (12.16,10.2147) -- (12.32,10.0268) -- (12.48,9.82892) -- (12.64,9.62176) -- (12.8,9.40598) -- (12.96,9.18227) -- (13.12,8.95134) -- (13.28,8.71391) -- (13.44,8.4707) -- (13.6,8.22245) -- (13.76,7.96989) --
 (13.92,7.71374) -- (14.08,7.45472) -- (14.24,7.19354) -- (14.4,6.93089) -- (14.56,6.66744) -- (14.72,6.40386) -- (14.88,6.14077) -- (15.04,5.8788) -- (15.2,5.61852) -- (15.36,5.3605) -- (15.52,5.10525) -- (15.68,4.85328) -- (15.84,4.60505) --
 (16,4.36099) -- (16.16,4.1215) -- (16.32,3.88694) -- (16.48,3.65764) -- (16.64,3.4339) -- (16.8,3.21598) -- (16.96,3.0041) -- (17.12,2.79846) -- (17.28,2.59923) -- (17.44,2.40653) -- (17.6,2.22046) -- (17.76,2.0411);
\draw [c,line width=1.8] (17.76,2.0411) -- (17.92,1.86848);
\definecolor{c}{rgb}{1,1,1};
\draw [color=c, fill=c] (12.4,9.91934) rectangle (19.6,12.6185);
\definecolor{c}{rgb}{0,0,0};
\draw [c,line width=0.9] (12.4,9.91934) -- (19.6,9.91934);
\draw [c,line width=0.9] (19.6,9.91934) -- (19.6,12.6185);
\draw [c,line width=0.9] (19.6,12.6185) -- (12.4,12.6185);
\draw [c,line width=0.9] (12.4,12.6185) -- (12.4,9.91934);
\draw [anchor= west] (12.76,12.2811) node[scale=1.08185, color=c, rotate=0]{$\chi^{2} / ndf $};
\draw [anchor= east] (19.24,12.2811) node[scale=1.08185, color=c, rotate=0]{ 227.7 / 30};
\draw [anchor= west] (12.76,11.6063) node[scale=1.08185, color=c, rotate=0]{Constant };
\draw [anchor= east] (19.24,11.6063) node[scale=1.08185, color=c, rotate=0]{$ 2.14e+04 \pm 4.23e+01$};
\draw [anchor= west] (12.76,10.9315) node[scale=1.08185, color=c, rotate=0]{Mean     };
\draw [anchor= east] (19.24,10.9315) node[scale=1.08185, color=c, rotate=0]{$   539 \pm 0.0$};
\draw [anchor= west] (12.76,10.2567) node[scale=1.08185, color=c, rotate=0]{Sigma    };
\draw [anchor= east] (19.24,10.2567) node[scale=1.08185, color=c, rotate=0]{$ 18.91 \pm 0.04$};
\draw [c,line width=0.9] (2,1.34957) -- (18,1.34957);
\draw [c,line width=0.9] (2.9697,1.67347) -- (2.9697,1.34957);
\draw [c,line width=0.9] (3.45455,1.51152) -- (3.45455,1.34957);
\draw [c,line width=0.9] (3.93939,1.51152) -- (3.93939,1.34957);
\draw [c,line width=0.9] (4.42424,1.51152) -- (4.42424,1.34957);
\draw [c,line width=0.9] (4.90909,1.51152) -- (4.90909,1.34957);
\draw [c,line width=0.9] (5.39394,1.67347) -- (5.39394,1.34957);
\draw [c,line width=0.9] (5.87879,1.51152) -- (5.87879,1.34957);
\draw [c,line width=0.9] (6.36364,1.51152) -- (6.36364,1.34957);
\draw [c,line width=0.9] (6.84848,1.51152) -- (6.84848,1.34957);
\draw [c,line width=0.9] (7.33333,1.51152) -- (7.33333,1.34957);
\draw [c,line width=0.9] (7.81818,1.67347) -- (7.81818,1.34957);
\draw [c,line width=0.9] (8.30303,1.51152) -- (8.30303,1.34957);
\draw [c,line width=0.9] (8.78788,1.51152) -- (8.78788,1.34957);
\draw [c,line width=0.9] (9.27273,1.51152) -- (9.27273,1.34957);
\draw [c,line width=0.9] (9.75758,1.51152) -- (9.75758,1.34957);
\draw [c,line width=0.9] (10.2424,1.67347) -- (10.2424,1.34957);
\draw [c,line width=0.9] (10.7273,1.51152) -- (10.7273,1.34957);
\draw [c,line width=0.9] (11.2121,1.51152) -- (11.2121,1.34957);
\draw [c,line width=0.9] (11.697,1.51152) -- (11.697,1.34957);
\draw [c,line width=0.9] (12.1818,1.51152) -- (12.1818,1.34957);
\draw [c,line width=0.9] (12.6667,1.67347) -- (12.6667,1.34957);
\draw [c,line width=0.9] (13.1515,1.51152) -- (13.1515,1.34957);
\draw [c,line width=0.9] (13.6364,1.51152) -- (13.6364,1.34957);
\draw [c,line width=0.9] (14.1212,1.51152) -- (14.1212,1.34957);
\draw [c,line width=0.9] (14.6061,1.51152) -- (14.6061,1.34957);
\draw [c,line width=0.9] (15.0909,1.67347) -- (15.0909,1.34957);
\draw [c,line width=0.9] (15.5758,1.51152) -- (15.5758,1.34957);
\draw [c,line width=0.9] (16.0606,1.51152) -- (16.0606,1.34957);
\draw [c,line width=0.9] (16.5455,1.51152) -- (16.5455,1.34957);
\draw [c,line width=0.9] (17.0303,1.51152) -- (17.0303,1.34957);
\draw [c,line width=0.9] (17.5152,1.67347) -- (17.5152,1.34957);
\draw [c,line width=0.9] (2.9697,1.67347) -- (2.9697,1.34957);
\draw [c,line width=0.9] (2.48485,1.51152) -- (2.48485,1.34957);
\draw [c,line width=0.9] (2,1.51152) -- (2,1.34957);
\draw [c,line width=0.9] (17.5152,1.67347) -- (17.5152,1.34957);
\draw [c,line width=0.9] (18,1.51152) -- (18,1.34957);
\draw [anchor=base] (2.9697,0.904212) node[scale=1.01821, color=c, rotate=0]{510};
\draw [anchor=base] (5.39394,0.904212) node[scale=1.01821, color=c, rotate=0]{520};
\draw [anchor=base] (7.81818,0.904212) node[scale=1.01821, color=c, rotate=0]{530};
\draw [anchor=base] (10.2424,0.904212) node[scale=1.01821, color=c, rotate=0]{540};
\draw [anchor=base] (12.6667,0.904212) node[scale=1.01821, color=c, rotate=0]{550};
\draw [anchor=base] (15.0909,0.904212) node[scale=1.01821, color=c, rotate=0]{560};
\draw [anchor=base] (17.5152,0.904212) node[scale=1.01821, color=c, rotate=0]{570};
\draw [c,line width=0.9] (2,1.34957) -- (2,12.1461);
\draw [c,line width=0.9] (2.48,1.38644) -- (2,1.38644);
\draw [c,line width=0.9] (2.24,1.67742) -- (2,1.67742);
\draw [c,line width=0.9] (2.24,1.96841) -- (2,1.96841);
\draw [c,line width=0.9] (2.24,2.25939) -- (2,2.25939);
\draw [c,line width=0.9] (2.48,2.55038) -- (2,2.55038);
\draw [c,line width=0.9] (2.24,2.84136) -- (2,2.84136);
\draw [c,line width=0.9] (2.24,3.13235) -- (2,3.13235);
\draw [c,line width=0.9] (2.24,3.42333) -- (2,3.42333);
\draw [c,line width=0.9] (2.48,3.71432) -- (2,3.71432);
\draw [c,line width=0.9] (2.24,4.0053) -- (2,4.0053);
\draw [c,line width=0.9] (2.24,4.29629) -- (2,4.29629);
\draw [c,line width=0.9] (2.24,4.58727) -- (2,4.58727);
\draw [c,line width=0.9] (2.48,4.87825) -- (2,4.87825);
\draw [c,line width=0.9] (2.24,5.16924) -- (2,5.16924);
\draw [c,line width=0.9] (2.24,5.46022) -- (2,5.46022);
\draw [c,line width=0.9] (2.24,5.75121) -- (2,5.75121);
\draw [c,line width=0.9] (2.48,6.04219) -- (2,6.04219);
\draw [c,line width=0.9] (2.24,6.33318) -- (2,6.33318);
\draw [c,line width=0.9] (2.24,6.62416) -- (2,6.62416);
\draw [c,line width=0.9] (2.24,6.91515) -- (2,6.91515);
\draw [c,line width=0.9] (2.48,7.20613) -- (2,7.20613);
\draw [c,line width=0.9] (2.24,7.49712) -- (2,7.49712);
\draw [c,line width=0.9] (2.24,7.7881) -- (2,7.7881);
\draw [c,line width=0.9] (2.24,8.07909) -- (2,8.07909);
\draw [c,line width=0.9] (2.48,8.37007) -- (2,8.37007);
\draw [c,line width=0.9] (2.24,8.66106) -- (2,8.66106);
\draw [c,line width=0.9] (2.24,8.95204) -- (2,8.95204);
\draw [c,line width=0.9] (2.24,9.24302) -- (2,9.24302);
\draw [c,line width=0.9] (2.48,9.53401) -- (2,9.53401);
\draw [c,line width=0.9] (2.24,9.82499) -- (2,9.82499);
\draw [c,line width=0.9] (2.24,10.116) -- (2,10.116);
\draw [c,line width=0.9] (2.24,10.407) -- (2,10.407);
\draw [c,line width=0.9] (2.48,10.6979) -- (2,10.6979);
\draw [c,line width=0.9] (2.24,10.9889) -- (2,10.9889);
\draw [c,line width=0.9] (2.24,11.2799) -- (2,11.2799);
\draw [c,line width=0.9] (2.24,11.5709) -- (2,11.5709);
\draw [c,line width=0.9] (2.48,11.8619) -- (2,11.8619);
\draw [c,line width=0.9] (2.48,1.38644) -- (2,1.38644);
\draw [c,line width=0.9] (2.48,11.8619) -- (2,11.8619);
\draw [anchor= east] (1.9,1.38644) node[scale=1.01821, color=c, rotate=0]{4000};
\draw [anchor= east] (1.9,2.55038) node[scale=1.01821, color=c, rotate=0]{6000};
\draw [anchor= east] (1.9,3.71432) node[scale=1.01821, color=c, rotate=0]{8000};
\draw [anchor= east] (1.9,4.87825) node[scale=1.01821, color=c, rotate=0]{10000};
\draw [anchor= east] (1.9,6.04219) node[scale=1.01821, color=c, rotate=0]{12000};
\draw [anchor= east] (1.9,7.20613) node[scale=1.01821, color=c, rotate=0]{14000};
\draw [anchor= east] (1.9,8.37007) node[scale=1.01821, color=c, rotate=0]{16000};
\draw [anchor= east] (1.9,9.53401) node[scale=1.01821, color=c, rotate=0]{18000};
\draw [anchor= east] (1.9,10.6979) node[scale=1.01821, color=c, rotate=0]{20000};
\draw [anchor= east] (1.9,11.8619) node[scale=1.01821, color=c, rotate=0]{22000};
\definecolor{c}{rgb}{1,1,1};
\draw [color=c, fill=c] (12.4,9.91934) rectangle (19.6,12.6185);
\definecolor{c}{rgb}{0,0,0};
\draw [c,line width=0.9] (12.4,9.91934) -- (19.6,9.91934);
\draw [c,line width=0.9] (19.6,9.91934) -- (19.6,12.6185);
\draw [c,line width=0.9] (19.6,12.6185) -- (12.4,12.6185);
\draw [c,line width=0.9] (12.4,12.6185) -- (12.4,9.91934);
\draw [anchor= west] (12.76,12.2811) node[scale=1.08185, color=c, rotate=0]{$\chi^{2} / ndf $};
\draw [anchor= east] (19.24,12.2811) node[scale=1.08185, color=c, rotate=0]{ 227.7 / 30};
\draw [anchor= west] (12.76,11.6063) node[scale=1.08185, color=c, rotate=0]{Constant };
\draw [anchor= east] (19.24,11.6063) node[scale=1.08185, color=c, rotate=0]{$ 2.14e+04 \pm 4.23e+01$};
\draw [anchor= west] (12.76,10.9315) node[scale=1.08185, color=c, rotate=0]{Mean     };
\draw [anchor= east] (19.24,10.9315) node[scale=1.08185, color=c, rotate=0]{$   539 \pm 0.0$};
\draw [anchor= west] (12.76,10.2567) node[scale=1.08185, color=c, rotate=0]{Sigma    };
\draw [anchor= east] (19.24,10.2567) node[scale=1.08185, color=c, rotate=0]{$ 18.91 \pm 0.04$};
\end{tikzpicture}
 
\begin{tikzpicture}
\pgfdeclareplotmark{cross} {
\pgfpathmoveto{\pgfpoint{-0.3\pgfplotmarksize}{\pgfplotmarksize}}
\pgfpathlineto{\pgfpoint{+0.3\pgfplotmarksize}{\pgfplotmarksize}}
\pgfpathlineto{\pgfpoint{+0.3\pgfplotmarksize}{0.3\pgfplotmarksize}}
\pgfpathlineto{\pgfpoint{+1\pgfplotmarksize}{0.3\pgfplotmarksize}}
\pgfpathlineto{\pgfpoint{+1\pgfplotmarksize}{-0.3\pgfplotmarksize}}
\pgfpathlineto{\pgfpoint{+0.3\pgfplotmarksize}{-0.3\pgfplotmarksize}}
\pgfpathlineto{\pgfpoint{+0.3\pgfplotmarksize}{-1.\pgfplotmarksize}}
\pgfpathlineto{\pgfpoint{-0.3\pgfplotmarksize}{-1.\pgfplotmarksize}}
\pgfpathlineto{\pgfpoint{-0.3\pgfplotmarksize}{-0.3\pgfplotmarksize}}
\pgfpathlineto{\pgfpoint{-1.\pgfplotmarksize}{-0.3\pgfplotmarksize}}
\pgfpathlineto{\pgfpoint{-1.\pgfplotmarksize}{0.3\pgfplotmarksize}}
\pgfpathlineto{\pgfpoint{-0.3\pgfplotmarksize}{0.3\pgfplotmarksize}}
\pgfpathclose
\pgfusepathqstroke
}
\pgfdeclareplotmark{cross*} {
\pgfpathmoveto{\pgfpoint{-0.3\pgfplotmarksize}{\pgfplotmarksize}}
\pgfpathlineto{\pgfpoint{+0.3\pgfplotmarksize}{\pgfplotmarksize}}
\pgfpathlineto{\pgfpoint{+0.3\pgfplotmarksize}{0.3\pgfplotmarksize}}
\pgfpathlineto{\pgfpoint{+1\pgfplotmarksize}{0.3\pgfplotmarksize}}
\pgfpathlineto{\pgfpoint{+1\pgfplotmarksize}{-0.3\pgfplotmarksize}}
\pgfpathlineto{\pgfpoint{+0.3\pgfplotmarksize}{-0.3\pgfplotmarksize}}
\pgfpathlineto{\pgfpoint{+0.3\pgfplotmarksize}{-1.\pgfplotmarksize}}
\pgfpathlineto{\pgfpoint{-0.3\pgfplotmarksize}{-1.\pgfplotmarksize}}
\pgfpathlineto{\pgfpoint{-0.3\pgfplotmarksize}{-0.3\pgfplotmarksize}}
\pgfpathlineto{\pgfpoint{-1.\pgfplotmarksize}{-0.3\pgfplotmarksize}}
\pgfpathlineto{\pgfpoint{-1.\pgfplotmarksize}{0.3\pgfplotmarksize}}
\pgfpathlineto{\pgfpoint{-0.3\pgfplotmarksize}{0.3\pgfplotmarksize}}
\pgfpathclose
\pgfusepathqfillstroke
}
\pgfdeclareplotmark{newstar} {
\pgfpathmoveto{\pgfqpoint{0pt}{\pgfplotmarksize}}
\pgfpathlineto{\pgfqpointpolar{44}{0.5\pgfplotmarksize}}
\pgfpathlineto{\pgfqpointpolar{18}{\pgfplotmarksize}}
\pgfpathlineto{\pgfqpointpolar{-20}{0.5\pgfplotmarksize}}
\pgfpathlineto{\pgfqpointpolar{-54}{\pgfplotmarksize}}
\pgfpathlineto{\pgfqpointpolar{-90}{0.5\pgfplotmarksize}}
\pgfpathlineto{\pgfqpointpolar{234}{\pgfplotmarksize}}
\pgfpathlineto{\pgfqpointpolar{198}{0.5\pgfplotmarksize}}
\pgfpathlineto{\pgfqpointpolar{162}{\pgfplotmarksize}}
\pgfpathlineto{\pgfqpointpolar{134}{0.5\pgfplotmarksize}}
\pgfpathclose
\pgfusepathqstroke
}
\pgfdeclareplotmark{newstar*} {
\pgfpathmoveto{\pgfqpoint{0pt}{\pgfplotmarksize}}
\pgfpathlineto{\pgfqpointpolar{44}{0.5\pgfplotmarksize}}
\pgfpathlineto{\pgfqpointpolar{18}{\pgfplotmarksize}}
\pgfpathlineto{\pgfqpointpolar{-20}{0.5\pgfplotmarksize}}
\pgfpathlineto{\pgfqpointpolar{-54}{\pgfplotmarksize}}
\pgfpathlineto{\pgfqpointpolar{-90}{0.5\pgfplotmarksize}}
\pgfpathlineto{\pgfqpointpolar{234}{\pgfplotmarksize}}
\pgfpathlineto{\pgfqpointpolar{198}{0.5\pgfplotmarksize}}
\pgfpathlineto{\pgfqpointpolar{162}{\pgfplotmarksize}}
\pgfpathlineto{\pgfqpointpolar{134}{0.5\pgfplotmarksize}}
\pgfpathclose
\pgfusepathqfillstroke
}
\definecolor{c}{rgb}{1,1,1};
\draw [color=c, fill=c] (0,0) rectangle (20,13.4957);
\draw [color=c, fill=c] (2,1.34957) rectangle (18,12.1461);
\definecolor{c}{rgb}{0,0,0};
\draw [c,line width=0.9] (2,1.34957) -- (2,12.1461) -- (18,12.1461) -- (18,1.34957) -- (2,1.34957);
\definecolor{c}{rgb}{1,1,1};
\draw [color=c, fill=c] (2,1.34957) rectangle (18,12.1461);
\definecolor{c}{rgb}{0,0,0};
\draw [c,line width=0.9] (2,1.34957) -- (2,12.1461) -- (18,12.1461) -- (18,1.34957) -- (2,1.34957);
\definecolor{c}{rgb}{0,0,0.6};
\draw [c,line width=0.9] (2,2.57523) -- (2.24242,2.57523) -- (2.24242,2.75923) -- (2.48485,2.75923) -- (2.48485,2.97755) -- (2.72727,2.97755) -- (2.72727,3.2364) -- (2.9697,3.2364) -- (2.9697,3.44847) -- (3.21212,3.44847) -- (3.21212,3.67926) --
 (3.45455,3.67926) -- (3.45455,4.11276) -- (3.69697,4.11276) -- (3.69697,4.43087) -- (3.93939,4.43087) -- (3.93939,4.66166) -- (4.18182,4.66166) -- (4.18182,4.68973) -- (4.42424,4.68973) -- (4.42424,5.58168) -- (4.66667,5.58168) -- (4.66667,7.02565)
 -- (4.90909,7.02565) -- (4.90909,6.22414) -- (5.15152,6.22414) -- (5.15152,6.77927) -- (5.39394,6.77927) -- (5.39394,6.76056) -- (5.63636,6.76056) -- (5.63636,7.04748) -- (5.87879,7.04748) -- (5.87879,7.29386) -- (6.12121,7.29386) --
 (6.12121,7.67123) -- (6.36364,7.67123) -- (6.36364,8.39477) -- (6.60606,8.39477) -- (6.60606,8.66298) -- (6.84848,8.66298) -- (6.84848,9.07777) -- (7.09091,9.07777) -- (7.09091,9.48321) -- (7.33333,9.48321) -- (7.33333,9.57365) -- (7.57576,9.57365)
 -- (7.57576,10.3253) -- (7.81818,10.3253) -- (7.81818,10.4313) -- (8.06061,10.4313) -- (8.06061,10.7432) -- (8.30303,10.7432) -- (8.30303,11.0332) -- (8.54545,11.0332) -- (8.54545,11.2702) -- (8.78788,11.2702) -- (8.78788,11.1611) --
 (9.0303,11.1611) -- (9.0303,11.6258) -- (9.27273,11.6258) -- (9.27273,11.6258) -- (9.51515,11.6258) -- (9.51515,11.3981) -- (9.75758,11.3981) -- (9.75758,11.2734) -- (10,11.2734) -- (10,11.632) -- (10.2424,11.632) -- (10.2424,11.3451) --
 (10.4848,11.3451) -- (10.4848,11.4605) -- (10.7273,11.4605) -- (10.7273,11.2609) -- (10.9697,11.2609) -- (10.9697,10.8929) -- (11.2121,10.8929) -- (11.2121,11.0738) -- (11.4545,11.0738) -- (11.4545,10.3565) -- (11.697,10.3565) -- (11.697,10.5155) --
 (11.9394,10.5155) -- (11.9394,9.82315) -- (12.1818,9.82315) -- (12.1818,9.39588) -- (12.4242,9.39588) -- (12.4242,9.60172) -- (12.6667,9.60172) -- (12.6667,8.83451) -- (12.9091,8.83451) -- (12.9091,8.55383) -- (13.1515,8.55383) -- (13.1515,7.97063)
 -- (13.3939,7.97063) -- (13.3939,7.68058) -- (13.6364,7.68058) -- (13.6364,7.09738) -- (13.8788,7.09738) -- (13.8788,6.79798) -- (14.1212,6.79798) -- (14.1212,6.72002) -- (14.3636,6.72002) -- (14.3636,5.84677) -- (14.6061,5.84677) --
 (14.6061,5.49748) -- (14.8485,5.49748) -- (14.8485,5.0858) -- (15.0909,5.0858) -- (15.0909,4.84878) -- (15.3333,4.84878) -- (15.3333,4.39033) -- (15.5758,4.39033) -- (15.5758,3.93811) -- (15.8182,3.93811) -- (15.8182,3.68861) -- (16.0606,3.68861) --
 (16.0606,3.28942) -- (16.303,3.28942) -- (16.303,3.25823) -- (16.5455,3.25823) -- (16.5455,3.11477) -- (16.7879,3.11477) -- (16.7879,2.7873) -- (17.0303,2.7873) -- (17.0303,2.17291) -- (17.2727,2.17291) -- (17.2727,2.04505) -- (17.5152,2.04505) --
 (17.5152,1.99515) -- (17.7576,1.99515) -- (17.7576,1.83921) -- (18,1.83921);
\definecolor{c}{rgb}{1,0,0};
\draw [c,line width=1.8] (2.08,2.36118) -- (2.24,2.50428) -- (2.4,2.65504) -- (2.56,2.81355) -- (2.72,2.97985) -- (2.88,3.15397) -- (3.04,3.33588) -- (3.2,3.52553) -- (3.36,3.72281) -- (3.52,3.92758) -- (3.68,4.13965) -- (3.84,4.35877) -- (4,4.58465)
 -- (4.16,4.81696) -- (4.32,5.05531) -- (4.48,5.29925) -- (4.64,5.54829) -- (4.8,5.80188) -- (4.96,6.05944) -- (5.12,6.32031) -- (5.28,6.58382) -- (5.44,6.84921) -- (5.6,7.11572) -- (5.76,7.38253) -- (5.92,7.6488) -- (6.08,7.91363) -- (6.24,8.17613)
 -- (6.4,8.43536) -- (6.56,8.69037) -- (6.72,8.94021) -- (6.88,9.18393) -- (7.04,9.42055) -- (7.2,9.64912) -- (7.36,9.86871) -- (7.52,10.0784) -- (7.68,10.2773) -- (7.84,10.4645) -- (8,10.6392) -- (8.16,10.8007) -- (8.32,10.9482) -- (8.48,11.081) --
 (8.64,11.1986) -- (8.8,11.3004) -- (8.96,11.3858) -- (9.12,11.4546) -- (9.28,11.5063) -- (9.44,11.5408) -- (9.6,11.5579) -- (9.76,11.5575) -- (9.92,11.5396);
\draw [c,line width=1.8] (9.92,11.5396) -- (10.08,11.5042) -- (10.24,11.4516) -- (10.4,11.3821) -- (10.56,11.2958) -- (10.72,11.1933) -- (10.88,11.075) -- (11.04,10.9414) -- (11.2,10.7932) -- (11.36,10.6311) -- (11.52,10.4557) -- (11.68,10.2679) --
 (11.84,10.0685) -- (12,9.85829) -- (12.16,9.63824) -- (12.32,9.40926) -- (12.48,9.17227) -- (12.64,8.92824) -- (12.8,8.67812) -- (12.96,8.42288) -- (13.12,8.16347) -- (13.28,7.90084) -- (13.44,7.63592) -- (13.6,7.36961) -- (13.76,7.10279) --
 (13.92,6.83631) -- (14.08,6.57099) -- (14.24,6.3076) -- (14.4,6.04687) -- (14.56,5.78949) -- (14.72,5.53611) -- (14.88,5.2873) -- (15.04,5.04362) -- (15.2,4.80556) -- (15.36,4.57355) -- (15.52,4.34799) -- (15.68,4.1292) -- (15.84,3.91749) --
 (16,3.71308) -- (16.16,3.51616) -- (16.32,3.32689) -- (16.48,3.14535) -- (16.64,2.97161) -- (16.8,2.80568) -- (16.96,2.64755) -- (17.12,2.49717) -- (17.28,2.35444) -- (17.44,2.21926) -- (17.6,2.09147) -- (17.76,1.97093);
\draw [c,line width=1.8] (17.76,1.97093) -- (17.92,1.85744);
\definecolor{c}{rgb}{1,1,1};
\draw [color=c, fill=c] (12.4,9.91934) rectangle (19.6,12.6185);
\definecolor{c}{rgb}{0,0,0};
\draw [c,line width=0.9] (12.4,9.91934) -- (19.6,9.91934);
\draw [c,line width=0.9] (19.6,9.91934) -- (19.6,12.6185);
\draw [c,line width=0.9] (19.6,12.6185) -- (12.4,12.6185);
\draw [c,line width=0.9] (12.4,12.6185) -- (12.4,9.91934);
\draw [anchor= west] (12.76,12.2811) node[scale=1.40004, color=c, rotate=0]{$\chi^{2} / ndf $};
\draw [anchor= east] (19.24,12.2811) node[scale=1.40004, color=c, rotate=0]{ 233.1 / 63};
\draw [anchor= west] (12.76,11.6063) node[scale=1.40004, color=c, rotate=0]{Constant };
\draw [anchor= east] (19.24,11.6063) node[scale=1.40004, color=c, rotate=0]{$  3543 \pm 12.4$};
\draw [anchor= west] (12.76,10.9315) node[scale=1.40004, color=c, rotate=0]{Mean     };
\draw [anchor= east] (19.24,10.9315) node[scale=1.40004, color=c, rotate=0]{$  1299 \pm 0.1$};
\draw [anchor= west] (12.76,10.2567) node[scale=1.40004, color=c, rotate=0]{Sigma    };
\draw [anchor= east] (19.24,10.2567) node[scale=1.40004, color=c, rotate=0]{$ 33.15 \pm 0.10$};
\draw [c,line width=0.9] (2,1.34957) -- (18,1.34957);
\draw [c,line width=0.9] (2.48485,1.67347) -- (2.48485,1.34957);
\draw [c,line width=0.9] (3.09091,1.51152) -- (3.09091,1.34957);
\draw [c,line width=0.9] (3.69697,1.51152) -- (3.69697,1.34957);
\draw [c,line width=0.9] (4.30303,1.51152) -- (4.30303,1.34957);
\draw [c,line width=0.9] (4.90909,1.67347) -- (4.90909,1.34957);
\draw [c,line width=0.9] (5.51515,1.51152) -- (5.51515,1.34957);
\draw [c,line width=0.9] (6.12121,1.51152) -- (6.12121,1.34957);
\draw [c,line width=0.9] (6.72727,1.51152) -- (6.72727,1.34957);
\draw [c,line width=0.9] (7.33333,1.67347) -- (7.33333,1.34957);
\draw [c,line width=0.9] (7.93939,1.51152) -- (7.93939,1.34957);
\draw [c,line width=0.9] (8.54545,1.51152) -- (8.54545,1.34957);
\draw [c,line width=0.9] (9.15152,1.51152) -- (9.15152,1.34957);
\draw [c,line width=0.9] (9.75758,1.67347) -- (9.75758,1.34957);
\draw [c,line width=0.9] (10.3636,1.51152) -- (10.3636,1.34957);
\draw [c,line width=0.9] (10.9697,1.51152) -- (10.9697,1.34957);
\draw [c,line width=0.9] (11.5758,1.51152) -- (11.5758,1.34957);
\draw [c,line width=0.9] (12.1818,1.67347) -- (12.1818,1.34957);
\draw [c,line width=0.9] (12.7879,1.51152) -- (12.7879,1.34957);
\draw [c,line width=0.9] (13.3939,1.51152) -- (13.3939,1.34957);
\draw [c,line width=0.9] (14,1.51152) -- (14,1.34957);
\draw [c,line width=0.9] (14.6061,1.67347) -- (14.6061,1.34957);
\draw [c,line width=0.9] (15.2121,1.51152) -- (15.2121,1.34957);
\draw [c,line width=0.9] (15.8182,1.51152) -- (15.8182,1.34957);
\draw [c,line width=0.9] (16.4242,1.51152) -- (16.4242,1.34957);
\draw [c,line width=0.9] (17.0303,1.67347) -- (17.0303,1.34957);
\draw [c,line width=0.9] (2.48485,1.67347) -- (2.48485,1.34957);
\draw [c,line width=0.9] (17.0303,1.67347) -- (17.0303,1.34957);
\draw [c,line width=0.9] (17.6364,1.51152) -- (17.6364,1.34957);
\draw [anchor=base] (2.48485,0.904212) node[scale=1.01821, color=c, rotate=0]{1240};
\draw [anchor=base] (4.90909,0.904212) node[scale=1.01821, color=c, rotate=0]{1260};
\draw [anchor=base] (7.33333,0.904212) node[scale=1.01821, color=c, rotate=0]{1280};
\draw [anchor=base] (9.75758,0.904212) node[scale=1.01821, color=c, rotate=0]{1300};
\draw [anchor=base] (12.1818,0.904212) node[scale=1.01821, color=c, rotate=0]{1320};
\draw [anchor=base] (14.6061,0.904212) node[scale=1.01821, color=c, rotate=0]{1340};
\draw [anchor=base] (17.0303,0.904212) node[scale=1.01821, color=c, rotate=0]{1360};
\draw [c,line width=0.9] (2,1.34957) -- (2,12.1461);
\draw [c,line width=0.9] (2.48,2.07) -- (2,2.07);
\draw [c,line width=0.9] (2.24,2.38187) -- (2,2.38187);
\draw [c,line width=0.9] (2.24,2.69374) -- (2,2.69374);
\draw [c,line width=0.9] (2.24,3.00561) -- (2,3.00561);
\draw [c,line width=0.9] (2.24,3.31749) -- (2,3.31749);
\draw [c,line width=0.9] (2.48,3.62936) -- (2,3.62936);
\draw [c,line width=0.9] (2.24,3.94123) -- (2,3.94123);
\draw [c,line width=0.9] (2.24,4.2531) -- (2,4.2531);
\draw [c,line width=0.9] (2.24,4.56498) -- (2,4.56498);
\draw [c,line width=0.9] (2.24,4.87685) -- (2,4.87685);
\draw [c,line width=0.9] (2.48,5.18872) -- (2,5.18872);
\draw [c,line width=0.9] (2.24,5.50059) -- (2,5.50059);
\draw [c,line width=0.9] (2.24,5.81247) -- (2,5.81247);
\draw [c,line width=0.9] (2.24,6.12434) -- (2,6.12434);
\draw [c,line width=0.9] (2.24,6.43621) -- (2,6.43621);
\draw [c,line width=0.9] (2.48,6.74809) -- (2,6.74809);
\draw [c,line width=0.9] (2.24,7.05996) -- (2,7.05996);
\draw [c,line width=0.9] (2.24,7.37183) -- (2,7.37183);
\draw [c,line width=0.9] (2.24,7.6837) -- (2,7.6837);
\draw [c,line width=0.9] (2.24,7.99557) -- (2,7.99557);
\draw [c,line width=0.9] (2.48,8.30745) -- (2,8.30745);
\draw [c,line width=0.9] (2.24,8.61932) -- (2,8.61932);
\draw [c,line width=0.9] (2.24,8.93119) -- (2,8.93119);
\draw [c,line width=0.9] (2.24,9.24307) -- (2,9.24307);
\draw [c,line width=0.9] (2.24,9.55494) -- (2,9.55494);
\draw [c,line width=0.9] (2.48,9.86681) -- (2,9.86681);
\draw [c,line width=0.9] (2.24,10.1787) -- (2,10.1787);
\draw [c,line width=0.9] (2.24,10.4906) -- (2,10.4906);
\draw [c,line width=0.9] (2.24,10.8024) -- (2,10.8024);
\draw [c,line width=0.9] (2.24,11.1143) -- (2,11.1143);
\draw [c,line width=0.9] (2.48,11.4262) -- (2,11.4262);
\draw [c,line width=0.9] (2.48,2.07) -- (2,2.07);
\draw [c,line width=0.9] (2.24,1.75812) -- (2,1.75812);
\draw [c,line width=0.9] (2.24,1.44625) -- (2,1.44625);
\draw [c,line width=0.9] (2.48,11.4262) -- (2,11.4262);
\draw [c,line width=0.9] (2.24,11.738) -- (2,11.738);
\draw [c,line width=0.9] (2.24,12.0499) -- (2,12.0499);
\draw [anchor= east] (1.9,2.07) node[scale=1.01821, color=c, rotate=0]{500};
\draw [anchor= east] (1.9,3.62936) node[scale=1.01821, color=c, rotate=0]{1000};
\draw [anchor= east] (1.9,5.18872) node[scale=1.01821, color=c, rotate=0]{1500};
\draw [anchor= east] (1.9,6.74809) node[scale=1.01821, color=c, rotate=0]{2000};
\draw [anchor= east] (1.9,8.30745) node[scale=1.01821, color=c, rotate=0]{2500};
\draw [anchor= east] (1.9,9.86681) node[scale=1.01821, color=c, rotate=0]{3000};
\draw [anchor= east] (1.9,11.4262) node[scale=1.01821, color=c, rotate=0]{3500};
\definecolor{c}{rgb}{1,1,1};
\draw [color=c, fill=c] (12.4,9.91934) rectangle (19.6,12.6185);
\definecolor{c}{rgb}{0,0,0};
\draw [c,line width=0.9] (12.4,9.91934) -- (19.6,9.91934);
\draw [c,line width=0.9] (19.6,9.91934) -- (19.6,12.6185);
\draw [c,line width=0.9] (19.6,12.6185) -- (12.4,12.6185);
\draw [c,line width=0.9] (12.4,12.6185) -- (12.4,9.91934);
\draw [anchor= west] (12.76,12.2811) node[scale=1.40004, color=c, rotate=0]{$\chi^{2} / ndf $};
\draw [anchor= east] (19.24,12.2811) node[scale=1.40004, color=c, rotate=0]{ 233.1 / 63};
\draw [anchor= west] (12.76,11.6063) node[scale=1.40004, color=c, rotate=0]{Constant };
\draw [anchor= east] (19.24,11.6063) node[scale=1.40004, color=c, rotate=0]{$  3543 \pm 12.4$};
\draw [anchor= west] (12.76,10.9315) node[scale=1.40004, color=c, rotate=0]{Mean     };
\draw [anchor= east] (19.24,10.9315) node[scale=1.40004, color=c, rotate=0]{$  1299 \pm 0.1$};
\draw [anchor= west] (12.76,10.2567) node[scale=1.40004, color=c, rotate=0]{Sigma    };
\draw [anchor= east] (19.24,10.2567) node[scale=1.40004, color=c, rotate=0]{$ 33.15 \pm 0.10$};
\end{tikzpicture}
 
\begin{tikzpicture}
\pgfdeclareplotmark{cross} {
\pgfpathmoveto{\pgfpoint{-0.3\pgfplotmarksize}{\pgfplotmarksize}}
\pgfpathlineto{\pgfpoint{+0.3\pgfplotmarksize}{\pgfplotmarksize}}
\pgfpathlineto{\pgfpoint{+0.3\pgfplotmarksize}{0.3\pgfplotmarksize}}
\pgfpathlineto{\pgfpoint{+1\pgfplotmarksize}{0.3\pgfplotmarksize}}
\pgfpathlineto{\pgfpoint{+1\pgfplotmarksize}{-0.3\pgfplotmarksize}}
\pgfpathlineto{\pgfpoint{+0.3\pgfplotmarksize}{-0.3\pgfplotmarksize}}
\pgfpathlineto{\pgfpoint{+0.3\pgfplotmarksize}{-1.\pgfplotmarksize}}
\pgfpathlineto{\pgfpoint{-0.3\pgfplotmarksize}{-1.\pgfplotmarksize}}
\pgfpathlineto{\pgfpoint{-0.3\pgfplotmarksize}{-0.3\pgfplotmarksize}}
\pgfpathlineto{\pgfpoint{-1.\pgfplotmarksize}{-0.3\pgfplotmarksize}}
\pgfpathlineto{\pgfpoint{-1.\pgfplotmarksize}{0.3\pgfplotmarksize}}
\pgfpathlineto{\pgfpoint{-0.3\pgfplotmarksize}{0.3\pgfplotmarksize}}
\pgfpathclose
\pgfusepathqstroke
}
\pgfdeclareplotmark{cross*} {
\pgfpathmoveto{\pgfpoint{-0.3\pgfplotmarksize}{\pgfplotmarksize}}
\pgfpathlineto{\pgfpoint{+0.3\pgfplotmarksize}{\pgfplotmarksize}}
\pgfpathlineto{\pgfpoint{+0.3\pgfplotmarksize}{0.3\pgfplotmarksize}}
\pgfpathlineto{\pgfpoint{+1\pgfplotmarksize}{0.3\pgfplotmarksize}}
\pgfpathlineto{\pgfpoint{+1\pgfplotmarksize}{-0.3\pgfplotmarksize}}
\pgfpathlineto{\pgfpoint{+0.3\pgfplotmarksize}{-0.3\pgfplotmarksize}}
\pgfpathlineto{\pgfpoint{+0.3\pgfplotmarksize}{-1.\pgfplotmarksize}}
\pgfpathlineto{\pgfpoint{-0.3\pgfplotmarksize}{-1.\pgfplotmarksize}}
\pgfpathlineto{\pgfpoint{-0.3\pgfplotmarksize}{-0.3\pgfplotmarksize}}
\pgfpathlineto{\pgfpoint{-1.\pgfplotmarksize}{-0.3\pgfplotmarksize}}
\pgfpathlineto{\pgfpoint{-1.\pgfplotmarksize}{0.3\pgfplotmarksize}}
\pgfpathlineto{\pgfpoint{-0.3\pgfplotmarksize}{0.3\pgfplotmarksize}}
\pgfpathclose
\pgfusepathqfillstroke
}
\pgfdeclareplotmark{newstar} {
\pgfpathmoveto{\pgfqpoint{0pt}{\pgfplotmarksize}}
\pgfpathlineto{\pgfqpointpolar{44}{0.5\pgfplotmarksize}}
\pgfpathlineto{\pgfqpointpolar{18}{\pgfplotmarksize}}
\pgfpathlineto{\pgfqpointpolar{-20}{0.5\pgfplotmarksize}}
\pgfpathlineto{\pgfqpointpolar{-54}{\pgfplotmarksize}}
\pgfpathlineto{\pgfqpointpolar{-90}{0.5\pgfplotmarksize}}
\pgfpathlineto{\pgfqpointpolar{234}{\pgfplotmarksize}}
\pgfpathlineto{\pgfqpointpolar{198}{0.5\pgfplotmarksize}}
\pgfpathlineto{\pgfqpointpolar{162}{\pgfplotmarksize}}
\pgfpathlineto{\pgfqpointpolar{134}{0.5\pgfplotmarksize}}
\pgfpathclose
\pgfusepathqstroke
}
\pgfdeclareplotmark{newstar*} {
\pgfpathmoveto{\pgfqpoint{0pt}{\pgfplotmarksize}}
\pgfpathlineto{\pgfqpointpolar{44}{0.5\pgfplotmarksize}}
\pgfpathlineto{\pgfqpointpolar{18}{\pgfplotmarksize}}
\pgfpathlineto{\pgfqpointpolar{-20}{0.5\pgfplotmarksize}}
\pgfpathlineto{\pgfqpointpolar{-54}{\pgfplotmarksize}}
\pgfpathlineto{\pgfqpointpolar{-90}{0.5\pgfplotmarksize}}
\pgfpathlineto{\pgfqpointpolar{234}{\pgfplotmarksize}}
\pgfpathlineto{\pgfqpointpolar{198}{0.5\pgfplotmarksize}}
\pgfpathlineto{\pgfqpointpolar{162}{\pgfplotmarksize}}
\pgfpathlineto{\pgfqpointpolar{134}{0.5\pgfplotmarksize}}
\pgfpathclose
\pgfusepathqfillstroke
}
\definecolor{c}{rgb}{1,1,1};
\draw [color=c, fill=c] (0,0) rectangle (20,13.4957);
\draw [color=c, fill=c] (2,1.34957) rectangle (18,12.1461);
\definecolor{c}{rgb}{0,0,0};
\draw [c,line width=0.9] (2,1.34957) -- (2,12.1461) -- (18,12.1461) -- (18,1.34957) -- (2,1.34957);
\definecolor{c}{rgb}{1,1,1};
\draw [color=c, fill=c] (2,1.34957) rectangle (18,12.1461);
\definecolor{c}{rgb}{0,0,0};
\draw [c,line width=0.9] (2,1.34957) -- (2,12.1461) -- (18,12.1461) -- (18,1.34957) -- (2,1.34957);
\definecolor{c}{rgb}{0,0,0.6};
\draw [c,line width=0.9] (2,2.03357) -- (2.45714,2.03357) -- (2.45714,2.42939) -- (2.91429,2.42939) -- (2.91429,2.86815) -- (3.37143,2.86815) -- (3.37143,3.40188) -- (3.82857,3.40188) -- (3.82857,3.93528) -- (4.28571,3.93528) -- (4.28571,4.59442) --
 (4.74286,4.59442) -- (4.74286,5.19508) -- (5.2,5.19508) -- (5.2,5.98132) -- (5.65714,5.98132) -- (5.65714,6.67392) -- (6.11429,6.67392) -- (6.11429,7.45407) -- (6.57143,7.45407) -- (6.57143,8.26566) -- (7.02857,8.26566) -- (7.02857,8.8697) --
 (7.48571,8.8697) -- (7.48571,9.77661) -- (7.94286,9.77661) -- (7.94286,10.308) -- (8.4,10.308) -- (8.4,10.8417) -- (8.85714,10.8417) -- (8.85714,11.1777) -- (9.31429,11.1777) -- (9.31429,11.3961) -- (9.77143,11.3961) -- (9.77143,11.4606) --
 (10.2286,11.4606) -- (10.2286,11.632) -- (10.6857,11.632) -- (10.6857,10.9806) -- (11.1429,10.9806) -- (11.1429,11.2017) -- (11.6,11.2017) -- (11.6,10.6196) -- (12.0571,10.6196) -- (12.0571,10.0933) -- (12.5143,10.0933) -- (12.5143,9.33347) --
 (12.9714,9.33347) -- (12.9714,8.675) -- (13.4286,8.675) -- (13.4286,7.93271) -- (13.8857,7.93271) -- (13.8857,7.07887) -- (14.3429,7.07887) -- (14.3429,6.17162) -- (14.8,6.17162) -- (14.8,5.56555) -- (15.2571,5.56555) -- (15.2571,4.66405) --
 (15.7143,4.66405) -- (15.7143,3.9613) -- (16.1714,3.9613) -- (16.1714,3.34475) -- (16.6286,3.34475) -- (16.6286,2.75153) -- (17.0857,2.75153) -- (17.0857,2.2205) -- (17.5429,2.2205) -- (17.5429,1.83921) -- (18,1.83921);
\definecolor{c}{rgb}{1,0,0};
\draw [c,line width=1.8] (2.08,1.67835) -- (2.24,1.8182) -- (2.4,1.96557) -- (2.56,2.12055) -- (2.72,2.28323) -- (2.88,2.45366) -- (3.04,2.63184) -- (3.2,2.81776) -- (3.36,3.01137) -- (3.52,3.21255) -- (3.68,3.42119) -- (3.84,3.63709) -- (4,3.86003)
 -- (4.16,4.08974) -- (4.32,4.32591) -- (4.48,4.56815) -- (4.64,4.81607) -- (4.8,5.06919) -- (4.96,5.32701) -- (5.12,5.58897) -- (5.28,5.85446) -- (5.44,6.12285) -- (5.6,6.39343) -- (5.76,6.66547) -- (5.92,6.93822) -- (6.08,7.21086) -- (6.24,7.48255)
 -- (6.4,7.75245) -- (6.56,8.01966) -- (6.72,8.28328) -- (6.88,8.54239) -- (7.04,8.79608) -- (7.2,9.04341) -- (7.36,9.28345) -- (7.52,9.5153) -- (7.68,9.73805) -- (7.84,9.95081) -- (8,10.1527) -- (8.16,10.343) -- (8.32,10.5208) -- (8.48,10.6854) --
 (8.64,10.8361) -- (8.8,10.9722) -- (8.96,11.0932) -- (9.12,11.1985) -- (9.28,11.2877) -- (9.44,11.3604) -- (9.6,11.4162) -- (9.76,11.455) -- (9.92,11.4764);
\draw [c,line width=1.8] (9.92,11.4764) -- (10.08,11.4805) -- (10.24,11.4672) -- (10.4,11.4366) -- (10.56,11.3888) -- (10.72,11.3241) -- (10.88,11.2426) -- (11.04,11.1448) -- (11.2,11.0311) -- (11.36,10.9021) -- (11.52,10.7582) -- (11.68,10.6) --
 (11.84,10.4284) -- (12,10.2439) -- (12.16,10.0474) -- (12.32,9.83961) -- (12.48,9.62146) -- (12.64,9.3938) -- (12.8,9.1575) -- (12.96,8.91349) -- (13.12,8.66268) -- (13.28,8.40601) -- (13.44,8.1444) -- (13.6,7.87877) -- (13.76,7.61003) --
 (13.92,7.33907) -- (14.08,7.06677) -- (14.24,6.79397) -- (14.4,6.5215) -- (14.56,6.25014) -- (14.72,5.98063) -- (14.88,5.71369) -- (15.04,5.44998) -- (15.2,5.19014) -- (15.36,4.93474) -- (15.52,4.68431) -- (15.68,4.43934) -- (15.84,4.20026) --
 (16,3.96747) -- (16.16,3.7413) -- (16.32,3.52205) -- (16.48,3.30996) -- (16.64,3.10524) -- (16.8,2.90805) -- (16.96,2.7185) -- (17.12,2.53666) -- (17.28,2.36258) -- (17.44,2.19626) -- (17.6,2.03765) -- (17.76,1.88671);
\draw [c,line width=1.8] (17.76,1.88671) -- (17.92,1.74333);
\definecolor{c}{rgb}{1,1,1};
\draw [color=c, fill=c] (12.4,9.91934) rectangle (19.6,12.6185);
\definecolor{c}{rgb}{0,0,0};
\draw [c,line width=0.9] (12.4,9.91934) -- (19.6,9.91934);
\draw [c,line width=0.9] (19.6,9.91934) -- (19.6,12.6185);
\draw [c,line width=0.9] (19.6,12.6185) -- (12.4,12.6185);
\draw [c,line width=0.9] (12.4,12.6185) -- (12.4,9.91934);
\draw [anchor= west] (12.76,12.2811) node[scale=1.01821, color=c, rotate=0]{$\chi^{2} / ndf $};
\draw [anchor= east] (19.24,12.2811) node[scale=1.01821, color=c, rotate=0]{ 347.5 / 32};
\draw [anchor= west] (12.76,11.6063) node[scale=1.01821, color=c, rotate=0]{Constant };
\draw [anchor= east] (19.24,11.6063) node[scale=1.01821, color=c, rotate=0]{$ 3.442e+04 \pm 5.320e+01$};
\draw [anchor= west] (12.76,10.9315) node[scale=1.01821, color=c, rotate=0]{Mean     };
\draw [anchor= east] (19.24,10.9315) node[scale=1.01821, color=c, rotate=0]{$ 537.2 \pm 0.0$};
\draw [anchor= west] (12.76,10.2567) node[scale=1.01821, color=c, rotate=0]{Sigma    };
\draw [anchor= east] (19.24,10.2567) node[scale=1.01821, color=c, rotate=0]{$  18.1 \pm 0.0$};
\draw [c,line width=0.9] (2,1.34957) -- (18,1.34957);
\draw [c,line width=0.9] (3.82857,1.67347) -- (3.82857,1.34957);
\draw [c,line width=0.9] (4.28571,1.51152) -- (4.28571,1.34957);
\draw [c,line width=0.9] (4.74286,1.51152) -- (4.74286,1.34957);
\draw [c,line width=0.9] (5.2,1.51152) -- (5.2,1.34957);
\draw [c,line width=0.9] (5.65714,1.51152) -- (5.65714,1.34957);
\draw [c,line width=0.9] (6.11429,1.67347) -- (6.11429,1.34957);
\draw [c,line width=0.9] (6.57143,1.51152) -- (6.57143,1.34957);
\draw [c,line width=0.9] (7.02857,1.51152) -- (7.02857,1.34957);
\draw [c,line width=0.9] (7.48571,1.51152) -- (7.48571,1.34957);
\draw [c,line width=0.9] (7.94286,1.51152) -- (7.94286,1.34957);
\draw [c,line width=0.9] (8.4,1.67347) -- (8.4,1.34957);
\draw [c,line width=0.9] (8.85714,1.51152) -- (8.85714,1.34957);
\draw [c,line width=0.9] (9.31429,1.51152) -- (9.31429,1.34957);
\draw [c,line width=0.9] (9.77143,1.51152) -- (9.77143,1.34957);
\draw [c,line width=0.9] (10.2286,1.51152) -- (10.2286,1.34957);
\draw [c,line width=0.9] (10.6857,1.67347) -- (10.6857,1.34957);
\draw [c,line width=0.9] (11.1429,1.51152) -- (11.1429,1.34957);
\draw [c,line width=0.9] (11.6,1.51152) -- (11.6,1.34957);
\draw [c,line width=0.9] (12.0571,1.51152) -- (12.0571,1.34957);
\draw [c,line width=0.9] (12.5143,1.51152) -- (12.5143,1.34957);
\draw [c,line width=0.9] (12.9714,1.67347) -- (12.9714,1.34957);
\draw [c,line width=0.9] (13.4286,1.51152) -- (13.4286,1.34957);
\draw [c,line width=0.9] (13.8857,1.51152) -- (13.8857,1.34957);
\draw [c,line width=0.9] (14.3429,1.51152) -- (14.3429,1.34957);
\draw [c,line width=0.9] (14.8,1.51152) -- (14.8,1.34957);
\draw [c,line width=0.9] (15.2571,1.67347) -- (15.2571,1.34957);
\draw [c,line width=0.9] (15.7143,1.51152) -- (15.7143,1.34957);
\draw [c,line width=0.9] (16.1714,1.51152) -- (16.1714,1.34957);
\draw [c,line width=0.9] (16.6286,1.51152) -- (16.6286,1.34957);
\draw [c,line width=0.9] (17.0857,1.51152) -- (17.0857,1.34957);
\draw [c,line width=0.9] (17.5429,1.67347) -- (17.5429,1.34957);
\draw [c,line width=0.9] (3.82857,1.67347) -- (3.82857,1.34957);
\draw [c,line width=0.9] (3.37143,1.51152) -- (3.37143,1.34957);
\draw [c,line width=0.9] (2.91429,1.51152) -- (2.91429,1.34957);
\draw [c,line width=0.9] (2.45714,1.51152) -- (2.45714,1.34957);
\draw [c,line width=0.9] (2,1.51152) -- (2,1.34957);
\draw [c,line width=0.9] (17.5429,1.67347) -- (17.5429,1.34957);
\draw [c,line width=0.9] (18,1.51152) -- (18,1.34957);
\draw [anchor=base] (3.82857,0.904212) node[scale=1.01821, color=c, rotate=0]{510};
\draw [anchor=base] (6.11429,0.904212) node[scale=1.01821, color=c, rotate=0]{520};
\draw [anchor=base] (8.4,0.904212) node[scale=1.01821, color=c, rotate=0]{530};
\draw [anchor=base] (10.6857,0.904212) node[scale=1.01821, color=c, rotate=0]{540};
\draw [anchor=base] (12.9714,0.904212) node[scale=1.01821, color=c, rotate=0]{550};
\draw [anchor=base] (15.2571,0.904212) node[scale=1.01821, color=c, rotate=0]{560};
\draw [anchor=base] (17.5429,0.904212) node[scale=1.01821, color=c, rotate=0]{570};
\draw [c,line width=0.9] (2,1.34957) -- (2,12.1461);
\draw [c,line width=0.9] (2.48,1.5377) -- (2,1.5377);
\draw [c,line width=0.9] (2.24,1.87572) -- (2,1.87572);
\draw [c,line width=0.9] (2.24,2.21374) -- (2,2.21374);
\draw [c,line width=0.9] (2.24,2.55176) -- (2,2.55176);
\draw [c,line width=0.9] (2.24,2.88978) -- (2,2.88978);
\draw [c,line width=0.9] (2.48,3.2278) -- (2,3.2278);
\draw [c,line width=0.9] (2.24,3.56582) -- (2,3.56582);
\draw [c,line width=0.9] (2.24,3.90384) -- (2,3.90384);
\draw [c,line width=0.9] (2.24,4.24186) -- (2,4.24186);
\draw [c,line width=0.9] (2.24,4.57988) -- (2,4.57988);
\draw [c,line width=0.9] (2.48,4.9179) -- (2,4.9179);
\draw [c,line width=0.9] (2.24,5.25592) -- (2,5.25592);
\draw [c,line width=0.9] (2.24,5.59395) -- (2,5.59395);
\draw [c,line width=0.9] (2.24,5.93197) -- (2,5.93197);
\draw [c,line width=0.9] (2.24,6.26999) -- (2,6.26999);
\draw [c,line width=0.9] (2.48,6.60801) -- (2,6.60801);
\draw [c,line width=0.9] (2.24,6.94603) -- (2,6.94603);
\draw [c,line width=0.9] (2.24,7.28405) -- (2,7.28405);
\draw [c,line width=0.9] (2.24,7.62207) -- (2,7.62207);
\draw [c,line width=0.9] (2.24,7.96009) -- (2,7.96009);
\draw [c,line width=0.9] (2.48,8.29811) -- (2,8.29811);
\draw [c,line width=0.9] (2.24,8.63613) -- (2,8.63613);
\draw [c,line width=0.9] (2.24,8.97415) -- (2,8.97415);
\draw [c,line width=0.9] (2.24,9.31217) -- (2,9.31217);
\draw [c,line width=0.9] (2.24,9.65019) -- (2,9.65019);
\draw [c,line width=0.9] (2.48,9.98821) -- (2,9.98821);
\draw [c,line width=0.9] (2.24,10.3262) -- (2,10.3262);
\draw [c,line width=0.9] (2.24,10.6643) -- (2,10.6643);
\draw [c,line width=0.9] (2.24,11.0023) -- (2,11.0023);
\draw [c,line width=0.9] (2.24,11.3403) -- (2,11.3403);
\draw [c,line width=0.9] (2.48,11.6783) -- (2,11.6783);
\draw [c,line width=0.9] (2.48,1.5377) -- (2,1.5377);
\draw [c,line width=0.9] (2.48,11.6783) -- (2,11.6783);
\draw [c,line width=0.9] (2.24,12.0163) -- (2,12.0163);
\draw [anchor= east] (1.9,1.5377) node[scale=1.01821, color=c, rotate=0]{5000};
\draw [anchor= east] (1.9,3.2278) node[scale=1.01821, color=c, rotate=0]{10000};
\draw [anchor= east] (1.9,4.9179) node[scale=1.01821, color=c, rotate=0]{15000};
\draw [anchor= east] (1.9,6.60801) node[scale=1.01821, color=c, rotate=0]{20000};
\draw [anchor= east] (1.9,8.29811) node[scale=1.01821, color=c, rotate=0]{25000};
\draw [anchor= east] (1.9,9.98821) node[scale=1.01821, color=c, rotate=0]{30000};
\draw [anchor= east] (1.9,11.6783) node[scale=1.01821, color=c, rotate=0]{35000};
\definecolor{c}{rgb}{1,1,1};
\draw [color=c, fill=c] (12.4,9.91934) rectangle (19.6,12.6185);
\definecolor{c}{rgb}{0,0,0};
\draw [c,line width=0.9] (12.4,9.91934) -- (19.6,9.91934);
\draw [c,line width=0.9] (19.6,9.91934) -- (19.6,12.6185);
\draw [c,line width=0.9] (19.6,12.6185) -- (12.4,12.6185);
\draw [c,line width=0.9] (12.4,12.6185) -- (12.4,9.91934);
\draw [anchor= west] (12.76,12.2811) node[scale=1.01821, color=c, rotate=0]{$\chi^{2} / ndf $};
\draw [anchor= east] (19.24,12.2811) node[scale=1.01821, color=c, rotate=0]{ 347.5 / 32};
\draw [anchor= west] (12.76,11.6063) node[scale=1.01821, color=c, rotate=0]{Constant };
\draw [anchor= east] (19.24,11.6063) node[scale=1.01821, color=c, rotate=0]{$ 3.442e+04 \pm 5.320e+01$};
\draw [anchor= west] (12.76,10.9315) node[scale=1.01821, color=c, rotate=0]{Mean     };
\draw [anchor= east] (19.24,10.9315) node[scale=1.01821, color=c, rotate=0]{$ 537.2 \pm 0.0$};
\draw [anchor= west] (12.76,10.2567) node[scale=1.01821, color=c, rotate=0]{Sigma    };
\draw [anchor= east] (19.24,10.2567) node[scale=1.01821, color=c, rotate=0]{$  18.1 \pm 0.0$};
\end{tikzpicture}
 
\begin{tikzpicture}
\pgfdeclareplotmark{cross} {
\pgfpathmoveto{\pgfpoint{-0.3\pgfplotmarksize}{\pgfplotmarksize}}
\pgfpathlineto{\pgfpoint{+0.3\pgfplotmarksize}{\pgfplotmarksize}}
\pgfpathlineto{\pgfpoint{+0.3\pgfplotmarksize}{0.3\pgfplotmarksize}}
\pgfpathlineto{\pgfpoint{+1\pgfplotmarksize}{0.3\pgfplotmarksize}}
\pgfpathlineto{\pgfpoint{+1\pgfplotmarksize}{-0.3\pgfplotmarksize}}
\pgfpathlineto{\pgfpoint{+0.3\pgfplotmarksize}{-0.3\pgfplotmarksize}}
\pgfpathlineto{\pgfpoint{+0.3\pgfplotmarksize}{-1.\pgfplotmarksize}}
\pgfpathlineto{\pgfpoint{-0.3\pgfplotmarksize}{-1.\pgfplotmarksize}}
\pgfpathlineto{\pgfpoint{-0.3\pgfplotmarksize}{-0.3\pgfplotmarksize}}
\pgfpathlineto{\pgfpoint{-1.\pgfplotmarksize}{-0.3\pgfplotmarksize}}
\pgfpathlineto{\pgfpoint{-1.\pgfplotmarksize}{0.3\pgfplotmarksize}}
\pgfpathlineto{\pgfpoint{-0.3\pgfplotmarksize}{0.3\pgfplotmarksize}}
\pgfpathclose
\pgfusepathqstroke
}
\pgfdeclareplotmark{cross*} {
\pgfpathmoveto{\pgfpoint{-0.3\pgfplotmarksize}{\pgfplotmarksize}}
\pgfpathlineto{\pgfpoint{+0.3\pgfplotmarksize}{\pgfplotmarksize}}
\pgfpathlineto{\pgfpoint{+0.3\pgfplotmarksize}{0.3\pgfplotmarksize}}
\pgfpathlineto{\pgfpoint{+1\pgfplotmarksize}{0.3\pgfplotmarksize}}
\pgfpathlineto{\pgfpoint{+1\pgfplotmarksize}{-0.3\pgfplotmarksize}}
\pgfpathlineto{\pgfpoint{+0.3\pgfplotmarksize}{-0.3\pgfplotmarksize}}
\pgfpathlineto{\pgfpoint{+0.3\pgfplotmarksize}{-1.\pgfplotmarksize}}
\pgfpathlineto{\pgfpoint{-0.3\pgfplotmarksize}{-1.\pgfplotmarksize}}
\pgfpathlineto{\pgfpoint{-0.3\pgfplotmarksize}{-0.3\pgfplotmarksize}}
\pgfpathlineto{\pgfpoint{-1.\pgfplotmarksize}{-0.3\pgfplotmarksize}}
\pgfpathlineto{\pgfpoint{-1.\pgfplotmarksize}{0.3\pgfplotmarksize}}
\pgfpathlineto{\pgfpoint{-0.3\pgfplotmarksize}{0.3\pgfplotmarksize}}
\pgfpathclose
\pgfusepathqfillstroke
}
\pgfdeclareplotmark{newstar} {
\pgfpathmoveto{\pgfqpoint{0pt}{\pgfplotmarksize}}
\pgfpathlineto{\pgfqpointpolar{44}{0.5\pgfplotmarksize}}
\pgfpathlineto{\pgfqpointpolar{18}{\pgfplotmarksize}}
\pgfpathlineto{\pgfqpointpolar{-20}{0.5\pgfplotmarksize}}
\pgfpathlineto{\pgfqpointpolar{-54}{\pgfplotmarksize}}
\pgfpathlineto{\pgfqpointpolar{-90}{0.5\pgfplotmarksize}}
\pgfpathlineto{\pgfqpointpolar{234}{\pgfplotmarksize}}
\pgfpathlineto{\pgfqpointpolar{198}{0.5\pgfplotmarksize}}
\pgfpathlineto{\pgfqpointpolar{162}{\pgfplotmarksize}}
\pgfpathlineto{\pgfqpointpolar{134}{0.5\pgfplotmarksize}}
\pgfpathclose
\pgfusepathqstroke
}
\pgfdeclareplotmark{newstar*} {
\pgfpathmoveto{\pgfqpoint{0pt}{\pgfplotmarksize}}
\pgfpathlineto{\pgfqpointpolar{44}{0.5\pgfplotmarksize}}
\pgfpathlineto{\pgfqpointpolar{18}{\pgfplotmarksize}}
\pgfpathlineto{\pgfqpointpolar{-20}{0.5\pgfplotmarksize}}
\pgfpathlineto{\pgfqpointpolar{-54}{\pgfplotmarksize}}
\pgfpathlineto{\pgfqpointpolar{-90}{0.5\pgfplotmarksize}}
\pgfpathlineto{\pgfqpointpolar{234}{\pgfplotmarksize}}
\pgfpathlineto{\pgfqpointpolar{198}{0.5\pgfplotmarksize}}
\pgfpathlineto{\pgfqpointpolar{162}{\pgfplotmarksize}}
\pgfpathlineto{\pgfqpointpolar{134}{0.5\pgfplotmarksize}}
\pgfpathclose
\pgfusepathqfillstroke
}
\definecolor{c}{rgb}{1,1,1};
\draw [color=c, fill=c] (0,0) rectangle (20,13.4957);
\draw [color=c, fill=c] (2,1.34957) rectangle (18,12.1461);
\definecolor{c}{rgb}{0,0,0};
\draw [c,line width=0.9] (2,1.34957) -- (2,12.1461) -- (18,12.1461) -- (18,1.34957) -- (2,1.34957);
\definecolor{c}{rgb}{1,1,1};
\draw [color=c, fill=c] (2,1.34957) rectangle (18,12.1461);
\definecolor{c}{rgb}{0,0,0};
\draw [c,line width=0.9] (2,1.34957) -- (2,12.1461) -- (18,12.1461) -- (18,1.34957) -- (2,1.34957);
\definecolor{c}{rgb}{0,0,0.6};
\draw [c,line width=0.9] (2,2.06599) -- (2.32653,2.06599) -- (2.32653,2.30652) -- (2.65306,2.30652) -- (2.65306,2.66616) -- (2.97959,2.66616) -- (2.97959,3.23425) -- (3.30612,3.23425) -- (3.30612,3.60764) -- (3.63265,3.60764) -- (3.63265,4.15283) --
 (3.95918,4.15283) -- (3.95918,4.51476) -- (4.28571,4.51476) -- (4.28571,5.11264) -- (4.61225,5.11264) -- (4.61225,5.55017) -- (4.93878,5.55017) -- (4.93878,4.78278) -- (5.26531,4.78278) -- (5.26531,6.59244) -- (5.59184,6.59244) -- (5.59184,7.15596)
 -- (5.91837,7.15596) -- (5.91837,7.44688) -- (6.2449,7.44688) -- (6.2449,8.16387) -- (6.57143,8.16387) -- (6.57143,8.56245) -- (6.89796,8.56245) -- (6.89796,8.5991) -- (7.22449,8.5991) -- (7.22449,9.30464) -- (7.55102,9.30464) -- (7.55102,9.74904)
 -- (7.87755,9.74904) -- (7.87755,9.98728) -- (8.20408,9.98728) -- (8.20408,10.0766) -- (8.53061,10.0766) -- (8.53061,10.6699) -- (8.85714,10.6699) -- (8.85714,10.5852) -- (9.18367,10.5852) -- (9.18367,11.0135) -- (9.5102,11.0135) -- (9.5102,10.9792)
 -- (9.83673,10.9792) -- (9.83673,11.451) -- (10.1633,11.451) -- (10.1633,11.4258) -- (10.4898,11.4258) -- (10.4898,11.632) -- (10.8163,11.632) -- (10.8163,11.1922) -- (11.1429,11.1922) -- (11.1429,11.1097) -- (11.4694,11.1097) -- (11.4694,10.8715)
 -- (11.7959,10.8715) -- (11.7959,10.7226) -- (12.1224,10.7226) -- (12.1224,10.1957) -- (12.449,10.1957) -- (12.449,10.0377) -- (12.7755,10.0377) -- (12.7755,9.30464) -- (13.102,9.30464) -- (13.102,8.99998) -- (13.4286,8.99998) -- (13.4286,8.80298)
 -- (13.7551,8.80298) -- (13.7551,8.47541) -- (14.0816,8.47541) -- (14.0816,7.73092) -- (14.4082,7.73092) -- (14.4082,7.13076) -- (14.7347,7.13076) -- (14.7347,6.55808) -- (15.0612,6.55808) -- (15.0612,5.97853) -- (15.3878,5.97853) --
 (15.3878,5.57307) -- (15.7143,5.57307) -- (15.7143,4.97749) -- (16.0408,4.97749) -- (16.0408,4.53309) -- (16.3673,4.53309) -- (16.3673,4.05891) -- (16.6939,4.05891) -- (16.6939,3.40148) -- (17.0204,3.40148) -- (17.0204,3.04413) -- (17.3469,3.04413)
 -- (17.3469,2.14158) -- (17.6735,2.14158) -- (17.6735,1.83921) -- (18,1.83921);
\definecolor{c}{rgb}{1,1,1};
\draw [color=c, fill=c] (12.4,9.91934) rectangle (19.6,12.6185);
\definecolor{c}{rgb}{0,0,0};
\draw [c,line width=0.9] (12.4,9.91934) -- (19.6,9.91934);
\draw [c,line width=0.9] (19.6,9.91934) -- (19.6,12.6185);
\draw [c,line width=0.9] (19.6,12.6185) -- (12.4,12.6185);
\draw [c,line width=0.9] (12.4,12.6185) -- (12.4,9.91934);
\draw [anchor= west] (12.76,12.2811) node[scale=1.40004, color=c, rotate=0]{$\chi^{2} / ndf $};
\draw [anchor= east] (19.24,12.2811) node[scale=1.40004, color=c, rotate=0]{ 182.5 / 46};
\draw [anchor= west] (12.76,11.6063) node[scale=1.40004, color=c, rotate=0]{Constant };
\draw [anchor= east] (19.24,11.6063) node[scale=1.40004, color=c, rotate=0]{$  6433 \pm 18.5$};
\draw [anchor= west] (12.76,10.9315) node[scale=1.40004, color=c, rotate=0]{Mean     };
\draw [anchor= east] (19.24,10.9315) node[scale=1.40004, color=c, rotate=0]{$  1292 \pm 0.1$};
\draw [anchor= west] (12.76,10.2567) node[scale=1.40004, color=c, rotate=0]{Sigma    };
\draw [anchor= east] (19.24,10.2567) node[scale=1.40004, color=c, rotate=0]{$ 33.73 \pm 0.13$};
\definecolor{c}{rgb}{1,0,0};
\draw [c,line width=1.8] (2.08,1.46476) -- (2.24,1.68118) -- (2.4,1.90233) -- (2.56,2.12803) -- (2.72,2.35807) -- (2.88,2.59224) -- (3.04,2.83029) -- (3.2,3.07197) -- (3.36,3.31699) -- (3.52,3.56507) -- (3.68,3.81588) -- (3.84,4.06909) -- (4,4.32435)
 -- (4.16,4.58129) -- (4.32,4.83952) -- (4.48,5.09864) -- (4.64,5.35824) -- (4.8,5.61788) -- (4.96,5.87712) -- (5.12,6.1355) -- (5.28,6.39257) -- (5.44,6.64784) -- (5.6,6.90084) -- (5.76,7.15107) -- (5.92,7.39804) -- (6.08,7.64126) -- (6.24,7.88022)
 -- (6.4,8.11443) -- (6.56,8.34339) -- (6.72,8.56661) -- (6.88,8.7836) -- (7.04,8.99387) -- (7.2,9.19697) -- (7.36,9.39241) -- (7.52,9.57977) -- (7.68,9.75859) -- (7.84,9.92846) -- (8,10.089) -- (8.16,10.2398) -- (8.32,10.3804) -- (8.48,10.5107) --
 (8.64,10.6301) -- (8.8,10.7386) -- (8.96,10.8357) -- (9.12,10.9212) -- (9.28,10.995) -- (9.44,11.0568) -- (9.6,11.1064) -- (9.76,11.1439) -- (9.92,11.169);
\draw [c,line width=1.8] (9.92,11.169) -- (10.08,11.1817) -- (10.24,11.1819) -- (10.4,11.1697) -- (10.56,11.1452) -- (10.72,11.1082) -- (10.88,11.0591) -- (11.04,10.9978) -- (11.2,10.9245) -- (11.36,10.8395) -- (11.52,10.7429) -- (11.68,10.6349) --
 (11.84,10.5159) -- (12,10.3861) -- (12.16,10.2459) -- (12.32,10.0955) -- (12.48,9.93539) -- (12.64,9.76591) -- (12.8,9.58745) -- (12.96,9.40045) -- (13.12,9.20533) -- (13.28,9.00255) -- (13.44,8.79257) -- (13.6,8.57586) -- (13.76,8.35289) --
 (13.92,8.12416) -- (14.08,7.89016) -- (14.24,7.65139) -- (14.4,7.40834) -- (14.56,7.16152) -- (14.72,6.91141) -- (14.88,6.65852) -- (15.04,6.40334) -- (15.2,6.14633) -- (15.36,5.88799) -- (15.52,5.62878) -- (15.68,5.36915) -- (15.84,5.10954) --
 (16,4.85039) -- (16.16,4.59211) -- (16.32,4.33511) -- (16.48,4.07978) -- (16.64,3.82647) -- (16.8,3.57555) -- (16.96,3.32736) -- (17.12,3.0822) -- (17.28,2.84037) -- (17.44,2.60216) -- (17.6,2.36783) -- (17.76,2.13761);
\draw [c,line width=1.8] (17.76,2.13761) -- (17.92,1.91172);
\definecolor{c}{rgb}{0,0,0};
\draw [c,line width=0.9] (2,1.34957) -- (18,1.34957);
\draw [c,line width=0.9] (3.30612,1.67347) -- (3.30612,1.34957);
\draw [c,line width=0.9] (3.63265,1.51152) -- (3.63265,1.34957);
\draw [c,line width=0.9] (3.95918,1.51152) -- (3.95918,1.34957);
\draw [c,line width=0.9] (4.28571,1.51152) -- (4.28571,1.34957);
\draw [c,line width=0.9] (4.61225,1.51152) -- (4.61225,1.34957);
\draw [c,line width=0.9] (4.93878,1.67347) -- (4.93878,1.34957);
\draw [c,line width=0.9] (5.26531,1.51152) -- (5.26531,1.34957);
\draw [c,line width=0.9] (5.59184,1.51152) -- (5.59184,1.34957);
\draw [c,line width=0.9] (5.91837,1.51152) -- (5.91837,1.34957);
\draw [c,line width=0.9] (6.2449,1.51152) -- (6.2449,1.34957);
\draw [c,line width=0.9] (6.57143,1.67347) -- (6.57143,1.34957);
\draw [c,line width=0.9] (6.89796,1.51152) -- (6.89796,1.34957);
\draw [c,line width=0.9] (7.22449,1.51152) -- (7.22449,1.34957);
\draw [c,line width=0.9] (7.55102,1.51152) -- (7.55102,1.34957);
\draw [c,line width=0.9] (7.87755,1.51152) -- (7.87755,1.34957);
\draw [c,line width=0.9] (8.20408,1.67347) -- (8.20408,1.34957);
\draw [c,line width=0.9] (8.53061,1.51152) -- (8.53061,1.34957);
\draw [c,line width=0.9] (8.85714,1.51152) -- (8.85714,1.34957);
\draw [c,line width=0.9] (9.18367,1.51152) -- (9.18367,1.34957);
\draw [c,line width=0.9] (9.5102,1.51152) -- (9.5102,1.34957);
\draw [c,line width=0.9] (9.83673,1.67347) -- (9.83673,1.34957);
\draw [c,line width=0.9] (10.1633,1.51152) -- (10.1633,1.34957);
\draw [c,line width=0.9] (10.4898,1.51152) -- (10.4898,1.34957);
\draw [c,line width=0.9] (10.8163,1.51152) -- (10.8163,1.34957);
\draw [c,line width=0.9] (11.1429,1.51152) -- (11.1429,1.34957);
\draw [c,line width=0.9] (11.4694,1.67347) -- (11.4694,1.34957);
\draw [c,line width=0.9] (11.7959,1.51152) -- (11.7959,1.34957);
\draw [c,line width=0.9] (12.1224,1.51152) -- (12.1224,1.34957);
\draw [c,line width=0.9] (12.449,1.51152) -- (12.449,1.34957);
\draw [c,line width=0.9] (12.7755,1.51152) -- (12.7755,1.34957);
\draw [c,line width=0.9] (13.102,1.67347) -- (13.102,1.34957);
\draw [c,line width=0.9] (13.4286,1.51152) -- (13.4286,1.34957);
\draw [c,line width=0.9] (13.7551,1.51152) -- (13.7551,1.34957);
\draw [c,line width=0.9] (14.0816,1.51152) -- (14.0816,1.34957);
\draw [c,line width=0.9] (14.4082,1.51152) -- (14.4082,1.34957);
\draw [c,line width=0.9] (14.7347,1.67347) -- (14.7347,1.34957);
\draw [c,line width=0.9] (15.0612,1.51152) -- (15.0612,1.34957);
\draw [c,line width=0.9] (15.3878,1.51152) -- (15.3878,1.34957);
\draw [c,line width=0.9] (15.7143,1.51152) -- (15.7143,1.34957);
\draw [c,line width=0.9] (16.0408,1.51152) -- (16.0408,1.34957);
\draw [c,line width=0.9] (16.3673,1.67347) -- (16.3673,1.34957);
\draw [c,line width=0.9] (16.6939,1.51152) -- (16.6939,1.34957);
\draw [c,line width=0.9] (17.0204,1.51152) -- (17.0204,1.34957);
\draw [c,line width=0.9] (17.3469,1.51152) -- (17.3469,1.34957);
\draw [c,line width=0.9] (17.6735,1.51152) -- (17.6735,1.34957);
\draw [c,line width=0.9] (18,1.67347) -- (18,1.34957);
\draw [c,line width=0.9] (3.30612,1.67347) -- (3.30612,1.34957);
\draw [c,line width=0.9] (2.97959,1.51152) -- (2.97959,1.34957);
\draw [c,line width=0.9] (2.65306,1.51152) -- (2.65306,1.34957);
\draw [c,line width=0.9] (2.32653,1.51152) -- (2.32653,1.34957);
\draw [c,line width=0.9] (2,1.51152) -- (2,1.34957);
\draw [anchor=base] (3.30612,0.904212) node[scale=1.01821, color=c, rotate=0]{1250};
\draw [anchor=base] (4.93878,0.904212) node[scale=1.01821, color=c, rotate=0]{1260};
\draw [anchor=base] (6.57143,0.904212) node[scale=1.01821, color=c, rotate=0]{1270};
\draw [anchor=base] (8.20408,0.904212) node[scale=1.01821, color=c, rotate=0]{1280};
\draw [anchor=base] (9.83673,0.904212) node[scale=1.01821, color=c, rotate=0]{1290};
\draw [anchor=base] (11.4694,0.904212) node[scale=1.01821, color=c, rotate=0]{1300};
\draw [anchor=base] (13.102,0.904212) node[scale=1.01821, color=c, rotate=0]{1310};
\draw [anchor=base] (14.7347,0.904212) node[scale=1.01821, color=c, rotate=0]{1320};
\draw [anchor=base] (16.3673,0.904212) node[scale=1.01821, color=c, rotate=0]{1330};
\draw [anchor=base] (18,0.904212) node[scale=1.01821, color=c, rotate=0]{1340};
\draw [c,line width=0.9] (2,1.34957) -- (2,12.1461);
\draw [c,line width=0.9] (2.48,2.17365) -- (2,2.17365);
\draw [c,line width=0.9] (2.24,2.40273) -- (2,2.40273);
\draw [c,line width=0.9] (2.24,2.6318) -- (2,2.6318);
\draw [c,line width=0.9] (2.24,2.86087) -- (2,2.86087);
\draw [c,line width=0.9] (2.24,3.08994) -- (2,3.08994);
\draw [c,line width=0.9] (2.48,3.31901) -- (2,3.31901);
\draw [c,line width=0.9] (2.24,3.54808) -- (2,3.54808);
\draw [c,line width=0.9] (2.24,3.77715) -- (2,3.77715);
\draw [c,line width=0.9] (2.24,4.00623) -- (2,4.00623);
\draw [c,line width=0.9] (2.24,4.2353) -- (2,4.2353);
\draw [c,line width=0.9] (2.48,4.46437) -- (2,4.46437);
\draw [c,line width=0.9] (2.24,4.69344) -- (2,4.69344);
\draw [c,line width=0.9] (2.24,4.92251) -- (2,4.92251);
\draw [c,line width=0.9] (2.24,5.15158) -- (2,5.15158);
\draw [c,line width=0.9] (2.24,5.38065) -- (2,5.38065);
\draw [c,line width=0.9] (2.48,5.60972) -- (2,5.60972);
\draw [c,line width=0.9] (2.24,5.8388) -- (2,5.8388);
\draw [c,line width=0.9] (2.24,6.06787) -- (2,6.06787);
\draw [c,line width=0.9] (2.24,6.29694) -- (2,6.29694);
\draw [c,line width=0.9] (2.24,6.52601) -- (2,6.52601);
\draw [c,line width=0.9] (2.48,6.75508) -- (2,6.75508);
\draw [c,line width=0.9] (2.24,6.98415) -- (2,6.98415);
\draw [c,line width=0.9] (2.24,7.21322) -- (2,7.21322);
\draw [c,line width=0.9] (2.24,7.4423) -- (2,7.4423);
\draw [c,line width=0.9] (2.24,7.67137) -- (2,7.67137);
\draw [c,line width=0.9] (2.48,7.90044) -- (2,7.90044);
\draw [c,line width=0.9] (2.24,8.12951) -- (2,8.12951);
\draw [c,line width=0.9] (2.24,8.35858) -- (2,8.35858);
\draw [c,line width=0.9] (2.24,8.58765) -- (2,8.58765);
\draw [c,line width=0.9] (2.24,8.81672) -- (2,8.81672);
\draw [c,line width=0.9] (2.48,9.04579) -- (2,9.04579);
\draw [c,line width=0.9] (2.24,9.27487) -- (2,9.27487);
\draw [c,line width=0.9] (2.24,9.50394) -- (2,9.50394);
\draw [c,line width=0.9] (2.24,9.73301) -- (2,9.73301);
\draw [c,line width=0.9] (2.24,9.96208) -- (2,9.96208);
\draw [c,line width=0.9] (2.48,10.1912) -- (2,10.1912);
\draw [c,line width=0.9] (2.24,10.4202) -- (2,10.4202);
\draw [c,line width=0.9] (2.24,10.6493) -- (2,10.6493);
\draw [c,line width=0.9] (2.24,10.8784) -- (2,10.8784);
\draw [c,line width=0.9] (2.24,11.1074) -- (2,11.1074);
\draw [c,line width=0.9] (2.48,11.3365) -- (2,11.3365);
\draw [c,line width=0.9] (2.48,2.17365) -- (2,2.17365);
\draw [c,line width=0.9] (2.24,1.94458) -- (2,1.94458);
\draw [c,line width=0.9] (2.24,1.71551) -- (2,1.71551);
\draw [c,line width=0.9] (2.24,1.48644) -- (2,1.48644);
\draw [c,line width=0.9] (2.48,11.3365) -- (2,11.3365);
\draw [c,line width=0.9] (2.24,11.5656) -- (2,11.5656);
\draw [c,line width=0.9] (2.24,11.7947) -- (2,11.7947);
\draw [c,line width=0.9] (2.24,12.0237) -- (2,12.0237);
\draw [anchor= east] (1.9,2.17365) node[scale=1.01821, color=c, rotate=0]{2500};
\draw [anchor= east] (1.9,3.31901) node[scale=1.01821, color=c, rotate=0]{3000};
\draw [anchor= east] (1.9,4.46437) node[scale=1.01821, color=c, rotate=0]{3500};
\draw [anchor= east] (1.9,5.60972) node[scale=1.01821, color=c, rotate=0]{4000};
\draw [anchor= east] (1.9,6.75508) node[scale=1.01821, color=c, rotate=0]{4500};
\draw [anchor= east] (1.9,7.90044) node[scale=1.01821, color=c, rotate=0]{5000};
\draw [anchor= east] (1.9,9.04579) node[scale=1.01821, color=c, rotate=0]{5500};
\draw [anchor= east] (1.9,10.1912) node[scale=1.01821, color=c, rotate=0]{6000};
\draw [anchor= east] (1.9,11.3365) node[scale=1.01821, color=c, rotate=0]{6500};
\definecolor{c}{rgb}{1,1,1};
\draw [color=c, fill=c] (12.4,9.91934) rectangle (19.6,12.6185);
\definecolor{c}{rgb}{0,0,0};
\draw [c,line width=0.9] (12.4,9.91934) -- (19.6,9.91934);
\draw [c,line width=0.9] (19.6,9.91934) -- (19.6,12.6185);
\draw [c,line width=0.9] (19.6,12.6185) -- (12.4,12.6185);
\draw [c,line width=0.9] (12.4,12.6185) -- (12.4,9.91934);
\draw [anchor= west] (12.76,12.2811) node[scale=1.40004, color=c, rotate=0]{$\chi^{2} / ndf $};
\draw [anchor= east] (19.24,12.2811) node[scale=1.40004, color=c, rotate=0]{ 182.5 / 46};
\draw [anchor= west] (12.76,11.6063) node[scale=1.40004, color=c, rotate=0]{Constant };
\draw [anchor= east] (19.24,11.6063) node[scale=1.40004, color=c, rotate=0]{$  6433 \pm 18.5$};
\draw [anchor= west] (12.76,10.9315) node[scale=1.40004, color=c, rotate=0]{Mean     };
\draw [anchor= east] (19.24,10.9315) node[scale=1.40004, color=c, rotate=0]{$  1292 \pm 0.1$};
\draw [anchor= west] (12.76,10.2567) node[scale=1.40004, color=c, rotate=0]{Sigma    };
\draw [anchor= east] (19.24,10.2567) node[scale=1.40004, color=c, rotate=0]{$ 33.73 \pm 0.13$};
\end{tikzpicture}
 

\FloatBarrier
\newpage

\input{sections/appendix/PLACEHOLDER.txt}
\FloatBarrier
\newpage
\clearpage

%\begin{thebibliography}{9}

\end{thebibliography}


\input{./sections/code.tex}



\end{document}
