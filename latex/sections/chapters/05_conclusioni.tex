L'esperimento si prefiggeva di andare a studiare due proprietà del positronio: la probabilità di creare orto o para positronio e lo studio del decadimento degli stessi.\\

Per quanto riguarda la prima parte, il rapporto trovato si avvicina molto a quello stimato dalla QED, e considerando tutti i limiti
sperimentali dell'apparato e tutte le approssimazioni fatte in ambito della stima teorica dell'angolo solido si è soddisfatti del risultato che, sebbene non coincida esattamente con quello teorico, è dello stesso ordine di grandezza (la differenza si attesta attorno ad un fattore 2). Per quanto riguarda lo
studio del decadimento, invece, si è dispiaciuti di non essere riusciti a interpolare correttamente il valore della costante di decadimento dell'ortopositronio. Probabilmente
si riuscirebbe in questo intento andando a prendere un campione più grande di dati (nonostante le 20 ore il campione si è rivelato troppo esiguo per una buona statistica);
ciò nonostante si è contenti di aver visualizzato, una volta pulito opportunamente il campione, che il tempo di decadimento dell'ortopositronio è effettivamente maggiore, che si vede dal fatto che la gaussiana presenta una coda, che non si riesce a interpolare. Questo non è possibile perché i parametri del fit esponenziale
sono troppo sensibili all'intervallo scelto per il fit (in particolare avvicinandosi alla gaussiana la costante di decadimento aumenta senza inficiare  significativamente la qualità del fit). Inoltre si è notato che la pulizia dei dati risulta buona (i picchi in energia sono molto ben isolati), però lo spettro temporale legato a tale filtro però non presenta un'evidente coda esponenziale, quanto semplicemente dei dati fuori dal picco.\\

La stima teorica risulta in linea con quanto visto tramite la simulazione Monte Carlo. L'efficienza intrinseca dei rivelatori è stata stimata nel miglior modo possibile per correggere il rapporto tra le rate e non si ritiene particolarmente attendibile.
